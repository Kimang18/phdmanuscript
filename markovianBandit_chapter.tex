\chapter{Markovian Bandit Problem}
\label{ch:mab_problem}

In this chapter, we present two structured MDPs known as rested and restless markovian bandit.

\section{Rested Markovian bandit}
\label{sec:rested_mab_pb}

Motivating the problem:
\begin{itemize}
    \item a generalization of multi-armed bandit problem?
    \item give examples
\end{itemize}

\subsection{Problem formulation}
\label{subsec:rested_pb_formul}

\begin{itemize}
    \item explain a Markovian bandit: $n$ arms denote by $M$, Arm $i$ is $\langle\gS_i, \vr_i,\mP_i\rangle$
    \item state $\vx=(x_1\ \dots\ x_n)$ and $\vx_t=(x_{1,t}\ \dots\ x_{n,t})$
    \item reward $r(\vx, a)=r_a(x_a)=r(x_a)$
    \item transition $P(\vy \mid\vx,a)=P_a(x_a,y_a)=P(x_a,y_a)$
    \item objective function: maximize the value function $\max_{\pi}V^\pi_M$ 
\end{itemize}

\subsection{Gittins Index Policy}
\label{subsec:gittins_idx}

\subsubsection{Definition}

\subsubsection{Optimality}

\subsubsection{Value of Gittins index policy}

Do we need this section?

\subsubsection{Upper bound on the number of switching}

\begin{itemize}
    \item Prove the upper bound
    \item applicability: switching cost?
\end{itemize}

\subsection{Gittins index computation}

\subsubsection{Largest index remaining}

\begin{itemize}
    \item algorithm explanation
    \item complexity
    \item pitfall?
\end{itemize}

\subsubsection{State elimination}

\begin{itemize}
    \item algorithm explanation
    \item complexity
    \item pitfall?
\end{itemize}

\section{Restless Markovian bandit}
\label{sec:restless_mab_pb}

Motivating the problem:
\begin{itemize}
    \item a generalization of rested bandit problem?
    \item give examples
\end{itemize}

\subsection{Problem formulation}
\label{subsec:restless_pb_formul}

\begin{itemize}
    \item explain a restless bandit: $n$ arms denote by $M$, Arm $i$ is $\langle\gS_i, \{0,1\}, r_i,P_i\rangle$
    \item state $\vx=(x_1\ \dots\ x_n)$ and $\vx_t=(x_{1,t}\ \dots\ x_{n,t})$
    \item action $\va=(a_1\ \dots\ a_n)$ such that $\sum_{i=1}^{n}a_i =m$
    \item reward $r(\vx, \va)=\sum_{i=1}^n r_i(x_i,a_i) =\sum_{i=1}^n r(x_i,a_i)$
    \item transition $P(\vy \mid\vx,a)=\prod_{i=1}^nP_i(y_i \mid x_i,a_i)=\prod_{i=1}^n P(y_i \mid x_i,a_i)$
    \item objective function: maximize the average reward $\max_{\pi}g^\pi_M$ 
\end{itemize}


\subsection{Whittle Index Policy}
\label{subsec:whittle_idx}

\subsubsection{Definition}

Lagrangian multipliers

\paragraph{$\lambda$-penalized MDP}

We are given an arm $\langle\gS,\{0,1\},r,P\rangle$. In $\lambda$-penalized MDP, consider the Bellman optimality equations: for each $i\in\gS$,
\begin{align}
    \max_{a\in\{0,1\}}{\left\{\sum_jP^a_{ij}g_j-g_i\right\}} =0 \label{eq:gain_max}\\
\max_{a\in\{0,1\}}{\left\{r^a_i-\lambda a-g_i +\sum_jP^a_{ij}h_j-h_i\right\}} =0 \label{eq:bias_max}
\end{align}

This system uniquely determined $\vg\in\sR^n$.
Let $\vg^*$ is be the solution of this system. Given $\vg^*$, let
\begin{align}
H:=\{\vh\in \sR^n \mid \vh \text{ satisfies } \eqref{eq:bias_max}\}
\end{align}
be the space of solutions of \eqref{eq:bias_max}.
For any $\vh\in H$, let $b(\vh):=r^1-r^0-\lambda \vone +(P^1-P^0)\vh$. 
Note that by [Theorem 3.1, Schweitzer, 1978], the space $H$ is nonempty: $\vh^*:=\max_d\{Z^d(r^d -\lambda \vd -\vg^*)\}$ where $Z^d:=(I-P^d +\bar{P}^d)^{-1}$ and $\vh^*\in H$.

\begin{defn}
\label{defn:BO_policy}
A policy $\pi$ is Bellman optimal if there exists $\vh\in H$ such that $\pi=\argmax_d\{r^d -\lambda \vd +P^d\vh\}$ and $P^\pi\vg^*=\vg^*$.
\end{defn}

Moreover, we also have the following lemma
\begin{lem}
    \label{lem:policy_improve}
    Let $\vg^\pi$ and $\vh^\pi$ be a solution of the following system
    \begin{align}
        (I-P^\pi)\vg^\pi &=\vzero \label{eq:gain_eval}\\
        \vg^\pi +(I-P^\pi)\vh^\pi &= r^\pi -\lambda\vpi \label{eq:bias_eval}.
    \end{align}
    \begin{enumerate}
        \item $\vg^\pi$ is uniquely determined by the system while $\vh^\pi$ is unique up to an element of the null space of $(I-P^\pi)$.
        \item $\pi$ is Bellman optimal if and only if $P^\pi\vg^*=\vg^*$ and
            \begin{align*}
                r^1_i-\lambda +\sum_{j}P^1_{ij}h^\pi_j &\ge r^0_i +\sum_{j}P^0_{ij}h^\pi_j \text{ for all } i\in\pi \text{ and } \\
                r^1_i-\lambda +\sum_{j}P^1_{ij}h^\pi_j &\le r^0_i +\sum_{j}P^0_{ij}h^\pi_j \text{ for all } i\notin\pi.
        \end{align*}
    \end{enumerate}
\end{lem}
The difference between Lemma~\ref{lem:policy_improve} and Definition~\ref{defn:BO_policy} is that in Definition~\ref{defn:BO_policy}, the policy $\pi$ satisfies the maximum in Equation~\eqref{eq:bias_max} for some $\vh\in H$ while in Lemma~\ref{lem:policy_improve}, the policy $\pi$ cannot be improved over its own bias $\vh^\pi$.

We are now interested in conditions for unicity of Bellman optimal policy.
Let $\Pi^*$ be the set of all Bellman optimal policies. 

\begin{lem}
\label{lem:unicity_BO}
Given a $\lambda$-penalized MDP, there exists only one Bellman optimal policy, $|\Pi^*|=1$, if for any pair $\vh, \tilde{\vh}\in H, b(\vh)_i\times b(\tilde{\vh})_i>0$ for all $i\in[n]$.
\end{lem}
\begin{proof}
    TODO
\end{proof}

\subsubsection{Optimality}

\subsection{Whittle index computation}

\subsubsection{Conditions for indexability...}

\subsubsection{Fast-pivoting algorithm}

