\begingroup
\let\clearpage\relax

\chapter{Markovian Bandit Problem}
\label{ch:mb}

In this chapter, we present two settings of markovian bandit problem: rested and restless.
In each setting, we explain the objective function and...

\section{Rested Markovian bandit}
\label{sec:rested_mab}

Rested markovian bandit is a model for dynamic allocation problem in which the decision maker wants to efficiently share its limited resource between various activities which are being pursued.

%Motivating the problem:
%\begin{itemize}
%    \item a generalization of multi-armed bandit problem?
%    \item give examples
%\end{itemize}

\subsection{Problem formulation}
\label{subsec:rested_formul}

We consider the rested markovian bandit having $n\in\N^+$ arms.
Each arm $\langle\gS_i, \vr_i, \mP_i\rangle$ for $i\in\{1,\dots,n\}=:[n]$ is a Markov reward process with a finite state space $\gS_i$, a mean reward vector $\vr_i\in[0,1]^{\abs{\gS_i}}$ and a stochastic matrix $\mP_i$.
In state $s_i\in\gS_i$, the arm $i$ incurs a random reward with expected value $r_i(s_i)$ and transitions to next state $s'_i$ with probability $P_i(s_i,s'_i)$.
Without loss of generality, we assume that the state space of the arms are pairwise distinct: $\gS_i\cap\gS_j=\emptyset$ for any $i\neq j$.
The state of arm $i$ will be denoted with an index $i$: we will denote such a state by $s_i$ or $s'_i$.
As state spaces are disjoint, this allows us to simplify the notation by dropping the index $i$ from expected reward and transition probability: when convenient, we will denote them by $r(s_i)$ instead of $r_i(s_i)$ and by $P(s_i,s'_i)$ instead of $P_i(s_i,s'_i)$ since no confusion is possible.

We consider the following sequential problem: At time step $1$, the state of all arms denoted by $\vs_1:=(s_{1,1},\dots,s_{1,n})$ is sampled according to some initial distribution $\rho$ over the state space $\gX:=\gS_1\times\dots\times\gS_n$.
At time step $t$, the decision maker observes the current state of all arms denoted by $\vs_t:=(s_{t,1},\dots,s_{t,n})$ and activates one arm $a_t\in\{1,\dots,n\}=:[n]$.
The activated arm incurs a random reward discounted like $\gamma^{t-1}r_t$ where $\gamma\in(0,1)$ is the discount factor.
%and $r_t$ is drawn from some distribution on $[0,1]$ with expected value $r(x_{t,a_t})$.
After that, the activated arm transitions to new state. %$s_{t+1,a_t}$ with probability $P(s_{t,a_t}, s_{t,a_t})$.
The other arms incur no rewards and make no state transitions.
The goal of the decision maker is to identify a policy $\pi:\gS\mapsto[n]$ that maximizes the expected cumulative discounted reward as defined in Section~\ref{ch:mdp:sec:discounted}:
\begin{equation}
    \label{ch:mb:value}
    v_\gamma^\pi(\vs):=\E^\pi\Bigl[\sum_{t=1}^{+\infty} \gamma^{t-1}r_t \mid \vs_1=\vs\Bigr]
\end{equation}
This is an infinite horizon discounted problem in which the rested markovian bandit is a specific instance of MDP -- that we denote by $M$ -- whose state space is $\gS$ and action space is $\gA:=[n]$.
If the decision maker activates an arm $a\in[n]$ when $M$ is in state $\vs=(s_1,\dots,s_n)$, then $M$ incurs a random reward with expected value $r(s_a)$ and transitions to new state $\vs'=(s'_1,\dots,s'_n)$ with probability $p(\vs'\mid \vs, a)$ that satisfies
\begin{equation}
    \label{ch:mb:defP}
    p(\vs'\mid \vs, a)
    = 
    \left\{
        \begin{array}{ll}
            P(s_a,s'_a) & \text{ if $s_b=s'_b$ for all $b\ne a$;}\\
            0 &\text{ otherwise}.
        \end{array}
    \right.
\end{equation}
%TODO:continue

\subsection{Gittins Index Policy}
\label{subsec:gittins_idx}

\subsubsection{Definition}

\subsubsection{Optimality}

\subsubsection{Value of Gittins index policy}

Do we need this section?

\subsubsection{Upper bound on the number of switching}

\begin{itemize}
    \item Prove the upper bound
    \item applicability: switching cost?
\end{itemize}

\subsection{Gittins index computation}

\subsubsection{Largest index remaining}

\begin{itemize}
    \item algorithm explanation
    \item complexity
    \item pitfall?
\end{itemize}

\subsubsection{State elimination}

\begin{itemize}
    \item algorithm explanation
    \item complexity
    \item pitfall?
\end{itemize}

\section{Restless Markovian bandit}
\label{sec:restless_mab_pb}

Motivating the problem:
\begin{itemize}
    \item a generalization of rested bandit problem?
    \item give examples
\end{itemize}

\subsection{Problem formulation}
\label{subsec:restless_pb_formul}

\begin{itemize}
    \item explain a restless bandit: $n$ arms denote by $M$, Arm $i$ is $\langle\gS_i, \{0,1\}, r_i,P_i\rangle$
    \item state $\vx=(x_1\ \dots\ x_n)$ and $\vx_t=(x_{1,t}\ \dots\ x_{n,t})$
    \item action $\va=(a_1\ \dots\ a_n)$ such that $\sum_{i=1}^{n}a_i =m$
    \item reward $r(\vx, \va)=\sum_{i=1}^n r_i(x_i,a_i) =\sum_{i=1}^n r(x_i,a_i)$
    \item transition $P(\vy \mid\vx,a)=\prod_{i=1}^nP_i(y_i \mid x_i,a_i)=\prod_{i=1}^n P(y_i \mid x_i,a_i)$
    \item objective function: maximize the average reward $\max_{\pi}g^\pi_M$ 
\end{itemize}


\subsection{Whittle Index Policy}
\label{subsec:whittle_idx}

\subsubsection{Definition}

Lagrangian multipliers

\subsubsection{Optimality}

\subsection{Whittle index computation}

\subsubsection{Conditions for indexability...}

\subsubsection{Fast-pivoting algorithm}


\section{Open questions}

\endgroup
