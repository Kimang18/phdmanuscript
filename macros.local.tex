% *** SPECIFIC MACROS ***
% specific macro definition for the paper
%
% requires amsthm, xspace

% theorem-like environment
\newtheorem{thm}{Theorem}
\newtheorem{lem}[thm]{Lemma}
\newtheorem{prop}{Proposition}
\newtheorem{propty}{Property}
\newtheorem{defn}{Definition}
\newtheorem{cor}[thm]{Corollary}
\newtheorem{conj}[thm]{Conjecture}
\newtheorem{asmp}[thm]{Assumption}

% asymptotic complexity (Landau) notations
\newcommand{\landauO}{\ensuremath{\mathcal{O}}\xspace}
\newcommand{\landauOmega}{\ensuremath{\Omega}\xspace}
\newcommand{\landauTheta}{\ensuremath{\Theta}\xspace}
\newcommand{\landauo}{\ensuremath{o}\xspace}
\newcommand{\landauomega}{\ensuremath{\omega}\xspace}
\newcommand{\landauorder}{\ensuremath{\sim}\xspace}

% scheduling notations
\newcommand{\graham}[3]{\mbox{\ensuremath{#1\mid#2\mid#3}}}
\newcommand{\Cmax}{\ensuremath{C_{\max}}\xspace}

% complexity classes
\newcommand{\cNP}{\textbf{NP}\xspace}  % bold notations
\newcommand{\cP}{\textbf{P}\xspace}
% \newcommand{\cNP}{\ensuremath{\mathcal{N\!P}}\xspace}  % round notations
% \newcommand{\cP}{\ensuremath{\mathcal{P}}\xspace}

% Machine states
\newcommand{\computing}{computing}
\newcommand{\idle}{idle}
\newcommand{\off}{of\!f}
\newcommand{\on}{on}
\newcommand{\ontooff}{\on\rightarrow\off}
\newcommand{\offtoon}{\off\rightarrow\on}

% CCGRID 2017
\newcommand{\ra}[1]{\renewcommand{\arraystretch}{#1}}

% Cluster 2017
\newcommand{\overbar}[1]{\mkern 1.5mu\overline{\mkern-1.5mu#1\mkern-1.5mu}\mkern 1.5mu}

\newlength\mylen % algorithm2e hack
\newcommand\myinput[1]{%
  \settowidth\mylen{\KwIn{}}%
  \setlength\hangindent{\mylen}%
  \hspace*{\mylen}#1\\}

\newcommand\myoutput[1]{% algorithm2e hack
  \settowidth\mylen{\KwOut{}}%
  \setlength\hangindent{\mylen}%
  \hspace*{\mylen}#1\\}


% Float environments
\renewcommand{\topfraction}{0.8}
\renewcommand{\bottomfraction}{0.8}



%%%%%%%%%%%%%%%%%%%%%%%
% Modular compilation %
%%%%%%%%%%%%%%%%%%%%%%%
\newif\ifwatermark
\watermarkfalse

\newif\iftotalcompilation
\totalcompilationtrue

\newcommand{\inputchapter}[2]{%
  \ifdef{#1}%
    {\input{#2}\cleardoublepage}%
    {}%
}

% Modeling notations
\newcommand{\model}[2][]{\ensuremath{\bm{M}_{#1}\ifthenelse{\equal{#2}{}}{}{\!-\!}{#2}}\xspace}
\newcommand{\modelp}[2][]{\ensuremath{\bm{M'}_{#1}\!\ifthenelse{\equal{#2}{}}{}{\!-\!}{#2}}\xspace}
\newcommand{\noise}[2][]{\ensuremath{\bm{N}_{#1}\ifthenelse{\equal{#2}{}}{}{\!-\!}{#2}}\xspace}
\newcommand{\noisep}[2][]{\ensuremath{\bm{N'}_{#1}\!\ifthenelse{\equal{#2}{}}{}{\!-\!}{#2}}\xspace}
\newcommand{\norm}{\ensuremath{\mathcal{N}}\xspace}
\newcommand{\mcdots}{\ensuremath{\!\cdot\!\cdot\!\cdot\!}\xspace}
\newcommand{\unif}[2]{\ensuremath{\mathcal{U}\left({#1},{#2}\right)}\xspace}

% Abreviations
\newcommand{\eg}{e.g.\@\xspace}
\newcommand{\ie}{i.e.\@\xspace}
\newcommand{\aka}{a.k.a.\@\xspace}
\newcommand{\resp}{resp.\@\xspace}
\newcommand{\etal}{et~al.\@\xspace}
\newcommand{\dgemm}{\texttt{dgemm}\@\xspace}
\newcommand{\recv}{\texttt{MPI\_Recv}\@\xspace}
\newcommand{\send}{\texttt{MPI\_Send}\@\xspace}
\newcommand{\isend}{\texttt{MPI\_Isend}\@\xspace}
\newcommand{\iprobe}{\texttt{MPI\_Iprobe}\@\xspace}
\newcommand{\pyce}{\texttt{pycewise}\@\xspace}
\newcommand{\peanut}{\texttt{peanut}\@\xspace}
\newcommand{\cashew}{\texttt{cashew}\@\xspace}
\newcommand{\execo}{\texttt{execo}\@\xspace}
\newcommand{\enoslib}{\texttt{EnOSlib}\@\xspace}
\newcommand{\cluster}[2][]{\textcolor{myblue}{\texttt{#2\ifthenelse{\equal{#1}{}}{}{-#1}}}\@\xspace}
\newcommand{\dahu}[1][]{\cluster[#1]{dahu}}
\newcommand{\dada}[1]{\cluster[#1]{dahu}}  % for some reason, optionnal arguments do not work in subfigure, so I need this crap...
\newcommand{\drac}{\cluster{drac}}
\newcommand{\yeti}[1][]{\cluster[#1]{yeti}}
\newcommand{\troll}[1][]{\cluster[#1]{troll}}
\newcommand{\gros}{\cluster{gros}}
\newcommand{\grisou}[1][]{\cluster[#1]{grisou}}
\newcommand{\grvingt}{\cluster{grvingt}}
\newcommand{\paravance}{\cluster{paravance}}
\newcommand{\parasilo}[1][]{\cluster[#1]{parasilo}}
\newcommand{\paranoia}{\cluster{paranoia}}
\newcommand{\chiclet}{\cluster{chiclet}}
\newcommand{\chetemi}{\cluster{chetemi}}
\newcommand{\pyxis}{\cluster{pyxis}}
\newcommand{\nova}{\cluster{nova}}
\newcommand{\taurus}{\cluster{taurus}}
\newcommand{\ecotype}[1][]{\cluster[#1]{ecotype}}

% Referencing several times a footnote
% See https://tex.stackexchange.com/a/54240
\newcommand{\savefootnote}[2]{\footnote{\label{#1}#2}}
\newcommand{\repeatfootnote}[1]{\textsuperscript{\ref{#1}}}


% MDP Notation
\newcommand{\StateSpace}{\mathcal{S}}       % MC state space -->
\newcommand{\sSpace}{\mathcal{S}}       % MC state space -->
\newcommand{\nStateSpace}{S}                % state space size
\newcommand{\sSize}{S}                % state space size
\newcommand{\mdpTran}{p}                       % mdp transition kernel -->
\newcommand{\mdpRew}{r}                        % mdp reward function -->
\newcommand{\tran}{P}                       % MC transition kernel -->
\newcommand{\rew}{\mathbf{r}}                        % MC reward function -->
\newcommand{\pol}{\pi}                      % policy -->
\newcommand{\vfunc}{v}                      % value function -->
\newcommand{\qfunc}{q}                     % action-value function -->
\newcommand{\SC}{SC}                        % sample complexity -->
\newcommand{\Algo}{\mathfrak{A}}            % algorithm used -->
\newcommand{\Reg}{\mathrm{Regret}}              % regret -->
\newcommand{\BayReg}{\mathrm{BayesRegret}}              % bayesian regret -->
\newcommand{\nbStates}{k}                   % number of states in each Markovian bandit -->
\newcommand{\nbBandits}{n}                  % number of Markovian bandits -->
\newcommand{\state}{x}                      % Markovian bandit state -->
\newcommand{\statePrime}{y}                % Markovian bandit next state -->
\newcommand{\mdpModel}{\mathcal{M}}         % MDP Model
\newcommand{\mdpStateSpace}{\mathcal{E}}    % MDP state space

\newcommand{\mSpace}{\mathcal{E}}    % MDP state space
\newcommand{\mSize}{E}    % MDP state space
\newcommand{\ActionSpace}{\mathcal{A}}      % MDP action space -->
\newcommand{\aSpace}{\mathcal{A}}      % MDP action space -->

\newcommand{\Action}{a}                     % MDP action -->
\newcommand{\ActionPrime}{b}                % MDP next action -->
\newcommand{\mdpState}{\mathbf{x}}                   % MDP state -->
\newcommand{\mdpStatePrime}{\mathbf{y}}              % MDP next state -->
\newcommand{\discount}{\beta}               % MDP discount factor -->
\newcommand{\bandit}{\mathcal{B}}           % bandit symbol -->
\newcommand{\GIndex}{\gamma}                % Gittins Index symbol -->
\newcommand{\horizon}{H}                    % episode's length in FiniteHorizon setting -->
\newcommand{\nbEpisodes}{K}                 % the total number of episodes
% Estimate symbols
\newcommand{\estTran}{\hat{\tran}}          % estimated transition
\newcommand{\estRew}{\hat{\rew}}            % estimated reward
\newcommand{\estMdp}{\hat{\mdpModel}}       % estimated MDP Model
\newcommand{\estPol}{\hat{\pol}}       % estimated MDP Model
% Random Variables
\newcommand{\RVmdpState}{X}                 % MDP state random variable
\newcommand{\RVmdpStatePrime}{Y}                 % MDP state random variable
\newcommand{\RVaction}{A}                 % MDP state random variable
\newcommand{\RVstate}{X}                    % Markovian bandit state random variable
\newcommand{\RVrew}{R}                      % reward random variable
\newcommand{\RVvfunc}{V}                    % reward random variable
\newcommand{\RVqfunc}{Q}                    % reward random variable


% Math notations
\newcommand{\widx}{\lambda}
\newcommand{\mcal}[1]{\mathcal{#1}}
\newcommand{\proba}[1]{\mathbb{P}\left(#1\right)}
\newcommand{\ex}[1]{\mathbb{E}\left[#1\right]}
\newcommand{\var}[1]{\mathbb{V}\left[#1\right]}
\newcommand{\pik}{\pi_k}
\newcommand{\Mk}{M_k}
\newcommand{\Mbar}{\bar{M}}
\newcommand{\ep}{\varepsilon}
\newcommand{\rhat}{\hat{r}}
\newcommand{\Phat}{\hat{P}}
\newcommand{\Ahat}{\hat{A}}
\newcommand{\Bhat}{\hat{B}}
\newcommand{\Qhat}{\hat{Q}}
\newcommand{\rbar}{\bar{r}}
\newcommand{\Pbar}{\bar{P}}
\newcommand{\I}{\mathbb{I}}
\newcommand{\R}{\mathbb{R}}
\newcommand{\N}{\mathbb{N}}
\newcommand{\ba}{\boldsymbol{a}}
\newcommand{\bA}{\boldsymbol{A}}
\newcommand{\bs}{\boldsymbol{s}}
\newcommand{\br}{\boldsymbol{r}}
\newcommand{\bb}{\boldsymbol{b}}
\newcommand{\bx}{\boldsymbol{x}}
\newcommand{\bX}{\boldsymbol{X}}
\newcommand{\by}{\boldsymbol{y}}
\newcommand{\bY}{\boldsymbol{Y}}
\newcommand{\voisin}{\mathcal{V}}
\newcommand{\brhat}{\hat{\br}}
\newcommand{\event}{\xi}
\newcommand{\mTilde}{\tilde{M}}
\newcommand{\aTilde}{\tilde{a}}
\newcommand{\piTilde}{\tilde{\pi}}
\DeclareMathOperator*{\argmax}{arg\,max}
\DeclareMathOperator*{\argmin}{arg\,min}


