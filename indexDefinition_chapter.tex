\begingroup

\let\clearpage\relax

\chapter{Indexability and Bellman optimality}
\label{ch:indexability}

In the previous chapter, we gave the formalism of Markovian bandits and recalled two index policies: Gittins and Whittle index policies.
In this chapter, we discuss the existence of the Whittle index in undiscounted restless Markovian arm and provide a few main contributions to characterize such an existence.
This work is a part of our article submitted to Mathematical Methods of Operations Research (MMOR) journal (\cite{gast2022computing}).

We present our main contributions in Section~\ref{ch:idx:sec:contrib}.
Then, the notations and $\lambda$-penalized MDP that is the basis for defining the indexability are described in Section~\ref{ch:idx:sec:notation}.
After that, we cover the ambiguities in the classical definition of indexability accompanied by simple examples in Section~\ref{ch:idx:sec:classic_idx}.
These ambiguities inspire us to introduce a new notion of Bellman optimality for MDPs with the average reward criterion in Section~\ref{ch:idx:sec:bell}.
Based on this notion, we introduce our definitions of indexability and Whittle index in Section~\ref{ch:idx:sec:idx}.
Next, we extend the discussion about indexability in Section~\ref{ch:idx:sec:more_defn}.
Then, Section~\ref{ch:idx:sec:prop_bell} contains the properties of Bellman optimal policies.
Finally, we conclude the chapter in Section~\ref{ch:idx:sec:conc} by summarizing the chapter and giving a brief introduction to the next chapter.


\section{Contributions}
\label{ch:idx:sec:contrib}

In this chapter, we discuss the definition of indexability and present two main contributions.

Our first contribution is to propose a univocal definition of indexability. This definition clarifies the ambiguities in the classical definitions: Classical definitions assume that an arm is indexable if the optimal set of active states is a non-increasing function of some penalty term $\lambda$. While this definition works for most practical cases, it is not always precise enough because the optimal set is not unique in general. In our definition, we specify the notion of increasingness that should be used. Our definition guarantees the uniqueness of Whittle indices.

Lastly, we introduce a new notion of Bellman optimality for MDPs with the average reward criterion. This notion is more restrictive than the classical gain optimality, but it is rich enough to allow us to redefine the indexability in undiscounted restless Markovian bandits. We also study the characterization of gain optimal and Bellman optimal policies. In particular, we provide a checkable condition for the unicity of a given Bellman optimal policy.
This condition is useful for index computation in the chapter that follows.
This is a new contribution to the vast literature of MDP and Markovian bandit.

\section{Notations and problem formulation}
\label{ch:idx:sec:notation}

Since index policies require computation on each arm individually, we will focus only on a single arm.
Also, this chapter and the next chapter concern the computational complexity.
So, we redefine some notations in the previous chapter.
These redefined notations are used uniquely for Part~\ref{part:idx}.

\subsection{Restless bandit arm}

An $n$-state restless bandit arm is a Markov decision process (MDP) with a discrete state space $[n]:=\{1,\dots, n\}$ and a binary action space $\{0, 1\}$, where $0$ denotes the action ``rest'', and $1$ denotes the action ``activate''. The time is discrete, and the evolution is Markovian: If the MDP is in state $i$, and action $a$ is chosen, the MDP incurs an instantaneous reward ${r}^a_i$ and transitions to a new state $j$ with probability ${P}^a_{ij}$. We denote this MDP by $\langle[n],\{0,1\},r,P\rangle$.

\subsection{\texorpdfstring{$\lambda$-p}{P}enalized MDP and policy structure}
\label{ch:idx:sec:penal_mdp}

We consider a single arm $\langle[n],\{0,1\},r,P\rangle$.
Following the Lagrangian relaxation in the previous chapter, we define a $\lambda$-penalized MDP\footnote{not to be confused with $\gamma$-discounted MDPs, where the discount is on rewards and not on actions.} for each $\lambda\in\R$ as the following.
The state and action spaces, and transition of this MDP are the same as the ones of the arm.
The instantaneous reward for action $a$ when the arm is in state $i$ is $r^a_i-\lambda a$.
Recall that the quantity $\lambda$ is a penalty for taking action ``activate''.

Since there are only two actions, a policy $\pi$ is a subset of the state space, $\pi\subseteq[n]$, such that the policy chooses to activate the arm in state $i$ if $i\in\pi$.
We say that $\pi$ is a set of \emph{active} states, and state $i$ is \emph{passive} if $i\not\in\pi$.
By abuse of notation, we will write $\pi_i=1$ if $i\in\pi$, and $\pi_i=0$ if $i\not\in\pi$. 
The sequential decision problem is the average reward criterion in the $\lambda$-penalized MDP.
For a given $\lambda$, we denote $\pi^*(\lambda)$ an optimal policy for the penalty $\lambda$.  %$\lambda$-penalized MDP.


\section{Discussion on the classical definition of indexability}
\label{ch:idx:sec:classic_idx}

The classical definition of indexability used in the literature \citep{whittle1996optimal,akbarzadeh2020conditions,nino2020fast,gibson2021novel,nakhleh2021neurwin} says that an arm is indexable if and only if the optimal policy $\pi^*(\lambda)$ is non-increasing in $\lambda$ (for the inclusion order).
If an arm is indexable, these papers define the Whittle index of a state $i$ as a real number $\lambda_i$ such that ${\pi^*(\lambda)=\{i\in[n]: \lambda_i\ > \lambda\}}$.
%This definition is ambiguous for two reasons: First, optimal policies are in general not unique.
%Hence, the notion of $\pi^*(\lambda)$ being non-increasing is unclear: should all optimal policies be non-increasing or at least one?
%Second, the notion of optimality for a policy  is also unclear: should it mean ``gain optimal'', ``bias optimal'' or another notion of optimality?
Yet, we argue that this definition has two problems:
\begin{enumerate}
    \item What does ``increasing'' mean when $\pi^*(\lambda)$ is not unique? Two possibilities are: for all penalties $\lambda<\lambda'$:
    \begin{enumerate}
        \item[($\exists$)] there exist policies $\pi,\pi'$ with $\pi$ optimal for $\lambda$ and $\pi'$ optimal for $\lambda'$ such that $\pi\supseteq\pi'$;
        \item[($\forall$)] for all policies $\pi,\pi'$ such that  $\pi$ is optimal for $\lambda$ and $\pi'$ is optimal for $\lambda'$, we have $\pi\supseteq\pi'$.
    \end{enumerate}
    \item What notion of ``optimality'' should be used? Should it be ``gain optimal'', ``bias optimal'' or another notion of optimality?
\end{enumerate}

The most problematic choice is the notion of increasingness that applies to both discounted and undiscounted restless arm: Interpretation $(\exists)$ is more permissive: For instance, consider an arm with two states, and assume that the optimal policy is $\{1,2\}$ for $\lambda<0$, and either $\{1\}$ or $\emptyset$ for $\lambda>0$.
Interpretation $(\exists)$ says that the arm is indexable while interpretation ($\forall$) says that this arm is not indexable.
If the arm is indexable, what should be the index of state $1$?
Any choice of $\lambda_1\in[0,+\infty]$ seems reasonable.
Saying that the arm is not indexable clarifies the situation.
So, we will choose the interpretation ($\forall$) in our new definition of indexability in Section~\ref{ch:idx:sec:idx}.

The choice of optimality is due to the average reward criterion.
To show it, let us consider the example in \figurename~\ref{fig:ambiguous_example}.
In this example, the gain optimal policies are $\{1,2\}$ and $\{2\}$ for $\lambda<0$, and $\{1\}$ and $\emptyset$ for $\lambda>0$.
According to the interpretation~$(\exists)$, the problem is indexable, but the index for state $1$ is unclear.
According to the interpretation~$(\forall)$, the problem is not be indexable.
Yet, it is reasonable that this example is indexable, and the index of state $1$ and $2$ is $\widx_1=\widx_2=0$.
We argue that it is not the interpretation~$(\forall)$ to blame for the non-indexability of this example but the optimality criterion.
That is, for $\lambda<0$, the policy $\{2\}$ is gain optimal, but it rests the arm in state $1$ and gets a reward of $1$ while it is ``better'' to activate the arm in state $1$ and get a reward of $1-\lambda$. 
The policy $\{2\}$ is gain optimal because state $1$ is a transient state, and state $2$ is the recurrent state.
So, ``acting optimally'' in the recurrent states ``leads to'' gain optimality (we will discuss this in Section~\ref{ch:idx:ssec:gain_bell}).
The same discussion goes for policy $\{1\}$ when $\lambda>0$.
This inspires us to introduce a new notion of optimality for policy $\{1,2\}$ when $\lambda<0$, and policy $\emptyset$ when $\lambda>0$.

\begin{figure}[ht]
    \centering
    \begin{tabular}{ccc}
        \begin{minipage}{.25\linewidth}
            \begin{tikzpicture}[on grid, state/.style={circle,draw}, >= stealth', auto, prob/.style = {inner sep=1pt,font=\scriptsize}]
                \node[state,color=blue]  (A) {$2$};
                \node[state,color=blue]  (B) [left =1.5cm of A]   {$1$};
                \path[->]
                    (A) edge[loop above,color=black]  node{$1{-}\lambda$} (A)
                    (A) edge[loop right, color=red, dashed]     node{$1$} (A)
                    (B) edge[bend left, color=black]     node{$1{-}\lambda$} (A)
                    (B) edge[bend right, color=red, dashed]     node[below]{$1$} (A);
            \end{tikzpicture}
        \end{minipage}
        &
        \begin{minipage}{.7\linewidth}
            The gain optimal policies are $\{1,2\}$ and $\{2\}$ for $\lambda<0$, and $\{1\}$ and $\emptyset$ for $\lambda>0$.
            When $\lambda<0$, policy $\{2\}$ is gain optimal but acts suboptimal in state $1$.
            When $\lambda>0$, policy $\{1\}$ is gain optimal but acts suboptimal in state $1$.
        \end{minipage}\\
    \end{tabular}
    
    \caption{Ambiguous example for classical definition of indexability: All transitions are deterministic, and all labels on transitions indicate rewards. Solid black arrows correspond to the action ``activate'', and dashed red arrows to the action ``rest''.
}
    \label{fig:ambiguous_example}
\end{figure}

\section{Definition of Bellman optimality}
\label{ch:idx:sec:bell}

%Let us fix a value of penalty $\lambda$.
Given a policy $\pi$, we denote by $g^\pi_i$ the gain of state $i$ when following policy $\pi$ in $\lambda$-penalized MDP.
Let $g^*_i=\max_\pi g^\pi_i$ be the maximal gain starting from state $i$.
As mentioned in Section~\ref{ch:mdp:ssec:gain}, the optimal gain $\vg^*$ is uniquely defined.
We recall from Section~\ref{ch:mdp:ssec:gain} that a policy $\pi$ is \emph{gain optimal} if $g^\pi_i=g^*_i$ for all state $i$.
Also, the vector $\vg^*$ is the optimal gain if and only if there exists a vector $\vh^*$, called \emph{optimal bias} vector that satisfies the Bellman \emph{optimality} equations:  for all $i\in[n]$,
\begin{align}
    g^*_i &= \max_{a\in\{0,1\}} \Bigl(\sum_{j=1}^n P^{a}_{ij}g^*_j \Bigr) \nonumber \\%\label{eq:gain_opt} \\
    g^*_i + h^*_i &= \max_{a\in\{0,1\}} \Big( r^{a}_i -\lambda a_i+ \sum_{j=1}^n P^{a}_{ij}h^*_j \Big).  \label{ch:idx:eq:bias_opt}
\end{align}

%For $\lambda$-penalized MDPs, we define the gain and bias functions as in Section~\ref{ch:mdp:ssec:gain}, but these quantities now depend on $\lambda$.
%Hence, we will write them as functions of $\lambda$: For instance, the optimal gain is $\vg^*(\lambda)$, and we will use the notation $\vh^*(\lambda)$ to denote an optimal bias, and $\pi^*(\lambda)$ to denote an optimal policy.


%We denote by $\mP^\pi$ the transition matrix corresponding to the policy $\pi$, \ie, $P^\pi_{ij}=P^{\pi_i}_{ij}$, and by $\vr^\pi$ the corresponding reward vector, \ie, $r^\pi_i{=}r^{\pi_i}_i$.

We define a new notion of optimality as the following.
\begin{defn}[Bellman optimal policy]
    \label{ch:idx:defn:bell}
    A policy $\pi$ is \emph{Bellman optimal} if there exists a vector $\vh^*\in\R^n$ satisfying Bellman optimality equation \eqref{ch:idx:eq:bias_opt} with the optimal gain $\vg^*$ such that, for all state $i\in[n]$,
    \begin{equation}
        \label{ch:idx:eq:bell}
        \sum_{j=1}^nP^{\pi_i}_{ij}g^*_j =g^*_i \text{ and } \pi_i=\argmax_{a\in\{0,1\}} \Big(r^{a}_i - \lambda a_i + \sum_{j=1}^n P^{a}_{ij}h^*_j\Big).
    \end{equation}
\end{defn}
By \cite[Theorem~9.1.7]{puterman2014markov}, Bellman optimal policy always exists in MDPs with finite state and action spaces, and Bellman optimality implies gain optimality.
However, we will see in Section~\ref{ch:idx:ssec:gain_bell} that the converse is not true in general.
So, the notion of Bellman optimality is more restrictive than the notion of gain optimality.
%We give an example in \figurename~\ref{fig:gain_vs_bellman} to illustrate the difference between Bellman and gain optimalities.
%By this definition, policy $\pi^2$ in \figurename~\ref{fig:gain_vs_bellman} is Bellman optimal.

With this definition, in \figurename~\ref{fig:ambiguous_example}, policy $\{2\}$ is not Bellman optimal for $\lambda<0$, and policy $\{1\}$ is not Bellman optimal for $\lambda>0$.
The Bellman optimality is very natural, and we say that it is new because, to the best of our knowledge, the definition is never introduced the literature of MDP.
Note that the distinction between gain optimality and Bellman optimality disappears in infinite horizon discounted problems or in ergodic MDPs with the average reward criterion.

In general MDPs, a Bellman optimal policy can be obtained using \emph{Multichain Policy Iteration} algorithm.
We refer to \cite[Section~9.2.1]{puterman2014markov} for more detail about this algorithm.

We stay focused on the indexability of restless Markovian arm in this chapter's progression and will discuss the Bellman optimality more at the end of the chapter.

\section{Our unambiguous definition of indexability}
\label{ch:idx:sec:idx}

To solve the two ambiguities in the classical definition of indexability, we use the following definition.  
\begin{defn}[Indexability]
    \label{ch:idx:defn:idx}
    Given a finite-state arm, let $\Pi^*(\lambda)$ be the set of all Bellman optimal policies for a penalty $\lambda$.
    We say that the arm is indexable if for all $\lambda<\lambda'$, and all policies $\pi\in\Pi^*(\lambda)$ and $\pi'\in\Pi^*(\lambda')$, $\pi\supseteq\pi'$.
\end{defn}
This definition says that the function $\pi^*(\lambda)\supseteq\pi^*(\lambda')$ regardless of the choice of Bellman optimal policies.
With this definition, the example in \figurename~\ref{fig:ambiguous_example} is indexable.
Indeed, for $\lambda<0$, $\{1,2\}$ is the only Bellman optimal policy, and for $\lambda>0$, $\emptyset$ is the only Bellman optimal policy.
%So, the definition implies that the example is indexable.
%We will see that the indices are $\lambda_1=\lambda_2=0$.

As we show next, Definition~\ref{ch:idx:defn:idx} guarantees that the Whittle indices are uniquely defined when they exist.
%We will revisit this lemma again in the next chapter in order to propose a characterization of any indexable arm, which we will later use to derive our algorithm for computing the Whittle index as well as the indexability.
\begin{lem}[Definition of the Whittle index]
    \label{ch:idx:lem:idx}
    In an $n$-state arm, the two following properties are equivalent: 
    \begin{enumerate}
        \item[(i)] The arm is indexable.
        \item[(ii)] For all state $i\in[n]$, there exists a unique penalty $\lambda_i$ -- called the Whittle index of state $i$ -- such that if $\pi\in\Pi^*(\lambda)$ is any Bellman optimal policy for the penalty $\lambda$, then $\pi_i=1$ if $\lambda<\lambda_i$, and $\pi_i=0$ if $\lambda>\lambda_i$. 
    \end{enumerate}
\end{lem}

With this lemma, the indices in Figure~\ref{fig:ambiguous_example} are $\lambda_1=\lambda_2=0$.
We should stress that, by Lemma~\ref{ch:idx:lem:idx}, the Whittle index $\lambda_i$ can either be finite or infinite. When we say that ``a policy $\pi$ is optimal for the penalty $+\infty$'', this means ``there exists a penalty $\bar{\lambda}$ such that $\pi$ is optimal for all $\lambda\ge\bar{\lambda}$''.
Similarly, when we say that ``a policy $\pi$ is optimal for the penalty $-\infty$'', this means ``there exists a penalty $\underline{\lambda}$ such that $\pi$ is optimal for all $\lambda\le\underline{\lambda}$''.

\begin{proof}
    The lemma is a direct consequence of the definition of indexability.

    (i) $\Rightarrow$ (ii) -- Assume first that the arm is indexable.
    Let $i\in[n]$ be a state, and let $\lambda_i=\sup\{\lambda : \exists\pi\in\Pi^*(\lambda)\text{ such that }\pi_i=0\}$.
    By Definition~\ref{ch:idx:defn:idx}, if $\pi'$ is a Bellman optimal policy for a penalty $\lambda>\lambda_i$, then $\pi'\subseteq\pi$, which in turn implies that $\pi'_i=0$.
    Similarly, if $\lambda<\lambda_i$, then $\pi'_i=1$. This implies (ii).

    (ii) $\Rightarrow$ (i) -- Assume $(ii)$, and let $\sigma^k$ be the state with the $k$th smallest index (where ties are broken arbitrarily) for $1\le k\le n$.
    Let $\lambda_{\sigma^k}$ be the index of the state $\sigma^k$, and let $\lambda\in(\lambda_{\sigma^{k-1}}, \lambda_{\sigma^k})$. 
    By $(ii)$, any Bellman optimal policy for the penalty $\lambda<\lambda_{\sigma^{k-1}}$ contains $\pi^{k}:=[n]\setminus\{\sigma^1,\dots, \sigma^{k-1}\}$. Similarly, $\pi^{k}$ contains any Bellman optimal policy for the penalty $\lambda>\lambda_{\sigma^{k-1}}$.
    So, the condition in Definition~\ref{ch:idx:defn:idx} is satisfied.
\end{proof}

\section{More definitions of indexability?}
\label{ch:idx:sec:more_defn}

In our new definition of indexability, we clarify the notion of increasingness by using the interpretation~$\forall$.
By doing so, we already solve the ambiguity in the indexability of the discounted restless bandit.
Meanwhile, the last remaining ambiguity is due to the average reward setting.
We solve this by using the notion of Bellman optimality because we believe that this notion is the most natural one.
%It coincides with the notion of optimality in infinite-horizon discounted MDPs.
This allows the example of \figurename~\ref{fig:ambiguous_example} to be indexable.
However, we must recognize that this new definition still does not cover all the problem of indexability in the undiscounted restless bandit.
Indeed, let us consider the two examples in \figurename~\ref{fig:ambiguous_example2}.

\begin{figure}[ht]
   \centering
   \begin{tabular}{ccc}
       \begin{minipage}{.35\linewidth}
           \begin{tikzpicture}[on grid, state/.style={circle,draw}, >= stealth', auto, prob/.style = {inner sep=1pt,font=\scriptsize}]
           \node[state,color=blue]  (A) {$2$};
           \node[state,color=blue]  (B) [left =1.5cm of A]   {$1$};
           \path[->]
               (A) edge[loop above,color=black]  node{$1{-}\lambda$} (A)
               (A) edge[loop right, color=red, dashed]     node{$1$} (A)
               (B) edge[color=black]     node{$1{-}\lambda$} (A)
               (B) edge[loop left, color=red, dashed]     node[left]{$1$} (B);
           \end{tikzpicture}
       \end{minipage}
       &
       \begin{minipage}{.35\linewidth}
           \begin{tikzpicture}[state/.style={circle,draw}, >= stealth', auto, prob/.style = {inner sep=1pt,font=\scriptsize}]
               \node[state,color=blue] at (0,0) (A) {$1$};
               \node[state,color=blue] at (2,.5)  (B) {$2$};
               \node[state,color=blue] at (2,-.5)  (C) {$3$};
               \node[state,color=blue] at (4,0)  (D) {$4$};
           \path[->]
               (A) edge[color=black]  node[above]{$1{-}\lambda$} (B)
               (A) edge[color=red, dashed]     node[below]{$1$} (C)
               (B) edge[color=red, dashed]     node[above]{$1$} (D)
               (C) edge[color=black]     node[below]{$1{-}\lambda$} (D)
               (B) edge[color=black, loop above] node{$-10-\lambda$} (B)
               (C) edge[color=red, dashed,loop below] node{$-10$} (C)
               (D) edge[color=red, dashed,loop above] node{$10$} (D)
               (D) edge[color=black, loop below] node{$-\lambda$} (D)
               ;
           \end{tikzpicture}
       \end{minipage}        \\
       (a) & (b)
    \end{tabular}
    \caption{Ambiguous examples: All transitions are deterministic, and all labels on transitions indicate rewards. Solid black arrows correspond to the action ``activate'', and dashed red arrows to the action ``rest''.
}
    \label{fig:ambiguous_example2}
\end{figure}

For example (a), the Bellman optimal and gain optimal policies are identical: policies $\{1,2\}$ and $\{2\}$ for $\lambda<0$, and $\{1\}$ and $\emptyset$ for $\lambda>0$.
Hence, example (a) is not indexable according to Definition~\ref{ch:idx:defn:idx}.
However, the intuition would suggest that the index for this problem should be $\lambda_1=\lambda_2=0$. This is what would have happened if we had used the notion of \textbf{bias optimality}, which is stronger than the notion of Bellman optimality. Yet, this would not solve the ambiguity of example (b), for which the index of state $1$ is not clear unless one would use an even stronger notion of optimality (such as the notion of \textbf{Blackwell optimality} discussed in \cite[Chapter~10]{puterman2014markov}). In this thesis, we do not use these stronger definitions because computing a Blackwell-optimal policy in a $n$-state MDP is, to the best of our knowledge, not computable in $O(n^3)$ (the complexity of computing Blackwell-optimal policies is unknown to this date). Hence, in this thesis, we use the notion of Bellman optimality that we believe to be the best trade-off between expressiveness and computability. We discuss the Bellman optimality more in the following section.

\section{Properties of Bellman optimal policies}
\label{ch:idx:sec:prop_bell}

In Section~\ref{ch:idx:sec:bell}, we introduce the notion of Bellman optimality in $\lambda$-penalized MDP.
However, this notion can be defined in any MDPs.
In this section, we discuss the properties of Bellman optimality and the difference between this optimality and the gain optimality.
To ease the exposition, we consider a fixed penalty $\lambda=0$ so that we can drop $\lambda$ from the notation.
Thus, all the technical lemmas in this section apply to general MDPs.

\subsection{Advantage function and characterization of Bellman optimality}

Let $\vg$ and $\vh$ be a solution of Bellman evaluation equations \eqref{eq:gain_eval} and \eqref{eq:bias_eval} for policy $\pi$.
We define the (dis-)advantage of action\footnote{Note that it is also a function of $\vh$.} $a$ in state $i$ over policy $\pi$ by
\begin{equation}
    \label{ch:idx:eq:advan}
    B^a_i(\vg,\vh) := r^a_i +\sum_{j\in[n]}P^a_{ij}h_j -g_i -h_i.
\end{equation}
If $\pi$ is unichain, then $\vh$ is determined up to a constant vector, and $B^a_i(\vg,\vh)$ is then uniquely defined for any state $i\in[n]$, $a\in\{0,1\}$, and $(\vg,\vh)$ that satisfies \eqref{eq:gain_eval} and \eqref{eq:bias_eval}.

For $(\vg^*,\vh^*)$ a solution of Bellman optimality equations \eqref{eq:gain_opt} and \eqref{eq:bias_opt},
an optimal action $a$ for state $i$ satisfies both $\sum_{j\in[n]}P^a_{ij}g^*_j =g^*_i$ and $B^a_i(\vg^*,\vh^*)=0$ \cite{puterman2014markov, schweitzer1978functional}.
\begin{lem}[Bellman optimality characterization]
    \label{ch:idx:lem:bell_charac}
    Consider a policy $\pi:[n]\mapsto\{0,1\}$.
    The three following properties are equivalent.
    \begin{enumerate}[label=(\roman*)]
        \item \label{it:opt_bm0} $\pi$ is Bellman optimal.
        \item \label{it:opt_bm1} There exists a vector $\vh^*\in\R^{n}$ satisfying Bellman optimality equation \eqref{eq:bias_opt} with $\vg^*$ such that $\sum_{j\in[n]}P^{\pi_i}_{ij}g^*_j=g^*_i$ and $B^{\pi_i}_i(\vg^*,\vh^*)=0$ for all states $i\in[n]$.
        \item \label{it:opt_bm2} There exists a vector $\vh\in\R^{n}$ satisfying Bellman evaluation equation \eqref{eq:bias_eval} for $\pi$ with $\vg^*$ such that, for all state $i\in[n]$, $\sum_{j\in[n]}P^{\pi_i}_{ij}g^*_j=g^*_i$ and $B^a_i(\vg^*,\vh)\le0$ for all action $a\in\{0,1\}$.
        \end{enumerate}
\end{lem}
\begin{proof}
    \ref{it:opt_bm0} $\Leftrightarrow$ \ref{it:opt_bm1} is a direct consequence of Definition~\ref{ch:idx:defn:bell} and \eqref{ch:idx:eq:advan}.

    \ref{it:opt_bm2} $\Rightarrow$ \ref{it:opt_bm1}: By definition of advantage function, $B^a_i(\vg^*,\vh)\le0$ for all action $a\in\{0,1\}$ and all state $i\in[n]$ if and only if $\vh$ satisfies the optimality equation \eqref{eq:bias_opt}.
    Moreover, if $\vh$ satisfies the evaluation equation \eqref{eq:bias_eval} for $\pi$ with $\vg^*$, then $B^{\pi_i}_i(\vg^*,\vh)=0$ for all state $i\in[n]$.

    \ref{it:opt_bm0} $\Rightarrow$ \ref{it:opt_bm2} is a direct consequence of Definition~\ref{ch:idx:defn:bell} and \eqref{eq:bias_eval}.
\end{proof}

\subsection{Difference between gain and Bellman optimalities}
\label{ch:idx:ssec:gain_bell}

The immediate difference between both optimalities is that the Bellman optimality is more restrictive than the gain optimality.
Indeed, Lemma~\ref{ch:idx:lem:bell_charac} shows that a policy is Bellman optimal if and only if it achieves the maximum of the optimality equation \eqref{eq:bias_opt} on all states for some $\vh^*$.
On the other hand, the following lemma shows that a policy is gain optimal if and only if it achieves the maximum of \eqref{eq:bias_opt} on its recurrent states for any $\vh^*$.
%So, both lemmas show the difference between gain and Bellman optimal policies.
\begin{lem}[Gain optimality characterization]
    \label{ch:idx:lem:opt_pol}
    Let $\pi:[n]\mapsto\{0,1\}$ be a policy, and $\Phi^\pi$ be the set of recurrent states under policy $\pi$.
    The three properties below are equivalent.
    \begin{enumerate}[label=(\roman*)]
        \item \label{it:opt_pol1} For all state $i$, $\sum_{j\in[n]}P^{\pi_i}_{ij}g^*_j=g^*_i$ and for \textbf{all} $\vh^*\in \R^{n}$ satisfying Bellman optimality equation \eqref{eq:bias_opt} with $\vg^*$, $B^{\pi_i}_i(\vg^*,\vh^*)=0$ for all $i\in\Phi^\pi$.
        \item \label{it:opt_pol2} For all state $i$, $\sum_{j\in[n]}P^{\pi_i}_{ij}g^*_j=g^*_i$ and for \textbf{some} $\vh^*\in \R^{n}$ satisfying Bellman optimality equation \eqref{eq:bias_opt} with $\vg^*$, $B^{\pi_i}_i(\vg^*,\vh^*)=0$ for all $i\in\Phi^\pi$.
        \item \label{it:opt_pol3} $\pi$ is gain optimal.
    \end{enumerate}
\end{lem}
\begin{proof}
    First, given a policy $\pi$, let $\vr^\pi$ be the reward vector induced by $\pi$, $r^\pi_i=r^{\pi_i}_i$ for any $i\in[n]$.
    Also, let $\mP^\pi$ be the transition matrix of the Markov chain induced by $\pi$, $P^\pi_{ij}=P^{\pi_i}_{ij}$ for any $i,j\in[n]$.
    Next, we define the \emph{limiting matrix} $\bar{\mP}^\pi{:=}\displaystyle\Clim_{t\to+\infty}{(\mP^\pi)^t}$ (\cite[Appendix~A.4]{puterman2014markov}). % which describes the state transition in steady regime under policy $\pi$. That is, $\bar{P}^\pi(s,s')$ is the probability that the MDP transitions from state $s$ to $s'$ in steady regime.
    Since the MDP has a finite state space, $\bar{\mP}^\pi$ always exists and well-defined for any stationary policy $\pi$.
    It describes the state transition in the steady regime under policy $\pi$.
    That is, $\bar{P}^\pi_{ij}$ is the probability that the arm transitions from state $i$ to $j$ in the steady regime under policy $\pi$.
    %We denote by $\Phi^\pi$ the set of recurrent states under policy $\pi$.
    From \cite[Section~A.4]{puterman2014markov}, we have
    \begin{itemize}
        \item $\bar{\mP}^\pi\mP^\pi =\mP^\pi\bar{\mP}^\pi =\bar{\mP}^\pi$
        \item If $j\notin\Phi^\pi$, then $\bar{P}^\pi_{ij}=0$ for all $i\in[n]$
        \item If $\pi$ is unichain, then the rows of $\bar{\mP}^\pi$ are identical.
    \end{itemize}
    By \cite[Theorem~8.2.6]{puterman2014markov}, the gain of policy $\pi$ can be expressed by $g^\pi_i=\sum_{j\in[n]}\bar{P}^\pi_{ij}r^{\pi}_j$ for any $i\in[n]$.
    In vector notation, $\vg^\pi=\bar{\mP}^\pi\vr^\pi$.

    Now, we can prove the lemma.

    \ref{it:opt_pol1} $\Rightarrow$ \ref{it:opt_pol2} is trivial.

    \ref{it:opt_pol2} $\Rightarrow$ \ref{it:opt_pol3}: By definition of $B_i^a(\vg^*,\vh^*)$, we have $r^{\pi}_i-g^*_i = h^*_i-\sum_{j\in[n]} P^{\pi}_{ij}h^*_j$ for any recurrent state $i$ under $\pi$.
    Multiply this with $\bar{P}^\pi_{ki}$, and sum over $i\in[n]$ (if $i$ is not recurrent, then $\bar{P}^\pi_{ki}=0$) gives
    \begin{align*}
        \sum_{i\in[n]} \bar{P}^\pi_{ki}\Big(r^{\pi}_i-g^*_i\Big) {=} \sum_{i\in[n]} \bar{P}^\pi_{ki}h^*_i {-} \underbrace{\sum_{i\in[n]} \bar{P}^\pi_{ki}\sum_{j\in[n]} P^{\pi}_{ij}h^*_j}_{=\sum_{j\in[n]} \bar{P}^\pi_{kj}h^*_j \text{ since $\bar{\mP}^\pi\mP^\pi{=}\bar{\mP}^\pi$.}}
        = \vzero.
    \end{align*}
    The gain of $\pi$ is $\bar{\mP}^\pi \vr^\pi$.
    The above equation shows that $\bar{\mP}^\pi\vr^\pi = \bar{\mP}^\pi\vg^*$. Moreover, the assumption  $\mP^\pi\vg^*=\vg^*$ implies that $\bar{\mP}^\pi\vg^*=\vg^*$, which in turn implies that $\bar{\mP}^\pi\vr^\pi=\vg^*$. This shows that the gain of $\pi$ is $\vg^*$, and therefore $\pi$ is gain optimal.
    % So, we have $\mP^\pi\vg^*=\vg^*$ and $\bar{\mP}^\pi(\vr^\pi-\vg^*)=\vzero$.

    \ref{it:opt_pol3} $\Rightarrow$ \ref{it:opt_pol1}: If $\pi$ is gain optimal, then $\mP^\pi \vg^*=\vg^*$ and $\bar{\mP}^\pi(\vr^\pi-\vg^*)=\vzero$.
    The latter rewrites as $\sum_{i\in[n]}\bar{P}^\pi_{ki}\Big(r^{\pi}_i-g^*_i\Big) =0$ for all state $k$. For some $\vh^*$ that satisfies the optimality equation \eqref{eq:bias_opt} with $\vg^*$, we have: for all state $k$
    \begin{align*}
        %\label{eq:apx_proof_H}
        \sum_{i\in[n]}\bar{P}^\pi_{ki}B^{\pi_i}_i(\vg^*,\vh^*) = \sum_{i\in[n]}\bar{P}^\pi_{ki}\Big(r^{\pi}_i-g^*_i {+}\sum_{j\in[n]}P^{\pi}_{ij}h^*_j -h^*_i\Big) =0.
    \end{align*}
    As $\vh^*$ satisfies \eqref{eq:bias_opt}, we have $B^{a}_i(\vg^*,\vh^*) \le0$  for all action $a\in\{0,1\}$ and all state $i\in[n]$.
    In particular, $B^{\pi_i}_i(\vg^*,\vh^*) \le0$.
    This shows that for any state $i$ such that $\bar{P}^\pi_{ki}>0$, one must have $B^{\pi_i}_i(\vg^*,\vh^*) =0$.
    Such state $i$ are the recurrent states of $\pi$.
    This shows that $B^{\pi_i}_i(\vg^*,\vh^*) =0$ for any $i\in\Phi^\pi$.
\end{proof}

Lemma~\ref{ch:idx:lem:opt_pol} shows that gain optimal policies achieve the maximum of \eqref{eq:bias_opt} on their recurrent states.
We say that the gain optimal policies \emph{act optimally} in their recurrent states.
With this characterization, in an ergodic MDP, a policy is gain optimal if and only if it acts optimally in all states.
Therefore, the distinction between gain and Bellman optimal disappears in the ergodic MDP.
However, for MDPs that admit transient states, a policy can be gain optimal without acting optimally in all states, \ie, it acts optimally in its recurrent states, and possibly randomly in the transient states.
To illustrate this, let us consider the example in \figurename~\ref{fig:gain_vs_bellman}.
This example is a unichain $2$-state MDP where state $1$ is the transient and allows two possible actions, and state $2$ is the recurrent state and allows only one possible action.
Then, there are two gain optimal policies $\pi^1$ and $\pi^2$.
By computing the solution $(\vg^*,\vh^*)$ to the Bellman optimality equations \eqref{eq:gain_opt} and \eqref{eq:bias_opt}, we see that, for any $\vh^*$, $\pi^1$ does not achieve the maximum of the optimality equation \eqref{eq:bias_opt} on state $1$ while $\pi^2$ does.
\begin{figure}[ht]
    \centering
    \begin{tabular}{cc}
        \begin{minipage}{.25\linewidth}
            \begin{tikzpicture}[on grid, state/.style={circle,draw}, >= stealth', auto, prob/.style = {inner sep=1pt,font=\scriptsize}]
                \node[state,color=blue]  (A) {$2$};
                \node[state,color=blue]  (B) [left =1.5cm of A]   {$1$};
                \path[->]
                    (A) edge[loop above,color=red, dashed] node{$1$} (A)
                    (B) edge[bend left, color=black] node{$1$} (A)
                    (B) edge[bend right, color=red, dashed] node[below]{$0.5$} (A);
            \end{tikzpicture}
        \end{minipage}
        &
        \begin{minipage}{.7\linewidth}
            There are two policies $\pi^1:=\emptyset$ and $\pi^2:=\{1\}$.
            Both policies are gain optimal.
            By solving the Bellman optimality equations \eqref{eq:gain_opt} and \eqref{eq:bias_opt}, we have $\vg^*=\vone$ and $\vh^*{=}c\vone$ where $c$ can be any real number.
            Consequently, $\mP^{\pi^1}\vg^* =\vg^*$ and $\mP^{\pi^2}\vg^* =\vg^*$, but
            policy $\pi^1$ does not satisfy the maximum of \eqref{eq:bias_opt} on state $1$ for any $\vh^*$ while $\pi^2$ does.
            So, policy $\pi^2$ is more restrictive than $\pi^1$.
        \end{minipage}
    \end{tabular}
    \caption{An example where some gain optimal policies do not satisfy the Bellman optimality equation \eqref{eq:bias_opt} in all states of the MDP.
        The black arrow shows state transition of action activate, and the red ones for action rest.
        The numbers along the arrows show the reward when executing the actions.
}
    \label{fig:gain_vs_bellman}
\end{figure}

Beside showing the difference between gain and Bellman optimalities, Lemma~\ref{ch:idx:lem:opt_pol} is very useful to prove the properties of Bellman optimal policy in the following section.

\paragraph{Remark.} The characterization of gain optimal policies was analyzed in \cite{puterman2014markov, schweitzer1978functional}.
However, it was expressed and proved differently from what we have done for Lemma~\ref{ch:idx:lem:opt_pol}.

\subsection{Conditions for the unicity of the Bellman optimal policy}
\label{ch:idx:ssec:unicity_bell}

As mentioned in Section~\ref{ch:idx:sec:classic_idx}, we use the interpretation ``all Bellman optimal policies'' to define the indexability.
We will see in the next chapter that this direction will require the unicity of the Bellman optimal policy.
So, we anticipate it in this section.

\subsubsection{Conditions for Bellman optimal policies to induce the same bias vector}

The previous lemma shows that a policy is gain optimal if and only if the actions in its recurrent states satisfy the optimality equation \eqref{eq:bias_opt} for any $\vh^*$.
However, some gain optimal policies induce a bias vector that does not satisfy \eqref{eq:bias_opt}.
That is, in \figurename~\ref{fig:gain_vs_bellman}, the bias of policy $\pi^1$ is $h^{\pi^1}_1=c^{\pi^1}-0.5$ and $h^{\pi^1}_2=c^{\pi^1}$, and the bias of policy $\pi^2$ is $\vh^{\pi^2}_1=c^{\pi^2}\vone$, where $c^{\pi^1}$ and $c^{\pi^2}$ are any real numbers (recall that $\vh^{\pi^1}$ is the solution of the evaluation equation \eqref{eq:bias_eval} for $\pi^1$, and $\vh^{\pi^2}$ for $\pi^2$).
So, $\vh^{\pi^1}$ does not satisfy the optimality equation \eqref{eq:bias_opt}, but $\vh^{\pi^2}$ does.
While $\pi^1$ and $\pi^2$ are both gain optimal policies, unichain, and share one common recurrent state, namely state $2$ (see \figurename~\ref{fig:gain_vs_bellman}), we still cannot generalize the relationship between $h^{\pi^1}_i$ and $h^{\pi^2}_i$ for all state $i\in[n]$.
In contrast, the following lemma clarifies this question for Bellman optimal policies that are unichain and share at least one common recurrent state.
%In contrast, the following lemma shows the relationship between two Bellman optimal policies that are unichain and share at least one common recurrent state.
%The previous lemma shows that a gain optimal policy satisfies \eqref{eq:bias_opt} on its recurrent states for any $\vh$ a solution of \eqref{eq:bias_opt}.
%An interesting question to ask is whether a Bellman optimal policy satisfies \eqref{eq:bias_opt} on all states for any $\vh$ a solution of \eqref{eq:bias_opt}.

\begin{lem}
    \label{ch:idx:lem:equi_bias}
    Suppose that two policies $\pi$ and $\theta$ are Bellman optimal, unichain and have at least one common recurrent state: $\Phi^\pi\cap\Phi^\theta\neq\emptyset$.
    
    Then for any $\vh^\pi$ and $\vh^\theta$ solutions of Bellman evaluation equation \eqref{eq:bias_eval} for $\pi$ and $\theta$, there exists a constant $c$ such that $h^\pi_i -h^\theta_i =c$ for all state $i\in[n]$.
    Moreover, ${B_i^{\theta_i}(\vg^*, \vh^\pi)=B_i^{\pi_i}(\vg^*, \vh^\theta)=0}$ for all state $i\in[n]$.
\end{lem}
\begin{proof}
    We define $\bar{\mP}^\pi$ and $\bar{\mP}^\theta$ for policies $\pi$ and $\theta$ as we did in the proof of Lemma~\ref{ch:idx:lem:opt_pol}.
    Since $\pi$ and $\theta$ are Bellman optimal, $\vh^\pi$ and $\vh^\theta$ satisfy the optimality equation \eqref{eq:bias_opt} together with $\vg^*$ (see Lemma~\ref{ch:idx:lem:bell_charac}).
    In consequence, we have
    \begin{align*}
        \vh^\pi \ge \vr^\theta -\vg^* +\mP^\theta\vh^\pi.
    \end{align*}
    By Lemma~\ref{ch:idx:lem:opt_pol}~\ref{it:opt_pol1}, the above inequality is an equality for any state $i\in\Phi^\theta$ because policy $\theta$ is gain optimal.

    As $\vh^\theta$ satisfies the evaluation equation \eqref{eq:bias_eval}, we have
    \begin{align*}
        \vh^\theta -\vh^\pi &\le \vr^\theta -\vg^* +\mP^\theta\vh^\theta -(\vr^\theta -\vg^* +\mP^\theta\vh^\pi) = \mP^\theta(\vh^\theta -\vh^\pi),
    \end{align*}
    with equality for any state $i\in\Phi^\theta$.
    This shows that for all $t$, $\vh^\theta -\vh^\pi\le (\mP^\theta)^t(\vh^\theta -\vh^\pi)$, which implies that $\vh^\theta -\vh^\pi\le \bar{\mP}^\theta(\vh^\theta -\vh^\pi)$, with equality for any state $i\in\Phi^\theta$.
    Similarly, $\vh^\pi -\vh^\theta \le \bar{\mP}^\pi(\vh^\pi -\vh^\theta)$, with equality for any state $i\in\Phi^\pi$.

    Let $c^\pi_i=\sum_{j\in[n]}\bar{P}^\pi_{ij}\Big(h^\pi_j-h^\theta_j\Big)$, and $c^\theta_i=\sum_{j\in[n]}\bar{P}^\theta_{ij}\Big(h^\pi_j-h^\theta_j\Big)$.
    By what we have just shown, for all state $i$, we have
    \begin{align*}
        c^\theta_i \underbrace{\le}_{\text{equality if $i\in\Phi^\theta$}} h^\pi_i-h^\theta_i \underbrace{\le}_{\text{equality if $i\in\Phi^\pi$}} c^\pi_i
    \end{align*}
    As both policies are unichain, $c^\pi_i$ and $c^\theta_i$ do not depend on state $i$.
    Moreover, if there exists $i\in\Phi^\theta\cap\Phi^\pi$, then $c^\pi_i=c^\theta_i = c$. In consequence,  $h^\pi_i-h^\theta_i=c$ for all state $i$. 
    Furthermore, by definition of advantage function, $B_i^{\pi_i}(\vg^*, \vh^\pi)=B_i^{\pi_i}(\vg^*, \vh^\theta)=0$ and $B_i^{\theta_i}(\vg^*, \vh^\theta)=B_i^{\theta_i}(\vg^*, \vh^\pi)=0$ for all state $i$.
\end{proof}

\subsubsection{Unicity of the Bellman optimal policy}
\label{ch:idx:sssec:unicity}

By Definition~\ref{ch:idx:defn:bell}, checking if a given Bellman optimal policy is unique is not trivial due to the multiplicity of $\vh^*$ that satisfies the optimality equation \eqref{eq:bias_opt} with $\vg^*$.
So, the following lemma shows a condition that can be used to verify if a given Bellman optimal and unichain policy is the unique Bellman optimal policy.
\begin{lem}[Condition for unicity of Bellman optimal policy]
    \label{ch:idx:lem:unicity_BO}
    Let $\pi$ be a Bellman optimal policy that is unichain. If $\pi$ is not the unique Bellman optimal policy, then there exists a state $i$ and an action $a\neq\pi_i$ such that $B_i^a(\vg^*,\vh^\pi)=0$.
\end{lem}

\begin{proof}
    Let $\theta\neq\pi$ be another Bellman optimal policy.
    Since $\theta$ is gain optimal, and $\vh^\pi$ satisfies the optimality equation \eqref{eq:bias_opt}, Lemma~\ref{ch:idx:lem:opt_pol}~\ref{it:opt_pol1} implies that $B_i^{\theta_i}(\vg^*,\vh^\pi) =0$ for any $i\in\Phi^\theta$.
    If there exists $i\in\Phi^\theta$ such that $\theta_i\neq\pi_i$, then the proof is concluded.
    Otherwise, $\theta_i=\pi_i$ for any state $i\in\Phi^\theta$.
    In such a case, $\theta$ and $\pi$ coincide for all recurrent states of $\theta$, and $\Phi^\theta=\Phi^\pi$.
    Moreover, as $\pi$ is unichain, $\theta$ is also unichain.
    Hence, Lemma~\ref{ch:idx:lem:equi_bias} implies that $B_i^{\theta_i}(\vg^*, \vh^\pi)=0$ for all state $i$.
    Since $\theta\neq\pi$, there exists at least one state $i\in[n]$ such that $\theta_i\neq\pi_i$.
\end{proof}
Lemma~\ref{ch:idx:lem:unicity_BO} implies that a Bellman optimal and unichain policy $\pi$ is the unique Bellman optimal policy if and only if $B_i^a(\vg^*, \vh^\pi)<0$ for all action $a\neq\pi_i$ and all state $i$.
%While this lemma provides a characterization of unicity for Bellman optimal policy in average reward criterion, the Bellman optimality coincides with the classical optimality in infinite-horizon discounted MDPs.
%So, the lemma can be equally used to characterize unicity of optimal policy in discounted MDPs.
To the best of our knowledge, this lemma is a new contribution to the literature of MDP.
Also, it will play an important role in the indexability testing and index computation in the next chapter.

Note that if a weakly communicating MDP has a single Bellman optimal policy, the policy must be unichain.
Indeed, suppose that, in a given weakly communicating MDP, there exists a Bellman optimal policy that is multichain with two recurrent classes $\Phi^1$ and $\Phi^2$.
Then, we can construct two unichain Bellman optimal policies:
%, each satisfies \eqref{eq:bias_opt} for some possibly different $\vh$ a solution to \eqref{eq:bias_opt}
a policy $\pi^1$ whose set of recurrent states is $\Phi^1$ and the other policy $\pi^2$ whose set of recurrent states is $\Phi^2$.
Since the MDP is weakly communicating, the optimal gain is state-independent: the optimal gain of any state in $\Phi^1$ is the same as the optimal gain of any state in $\Phi^2$.
So, both $\pi^1$ and $\pi^2$ are gain optimal.
Finally, since the original Bellman optimal policy is multichain, it is possible to construct two optimal bias vectors $\vh^{*1}$ and $\vh^{*2}$ such that $\pi^1$ satisfies \eqref{eq:bias_opt} with $\vh^{*1}$ and $\pi^2$ with $\vh^{*2}$.
Therefore, both $\pi^1$ and $\pi^2$ become Bellman optimal.
%These handmade unichain policies are Bellman optimal.

In MDPs that are not weakly communicating (see \figurename~\ref{ch:mdp:fig:mdp_class} and Definition~\ref{ch:mdp:defn:mdp_class}), it is possible that the unique Bellman optimal policy is multichain as given by the example in \figurename~\ref{fig:unique_mutichain}.
\begin{figure}[ht]
    \centering
    \begin{tabular}{cc}
        \begin{minipage}{.25\linewidth}
            \begin{tikzpicture}[on grid, state/.style={circle,draw}, >= stealth', auto, prob/.style = {inner sep=1pt,font=\scriptsize}]
                \node[state,color=blue]  (A) {$2$};
                \node[state,color=blue]  (B) [left =1.5cm of A]   {$1$};
                \path[->]
                    (A) edge[loop above,color=black] node{$0$} (A)
                    (A) edge[loop below,color=red, dashed] node{$0.5$} (A)
                    (B) edge[bend left, color=black] node{$0$} (A)
                    (B) edge[loop below, color=red, dashed] node[below]{$1$} (A);
            \end{tikzpicture}
        \end{minipage}
        &
        \begin{minipage}{.7\linewidth}
            The unique Bellman optimal policy is $\pi^*$ such that $\pi^*_1=0$ and $\pi^*_2=0$ (\ie, $\pi^*=\emptyset$).
            It is clear that $\pi^*$ is a multichain policy.
            This MDP is not weakly communicating because state $1$ is not accessible by state $2$, and there is no transient state.
        \end{minipage}
    \end{tabular}
    \caption{An example where the MDP is not weakly communicating, and the unique Bellman optimal policy is multichain.
        The black arrow shows state transition of action activate, and the red ones for action rest.
        The numbers along the arrows show the reward when executing the actions.
}
    \label{fig:unique_mutichain}
\end{figure}

\section{Conclusion}
\label{ch:idx:sec:conc}

In conclusion, we presented the ambiguities in the classical indexability's definition and proposed a new definition that clarified those ambiguities.
Along with this new definition of indexability, we gave the corresponding definition of the Whittle index.
We introduced a new notion of Bellman optimality for MDPs with the average reward criterion. We derived a technical lemma to check if a given Bellman optimal policy was unique.
This unicity property will play a crucial role in computing the Whittle index.

%This chapter about indexability terminates here.
In the next chapter, we will present a unified algorithm that can compute the Whittle index in both discounted and undiscounted restless Markovian bandit and the Gittins index in discounted rested bandit.

\endgroup
