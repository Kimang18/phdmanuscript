
%\dominitoc
%\faketableofcontents
\chapter{Markov Decision Process}
\label{ch:mdp}

%\minitoc

In this chapter, we define the notion of Markov decision process (MDP) used to mathematically describe the environment in Reinforcement learning (RL) framework.
A MDP models a discrete-time decision problem where the decision maker executes available ``action'' over time steps and receives immediate incentive known as ``reward'' for each time step.
In such problem, the decision maker seeks to maximize the expected cumulative rewards by identifying a sequence of actions that produces such effect.

In Section~\ref{ch:mdp:sec:defn}, we describe...

%\let\clearpage\relax

\section{Definitions and notations}
\label{ch:mdp:sec:defn}

In this section, we give the \emph{formalism} of Markov decision process and present several notations of \emph{optimality}.
We also lay out existing results (see \cite{puterman2014markov}) that are pillar of our contribution.
We essentially follow the notations of \cite{puterman2014markov}.

\subsection{State, action, reward, and state transition}

A Markov decision process $M$ is defined as a $4$-tuple $M:=\langle\gS, \gA, r, p\rangle$.
$\gS$ and $\gA:=\cup_{s\in\gS}\gA_s$ denote the \emph{state} and \emph{action} space of the MDP.
When the MDP is in state $s\in\gS$, the decision maker can execute one of the available actions in $\gA_s$.
As a result of executing action $a\in\gA_s$ in state $s$, the MDP incurs a random \emph{reward} with \emph{expected value} $r(s,a)$ and then transitions to new state $s'\in\gS$ with probability $p(s'\mid s, a)\in[0,1]$ where $\sum_{s'\in\gS}p(s'\mid s,a)=1$.
The name \emph{Markov} comes from the fact that the random reward and the next state depend only on state $s$ and action $a$ and are independent from anything else.
In this thesis, we consider the MDPs with \emph{finite} state and action spaces, $\abs{\gS}, \abs{\gA}\in\sN$.
Also, we assume that all random rewards are bounded and lie in $[0,1]$.
We provide an example of a MDP with two states and two actions per state in \figurename~\ref{fig:MDP_example}.

\begin{figure}[ht]
    \centering
    \begin{tikzpicture}[on grid, state/.style={circle,draw}, >= stealth', auto, prob/.style = {inner sep=1pt,font=\scriptsize}]
        \node[state,color=blue]  (A) {$s_1$};
        \node[state,color=blue]  (B) [left =3cm of A]   {$s_0$};
        \path[->]
            (A) edge[loop above, color=black] node{$a_1$} (A)
            (A) edge[loop right, color=red, dashed] node{$a_0$} (A)
            (B) edge[bend left, color=black] node{$a_1$} (A)
            (B) edge[bend right, color=red, dashed] node[below]{$a_0$} (A);
    \end{tikzpicture}
    \caption{Graphical representation of a MDP with $2$ states ($s_0$ and $s_1$) and 2 actions per state (black arrow and red dashed arrow). We have $p(s_0\mid s_0,a_0)=0, p(s_1\mid s_0,a_0)=1$ and so on.}
    \label{fig:MDP_example}
\end{figure}

\subsection{Sequential decision problem and policy}

In this thesis, the decision maker executes actions at \emph{discrete} time steps.
We denote the decision time by $t\ge1$ which is a positive integer.
At time step $t\ge1$, the MDP is in state $s_t$ and the decision maker executes an action $a_t$.
The MDP incurs a (random) reward denoted by $r_t$ and transitions to next state denoted by $s_{t+1}$.
This mechanism is repeated and one obtains a sequence of the form $\{s_1,a_1,r_1,\dots,s_t,a_t,r_t,s_{t+1},\dots\}$ that is called \emph{observations} (also known as a ``trajectory'').
We denote the observations up to time $t\ge1$ by
\begin{align*}
    O_t:=\{s_1,a_1,r_1,s_2,\dots,s_{t-1},a_{t-1},r_{t-1},s_t\}
\end{align*}
with a convention that $O_1:=\{s_1\}$.
A \emph{deterministic decision rule} $d$ is a function that maps the observations up to time $t$ to available actions at decision time $t$ with certainty, $d(O_t)\in\gA$.
The decision rules that depend on previous observations $\gO_t$ only through the current state $s_t$ of the MDP are called \emph{Markovian} decision rule, $d:\gS\mapsto\gA$.

In this thesis, we restrict our attention to \emph{deterministic Markovian decision rule} and a policy $\pi:=\{d_1,d_2,\dots\}$ is a sequence of deterministic Markovian decision rules for each time step such that at time $t\ge1$, the decision maker follows policy $\pi$ by executing the action $d_t(s_t)\in\gA$. 
We denote the set of all policies by $\Pi$.
A \emph{stationary} policy is a sequence of decision rules $\{d_t\}_{t\ge1}$ where $d_t=d$ for all $t\ge1$ and $d:\gS\mapsto\gA$ is a mapping function from state space to action space.
By abuse of notation, we will write $d$ and $\pi$ \emph{interchangeably} in the rest of the thesis when policy $\pi=\{d, d,...\}$ is stationary.
%Under policy $\pi$, the MDP becomes a Markov reward process (MRP) where the reward for each state is encoded by the vector $\vr^\pi$ and the state transition is governed by the matrix $\mP^\pi$.

The state of the MDP at time step $1$ is called \emph{initial state} and generally drawn from a probability distribution $\rho$ such that $\sum_{s\in\gS}\rho(s)=1$.
Given the initial state $s_1$, a policy $\pi\in\Pi$ incurs a sequence $\{s_1,a_1,r_1,\dots,s_t,a_t,r_t,\dots\}$ which is a \emph{stochastic process} with a well-defined probability distribution \cite[Section~2.1.6]{puterman2014markov}.
We will denote by $\sP^\pi(\cdot\mid s_1)$ the \emph{probability measure} associated with this stochastic process and denote by $\mathbb{E}^\pi[\cdot \mid s_1]$ the corresponding \emph{expectation}.


In the next sections, we specify the objective function of the decision problem.

%The policies that vary over time steps are called \emph{non-stationary} policies denoted by $\pi_t$ where $\pi_t(\gO_t)\in\gA_{s_t}$.
%The policies that do not vary over time steps are known as \emph{stationary} (or ``time-homogeneous'') policies.
%The time-dependent or not depending on the objective function of the decision problem.
%We talk more about this in the following sections.

\section{Finite horizon problems}

This section describes the \emph{finite horizon setting} and contains a brief presentation of optimality criterion in the setting.

In finite horizon setting, the decision maker can collect rewards from a MDP $M$ over a \emph{fixed} number of time steps called \emph{horizon} and denoted by $H\in\N^+$ where $\N^+:=\N\setminus\{0\}$ is the set of positive natural numbers.
Formally, the decision maker want to find a policy that verifies
%\begin{equation}
%    \label{eq:finite_obj}
%    \sup_{\pi\in\Pi}\sum_{s\in\gS}\rho(s)W_{M,1:H}^\pi(s) \text{ where } W_{M,1:H}^\pi(s){:=}\ex{\sum_{t=1}^H r_t\mid s_1{=}s, a_t{=}\pi_t(s_t)}.
%\end{equation}
\begin{equation}
    \label{eq:finite_obj}
    \sup_{\pi\in\Pi}\sum_{s\in\gS}\rho(s)\E^\pi\left[\sum_{t=1}^H r_t\mid s_1=s\right].
\end{equation}
From \cite[Chapter~4]{puterman2014markov}, there always exists an \emph{optimal} policy $\pi^*=\{d_1^*,d_2^*,\dots,d^*_H\}$ that maximizes \Eqref{eq:finite_obj} for any $\rho$ and such that for all $1\le t\le H$, $d_t^*$ is a deterministic Markovian decision rule and independent of the initial state distribution.
%For a given policy $\pi=\{d_1,\dots,d_H\}$, for each state $s$, we define the expected cumulative reward from $s$ over time steps $h$ to $H$ when following policy $\pi$ by $W_{M,h:H}^\pi(s){:=}\E^\pi\left[\sum_{t=h}^H r_t\mid s_h{=}s\right]$.
%The Bellman \emph{optimality} equations in this setting are written: for $1\le h\le H-1$,
%\begin{equation}
%    \label{eq:finite_be_opt}
%    W_{M,h:H}^{\pi^*}(s) = \max_{a\in\gA_s}\p{r(s,a)+\sum_{s'\in\gS}p(s'\mid s,a)W_{M,h+1:H}^{\pi^*}(s')}
%\end{equation}
%and $W_{M,H:H}^{\pi^*}(s) =\max_{a\in\gA_s}r(s,a)$.
%The function $W_{1:H}$ is the expected cumulative reward from time step $1$ to $H$.
%Suppose that at time step $1\le h\le H$, a policy $\pi$ executes action $a=d_h(s)$ in state $s$ where $a\in\gA_s$ .
%Then, the Bellman \emph{evaluation} equation in this setting is
%\begin{equation}
%    \label{eq:finite_be_eval}
%    W_{M,h}^{\pi}(s) = r(s,a)+\sum_{s'\in\gS}p(s'\mid s,a)W_{M,h+1}^{\pi}(s')
%\end{equation}
%with a convention that $W_{H+1}=0$.
Such an optimal policy $\pi^*$ can be constructed using backward induction given in Algorithm~\ref{algo:backward}.

\begin{algorithm}[ht]
    \DontPrintSemicolon
    \SetKwInOut{Input}{Input}\SetKwInOut{Output}{Output}
    \Input{a MDP $M=\langle\gS,\gA,r,p\rangle$ and horizon $H$}
    \Init{}{Set $u_{H+1}(s)=0$ for all $s\in\gS$}
    \BlankLine
    \For{$h=H$ to $1$}{
        Set $d_h^*(s) := \displaystyle\argmax_{a\in\gA_s}\Big(r(s,a)+\sum_{s'\in\gS}p(s'\mid s,a)u_{h+1}(s')\Big)$ \tcp{ties are broken arbitrarily} \;
        Set $u_{h}(s) = \displaystyle r\big(s, d_h^*(s)\big) +\sum_{s'\in\gS}p\big(s' \mid s, d_h^*(s)\big)u_{h+1}(s). \label{eq:finite_be_eval2}$
    }
    \Return $\pi^*=\{d_1^*,\dots,d_H^*\}$ and $\vu_1$
    \caption{Backward Induction}
    \label{algo:backward}
\end{algorithm}
By \cite[Theorem~4.5.1]{puterman2014markov}, Algorithm~\ref{algo:backward} computes an optimal policy and the maximum value of \Eqref{eq:finite_obj} is given by $\sum_{s\in\gS}\rho(s)u_{1}(s)$ where $\vu_1$ is the second output of the algorithm.
%Note that the last part of Equations~\eqref{eq:finite_be_eval1} and \eqref{eq:finite_be_eval2} are known as Bellman \emph{evaluation} equations.

The backward induction has $\landauO(H\abs{\gS}^2\abs{\gA})$ computational complexity because for each time step, state, and action, the inner product between $p$ and $\vu$ costs $\abs{\gS}$ multiplications at worst.

\section{Infinite horizon problems}

Similar the previous section, we describe two settings for infinite horizon and provide detail about the optimality criteria in this section.

\subsection{Discounted Markov decision process}

In some problems, the immediate reward incurred by the MDP is more important then those rewards in the future time steps.
To capture this aspect, one can introduce a discount factor denoted by $\gamma$ where $\gamma\in[0,1)$.
If $\gamma=0$, the decision maker is solely interested in the immediate reward from the current state of the MDP.
In general, the decision maker wants to find a policy that verifies
\begin{equation}
    \label{eq:discount_obj}
    \sup_{\pi\in\Pi}\sum_{s\in\gS}\rho(s)\E^\pi\left[\sum_{t=1}^{+\infty} \gamma^{t-1}r_t \mid s_1=s\right].
\end{equation}
From \cite[Chapter~6]{puterman2014markov}, there always exists a deterministic stationary optimal policy $\pi^*$ that maximizes \Eqref{eq:discount_obj} for any initial distribution $\rho$.
For a given stationary policy $\pi:\gS\mapsto\gA$, the expectation in \Eqref{eq:discount_obj} is known as the \emph{value} of state $s$ under policy $\pi$ and denoted by $v^\pi(s)$.
It captures the cumulative discounted reward one would expect when following policy $\pi$ starting from state $s$.
For any $s$, since $0\le\gamma<1$ and $r_t\in[0,1]$, $v^\pi(s)$ is a geometric series bounded between $0$ and $1/(1-\gamma)$.
So, $v^\pi(s)$ exists and well-defined for any state $s$ and any policy $\pi\in\Pi$.
The vector $\vv^\pi\in\R^{\abs{\gS}}$ is known as the \emph{value function} of policy $\pi$ and satisfies the Bellman \emph{evaluation} equations: for each $s\in\gS$
\begin{equation}
    \label{eq:discount_be_eval}
    v^\pi(s) =r(s,\pi(s)) +\gamma\sum_{s'\in\gS}p(s'\mid s,\pi(s))v^\pi(s').
\end{equation}
The maximum value of the expectation in \Eqref{eq:discount_obj} is called the \emph{optimal value} denoted by $v^*(s)$.
By \cite[Theorem~6.2.5]{puterman2014markov}, the optimal value function $\vv^*\in\R^{\abs{\gS}}$ is unique: any optimal policies have the same optimal value function.
Moreover, $\vv^*$ satisfies the Bellman \emph{optimality} equations: for each $s\in\gS$
\begin{equation}
    \label{eq:discount_be_opt}
     v^*(s)= \max_{a\in\gA_s}\Big(r(s, a) +\gamma\sum_{s'\in\gS}p(s'\mid s,a)v^*(s')\Big)
\end{equation}
It is well-known that an optimal policy can be constructed using iterative algorithms such as Value Iteration given in Algorithm~\ref{algo:discounted_vi}.
From \cite[Chapter~6]{puterman2014markov}, Algorithm~\ref{algo:discounted_vi} always converges: $\lim_{n\to\infty}\vv_n=\vv^*$.
That is, Algorithm~\ref{algo:discounted_vi} stops after a finite number of iterations and the policy $\pi$ that is the first output satisfies $\norm{\vv^\pi-\vv^*}\le \varepsilon$.
The maximum value of \Eqref{eq:discount_obj} is given by $\sum_{s\in\gS}\rho(s)v^*(s)$.

\begin{algorithm}[ht]
    \DontPrintSemicolon
    \SetKwInOut{Input}{Input}\SetKwInOut{Output}{Output}
    \Input{a MDP $M=\langle\gS,\gA,r,p\rangle$, discount factor $\gamma\in[0,1)$, and accuracy $\varepsilon\in(0,1)$}
    \Init{}{Set $n=0$ and $v_0(s)=0$ for all $s\in\gS$}
    \BlankLine
    Set $v_1(s)=\max_{a\in\gA_s} r(s,a)$ for all $s\in\gS$ \;
    \While{$\displaystyle\max_{s\in\gS}\big(v_{n+1}(s)-v_n(s)\big) -\min_{s\in\gS}\big(v_{n+1}(s)-v_n(s)\big) \ge \frac{1-\gamma}{\gamma}\varepsilon$}{
        Set $n=n+1$ \;
        Set $v_{n+1}(s) = \displaystyle\max_{a\in\gA_s}\Big(r(s,a)+\sum_{s'\in\gS}p(s'\mid s,a)v_{n}(s')\Big)$ for all $s\in\gS$
    }
    Set $\pi(s):=\displaystyle\argmax_{a\in\gA_s}\Big(r(s,a)+\sum_{s'\in\gS}p(s'\mid s,a)v_{n+1}(s')\Big)$ for all $s\in\gS$ \tcp{ties are broken arbitrarily}\;
    \Return $\pi$ and $\vv_{n+1}$
    \caption{Value Iteration}
    \label{algo:discounted_vi}
\end{algorithm}

Similar to backward induction, Algorithm~\ref{algo:discounted_vi} has $\landauO(\abs{\gS}^2\abs{\gA})$ computational complexity where $\landauO(\cdot)$ hides the total number of iterations.

This setting with discount factor $\gamma$ can be seen as a finite horizon setting whose horizon is randomly sampled from \emph{geometric distribution} with parameter $(1-\gamma)$.
That is, at each time step, the decision maker can stop interacting with the MDP with probability $(1-\gamma)$.
So, the value of state $s$ under deterministic stationary policy $\pi$ is the expected sum of rewards over the random horizon:
\begin{equation}
    \label{eq:discount_finite}
    v^\pi(s) :=\E^\pi\left[\sum_{t=1}^{+\infty} \gamma^{t-1}r_t \mid s_1=s\right] =\ex{\E^\pi\left[\sum_{t=1}^H r_t\mid H, s_1=s\right]}
\end{equation}
where the outer expectation of the third term integrates over the randomness of the horizon $H$ whose expected value is $1/(1-\gamma)$.

\subsection{Average reward model}

In some problems, the decision maker is more interested in the average reward per decision time than the expected cumulative discounted reward.
We focus on such criterion in this section.

\subsubsection{Gain and bias}

Formally, the decision maker wants to identify a policy that verifies
\begin{equation}
    \label{eq:avg_obj}
    \sup_{\pi\in\Pi}\sum_{s\in\gS}\rho(s)\liminf_{T\to+\infty}\frac1T \E^\pi\left[ \sum_{t=1}^{T} r_t \mid s_1=s\right].
\end{equation}
By the assumption that $r_t\in[0,1]$ and the state space is finite, the infimum limit of \Eqref{eq:avg_obj} equals the supremum limit for any deterministic stationary policy $\pi$ (see \cite[Chapter~8]{puterman2014markov}).
Hence, the limit of \Eqref{eq:avg_obj} exists and is called the \emph{long-term} average reward or \emph{gain} of state $s$ under policy $\pi$.
Concretely, it is defined by
\begin{equation}
    \label{eq:gain_defn}
    g^\pi(s) := \lim_{T\to+\infty}\frac1T \E^\pi\left[ \sum_{t=1}^{T} r_t \mid s_1=s\right].
\end{equation}
This notion generalizes both finite and discounted settings when $H\to+\infty$ or $\gamma\to1$ because it is shown (see \cite[Sections~8.2.1 and 8.2.2]{puterman2014markov}) that 
\begin{align*}
    \E^\pi\left[\sum_{t=1}^H r_t\mid s_1{=}s\right] \underset{H\to+\infty}{\sim} Hg^\pi(s) \text{ and }
    \E^\pi\left[\sum_{t=1}^{+\infty} \gamma^{t-1}r_t \mid s_1{=}s\right] \underset{\gamma\to1}{\sim}
    g^\pi(s)/(1-\gamma).
\end{align*}
In consequence, if $\pi$ and $\pi'$ are two deterministic stationary policies such that $g^\pi(s)\le g^{\pi'}(s)$, then for $H$ big enough and $\gamma$ close enough to $1$, $\E^\pi\left[\sum_{t=1}^H r_t\mid s_1{=}s\right]\le \E^{\pi'}\left[\sum_{t=1}^H r_t\mid s_1{=}s\right]$ and $\E^\pi\left[\sum_{t=1}^{+\infty} \gamma^{t-1}r_t \mid s_1{=}s\right]\le \E^{\pi'}\left[\sum_{t=1}^{+\infty} \gamma^{t-1}r_t \mid s_1{=}s\right]$.

The gain of stationary policy captures the average reward obtained in \emph{steady regime}.
Another quantity associated to stationary policy $\pi$ is its \emph{bias function} that is defined by: for each $s\in\gS$
\begin{equation}
    \label{eq:bias_defn}
    h^\pi(s) :=\Clim_{T\to+\infty} \E^\pi\left[ \sum_{t=1}^{T} r_t -g^\pi(s_t) \mid s_1=s\right].
\end{equation}
The bias function captures the expected total difference between the reward and the average reward in steady regime. 
The \emph{Cesaro-limit} denoted by $\mathrm{C}$-$\mathrm{lim}$ is used because it is well-defined for any stationary policies albeit the ``classical'' limit may not exist for stationary policy that induces a stochastic process governed by \emph{periodic} Markov chain.
The difference of bias values $h^\pi(s)-h^\pi(s')$ captures the (dis-)advantage of starting at state $s$ rather than $s'$ when following policy $\pi$.
We denote by $sp(\vh^\pi):=\max_{s\in\gS}h^\pi(s) -\max_{s\in\gS}h^\pi(s)$ the range or \emph{span} of the bias function of policy $\pi$.

Any deterministic stationary policy $\pi$ induces a Markov reward process (MRP) where the reward function and state transition are respectively encoded by vector $\vr^\pi$ and stochastic matrix $\mP^\pi$ such that for any $s,s'\in\gS$, $r^\pi(s):=r\big(s,\pi(s)\big)$ and $P^\pi(s,s'):=p\big(s'\mid s,\pi(s)\big)$.
We define the \emph{limiting matrix} $\bar{\mP}^\pi:=\Clim_{t\to+\infty}{(\mP^\pi)^t}$ (\cite[Appendix~A.4]{puterman2014markov}). % which describes the state transition in steady regime under policy $\pi$. That is, $\bar{P}^\pi(s,s')$ is the probability that the MDP transitions from state $s$ to $s'$ in steady regime.
Since the MDP has finite state space, $\bar{\mP}^\pi$ always exists and well-defined for any stationary policy $\pi$.
It describes the state transition in steady regime under policy $\pi$. That is, $\bar{P}^\pi(s,s')$ is the probability that the MDP transitions from state $s$ to $s'$ in steady regime.
So, the gain of policy $\pi$ can be expressed by for any $s\in\gS$, $g^\pi(s)=\sum_{s'\in\gS}\bar{P}^\pi(s,s')r^\pi(s')$.
In vector notation, $\vg^\pi=\bar{\mP}^\pi\vr^\pi$.
However, computing $\vg^\pi$ through $\bar{\mP}^\pi$ might be inefficient.
Instead, the gain and bias can be computed using Bellman evaluation equations given in Proposition~\ref{prop:be_eval}.
\begin{prop}[{\cite[Theorem~8.2.6]{puterman2014markov}}]
    \label{prop:be_eval}
    For any stationary policy $\pi$, the gain $\vg^\pi$ and bias $\vh^\pi$ are a solution of the following system of Bellman evaluation equations: for any $s\in\gS$
    \begin{align}
        g(s) -\sum_{s'\in\gS}P^\pi(s,s')g(s') &=0 \label{eq:gain_eval}\\
        g(s) -r^\pi(s) +h(s) -\sum_{s'\in\gS}P^\pi(s,s')h(s') &=0. \label{eq:bias_eval}
    \end{align}
    Suppose that $\vg$ and $\vh$ satisfy \eqref{eq:gain_eval} and \eqref{eq:bias_eval}. Then, $\vg^\pi=\vg$ and $\vh^\pi=\vh+\vu$ where $\vu=\mP^\pi\vu$.
    Finally, if $\bar{\mP}^\pi\vh=\vzero$, then $\vh^\pi=\vh$.
\end{prop}
%Equations~\eqref{eq:gain_eval} and \eqref{eq:bias_eval} uniquely define $\vg$ and determine $\vh$ up to an element of null space of $(\mI-\mP^\pi)$ where $\mI$ is the identity matrix of size $\abs{\gS}\times\abs{\gS}$.

Since \Eqref{eq:avg_obj} compares policies based on their average reward in steady regime, we need to take into account the \emph{chain structure} induced by the policies.
We assume that the reader is familiar with the notion of \emph{transient} and (positive) \emph{recurrent} states and/or class of a Markov chain(see \cite[Appendix~A]{puterman2014markov} or \cite{levin2017markov} for more detail) and classify the MDPs according to the following definition.
\begin{defn}[Classification of MDPs]
    \label{defn:mdp_struct}
    We say that a MDP is
    \begin{enumerate}
        \item \textbf{ergodic} if the Markov chain induced by any deterministic stationary policy has a single recurrent class that equals the state space (\ie, all states are visited infinitely often with probability $1$ independently of initial state);
        \item \textbf{unichain} if the Markov chain induced by any deterministic stationary policy has a single recurrent class plus a --possibly empty-- set of transient states;
        \item \textbf{communicating} if for every pair of states $(s,s')\in\gS$, there exists a deterministic stationary policy under which $s'$ is accessible from $s$ in finite time with non-zero probability
        \item \textbf{weakly communicating} if the state space can be partitioned into two subsets $\gS^C$ and $\gS^T$ (with $\gS^T$ possibly empty), such that for every pair of states $(s,s')\in\gS^C$, there exists a deterministic stationary policy under which $s'$ is accessible from $s$ in finite time with non-zero probability, and all states in $\gS^T$ are transient under all deterministic stationary policies.
    \end{enumerate}
    We say that a MDP is \textbf{multichain} if it is not unichain.
\end{defn}

\subsubsection{Optimality criteria}


\paragraph{Gain and Bellman optimality.}
Similarly to the discounted setting, in MDPs with finite state and action spaces, there always exists an optimal deterministic stationary policy $\pi^*$ that satisfies \Eqref{eq:avg_obj} for any initial distribution $\rho$ (see \cite[Theorem~9.1.8]{puterman2014markov}).
Such an optimal policy $\pi^*$ induces the \emph{optimal gain} denoted by $\vg^*$ and the maximum value of \Eqref{eq:avg_obj} is given by $\sum_{s\in\gS} \rho(s)g^*(s)$.
From \cite[Chapter~9]{puterman2014markov}, the optimal gain $\vg^*$ satisfies the system of Bellman \emph{optimality} equations: for each $s\in\gS$,
\begin{align}
    g(s) &= \max_{a\in\gA_s} \sum_{s'\in\gS}p(s'\mid s,a)g(s') \label{eq:gain_opt} \\
    g(s) +h(s) &= \max_{a\in\gA_s}\Big(r(s,a) +\sum_{s'\in\gS}p(s'\mid s,a)h(s')\Big) \label{eq:bias_opt}.
\end{align}
So, $\vg^*$ is uniquely defined \eqref{eq:gain_opt} and \eqref{eq:bias_opt}.
Any policies that achieve $\vg^*$ are said to be \emph{gain optimal} (or average optimal): for any $\pi\in\Pi$, $g^\pi(s)\le g^*(s)$ for all $s\in\gS$.
The gain optimal policies give more importance to the reward in steady regime. 
That is, they act optimally in recurrent states and can possibly ignore transient states.
In this thesis, we distinguish gain optimal policies from the policies that attain the maximum in \eqref{eq:bias_opt} as the following
\begin{defn}[Bellman optimal policy]
    %Consider a MDP in which the space $\gS$ is finite and $\gA_s$ is finite for any $s\in\gS$.
    A policy $\pi$ is \emph{Bellman optimal} if there exists vector $\vh\in\R^{\abs{\gS}}$ that satisfies \eqref{eq:bias_opt} with $\vg^*$ such that $\pi(s)\in\argmax_{a\in\gA_s}\Big(r(s,a) +\sum_{s'\in\gS}p(s'\mid s,a)h(s')\Big)$ for all $s\in\gS$.
\end{defn}
By \cite[Theorem~9.1.7]{puterman2014markov}, Bellman optimal policy always exists in MDPs with finite state and action spaces and Bellman optimal implies gain optimal.
However, the converse is not true in general.
So, the notion of Bellman optimality is stronger than the notion of gain optimality.

\paragraph{Characterization of gain optimal policies.}

Given a deterministic stationary policy $\pi$, let $\vg$ and $\vh$ be a solution of \eqref{eq:gain_eval} and \eqref{eq:bias_eval}.
We define the advantage of action\footnote{Note that it is in function of $\vh$} $a\in\gA_s$ in state $s$ over policy $\pi$ by
\begin{equation}
    \label{eq:advan}
    B^a_s(\vh) := r(s,a) +\sum_{s'\in\gS}p(s'\mid s,a)h(s') -g(s) -h(s).
\end{equation}
The following lemma characterizes gain optimal policy by showing that the policy must satisfies \eqref{eq:bias_opt} on their recurrent states.
\begin{lem}
    \label{lem:opt_pol}
    Let $\pi:\gS\mapsto\gA$ be a policy and $\Phi^\pi$ be the set of recurrent states of policy $\pi$.
    The three properties below are equivalent.
    \begin{enumerate}[label=(\roman*)]
        \item \label{it:opt_pol1} $\mP^\pi\vg^*=\vg^*$ and for all $\vh\in \R^{\abs{\gS}}$ satisfying \eqref{eq:bias_opt}, $B^{\pi(s)}_s(\vh)=0$ for all $s\in\Phi^\pi$
        \item \label{it:opt_pol2} $\mP^\pi\vg^*=\vg^*$ and for some $\vh\in \R^{\abs{\gS}}$ satisfying \eqref{eq:bias_opt}, $B^{\pi(s)}_s(\vh)=0$ for all $s\in\Phi^\pi$
        \item \label{it:opt_pol3} $\pi$ is gain optimal.
    \end{enumerate}
\end{lem}
\begin{proof}
    \ref{it:opt_pol1} $\Rightarrow$ \ref{it:opt_pol2} is trivial.

    \ref{it:opt_pol2} $\Rightarrow$ \ref{it:opt_pol3}: By definition of $B_i^a(\vh)$, we have $r^{\pi}(s)-g^*(s) = h(s)-\sum_{s'\in\gS} P^{\pi}(s,s')h(s')$ for any recurrent state $s$ of $\pi$.  Multiply this with $\bar{P}^\pi(j,s)$ and sum over $s\in\gS$ (if $s$ is not recurrent, then $\bar{P}^\pi(j,s)=0$) gives
    \begin{align*}
        \sum_{s\in\gS} \bar{P}^\pi(j,s)\Big(r^{\pi}(s)-g^*(s)\Big) = \sum_{s\in\gS} \bar{P}^\pi(j,s)h(s) - \underbrace{\sum_{s\in\gS} \bar{P}^\pi(j,s)\sum_{s'\in\gS} P^{\pi}(s,s')h(s')}_{=\sum_{s'\in\gS} \bar{P}^\pi(j,s')h(s') \text{ since $\bar{\mP}^\pi\mP^\pi=\bar{\mP}^\pi$.}}
        &= \vzero.
    \end{align*}
    By Theorem~8.2.6 of \cite{puterman2014markov}, the gain of $\pi$ is $\bar{\mP}^\pi \vr^\pi$. The above equation shows that $\bar{\mP}^\pi\vr^\pi = \bar{\mP}^\pi\vg^*$. Moreover, the assumption  $\mP^\pi\vg^*=\vg^*$ implies that $\bar{\mP}^\pi\vg^*=\vg^*$ which in turn implies that $\bar{\mP}^\pi\vr^\pi=\vg^*$. This shows that the gain of $\pi$ is $\vg^*$ and therefore $\pi$ is gain optimal.
    % So, we have $\mP^\pi\vg^*=\vg^*$ and $\bar{\mP}^\pi(\vr^\pi-\vg^*)=\vzero$.

    \ref{it:opt_pol3} $\Rightarrow$ \ref{it:opt_pol1}: If $\pi$ is gain optimal, then $\mP^\pi \vg^*=\vg^*$ and $\bar{\mP}^\pi(\vr^\pi-\vg^*)=\vzero$.
    The latter rewrites as $\sum_{s\in\gS}\bar{P}^\pi(j,s)\Big(r^{\pi}(s)-g^*(s)\Big) =0$ for all state $j$. Let $\vh$ be a solution of \eqref{eq:bias_opt}.
    For all state $j$, we have    
    \begin{align}
        \label{eq:apx_proof_H}
        \sum_{s\in\gS}\bar{P}^\pi(j,s)B^{\pi(s)}_s(\vh) {=} \sum_{s\in\gS}\bar{P}^\pi(j,s)\Big(r^{\pi}(s)-g^*(s) {+}\sum_{s'\in\gS}P^{\pi}(s,s')h(s') -h(s)\Big) {=}0.
    \end{align}
    As $\vh$ satisfied \eqref{eq:bias_opt}, for all action $a\in\gA_s$, we have $B^{a}_s(\vh) \le0$ for all states $s\in\gS$ and in particular $B^{\pi(s)}_s(\vh) \le0$.
    This shows that for any state $s$ such that $\bar{P}^\pi(j,s)>0$, one must have $B^{\pi(s)}_s(\vh) =0$. Such state $s$ are the recurrent states of $\pi$.
    This shows that $B^{\pi(s)}_s(\vh) =0$ for all $s\in\Phi^\pi$.
\end{proof}

\paragraph{Characterization of Bellman optimal policies.}

The previous lemma shows that a policy is gain optimal if and only if the actions for the recurrent states of the policy satisfy \eqref{eq:bias_max}.
The following lemma shows the relationship between two policies that are unichain and satisfy \eqref{eq:bias_max} on all states.

\begin{lem}
    \label{lem:equi_bias}
    Suppose that two policies $\pi$ and $\theta$ are Bellman optimal, unichain and have at least one common recurrent state: $\gR^\pi\cap\gR^\theta\neq\emptyset$.
    
    Then for any $\vh^\pi$ and $\vh^\theta$ solutions of \eqref{eq:bias_eval1} for $\pi$ and $\theta$ respectively, there exists a constant $c$ such that for all state $i$: $h^\pi_i -h^\theta_i =c$. Moreover, in this case, ${B_i^{\theta_i}(\vh^\pi)=B_i^{\pi_i}(\vh^\theta)=0}$ for all $i$.
\end{lem}
\begin{proof}
    Since $\pi$ and $\theta$ are Bellman optimal, $\vh^\pi,\vh^\theta\in H$.
    In consequence, we have
    \begin{align*}
        \vh^\pi \ge \vr^\theta -\vg^* +\mP^\theta\vh^\pi.
    \end{align*}
    By Lemma~\ref{lem:opt_pol}~\ref{it:opt_pol1}, the above inequality is an equality for all $i\in\gR^\theta$ because $\theta$ is gain optimal.

    As $\vh^\theta$ satisfies \eqref{eq:bias_eval1}, we have
    \begin{align*}
        \vh^\theta -\vh^\pi &\le \vr^\theta -\vg^* +\mP^\theta\vh^\theta -(\vr^\theta -\vg^* +\mP^\theta\vh^\pi) = \mP^\theta(\vh^\theta -\vh^\pi),
    \end{align*}
    with equality for all state $i\in\gR^\theta$. This shows that for all $t$, $\vh^\theta -\vh^\pi\le (\mP^\theta)^t(\vh^\theta -\vh^\pi)$ which implies that $\vh^\theta -\vh^\pi\le \bar{\mP}^\theta(\vh^\theta -\vh^\pi)$ with equality for all states $i\in\gR^\theta$. Similarly, $\vh^\pi -\vh^\theta \le \bar{\mP}^\pi(\vh^\pi -\vh^\theta)$ with equality for any state $i\in\gR^\pi$.

    Let $c^\pi_i=\sum_{j\in\gS}\bar{P}^\pi_{ij}(h_j^\pi-h_j^\theta)$ and $c^\theta_i=\sum_{j\in\gS}\bar{P}^\theta_{ij}(h_j^\pi-h_j^\theta)$. By what we have just shown, for all state $i$, we have
    \begin{align*}
        c^\theta_i \underbrace{\le}_{\text{equality if $i\in\gR^\theta$}} h_i^\pi-h_i^\theta \underbrace{\le}_{\text{equality if $i\in\gR^\pi$}} c^\pi
    \end{align*}
    As both policies are unichain, $c^\pi_i$ and $c^\theta_i$ do not depend on $i$.  Moreover, if there exists $i\in\gR^\theta\cap\gR^\pi$, we have $c^\pi_i=c^\theta_i =: c$. In consequence,  $h_i^\pi-h_i^\theta=c$ for all state $i$. 
\end{proof}


\paragraph{Unicity of Bellman optimal policy.}
\label{ssec:unicity}

\begin{lem}
    \label{lem:unicity_BO}
    Let $\pi$ be a Bellman optimal policy that is unichain. If $\pi$ is not the unique Bellman optimal policy, then there exists a state $i$ and an action $a\neq\pi_i$ such that $B_i^a(\vh^\pi)=0$.
\end{lem}
\begin{proof}
    Let $\theta\neq\pi$ be another Bellman optimal policy. Since $\theta$ is gain optimal and $\vh^\pi\in H$, Lemma~\ref{lem:opt_pol} implies that $B_i^{\theta_i}(\vh^\pi) =0$ for all $i\in\gR^\theta$. If there exists $i\in\gR^\theta$ such that $\theta_i\neq\pi_i$, then the proof is concluded.  Otherwise, $\theta_i=\pi_i$ for all $i\in\gR^\theta$. This show that $\pi$ and $\theta$ coincide for all recurrent states of $\theta$ and that $\gR^\theta=\gR^\pi$. Moreover, as $\pi$ is unichain, $\theta$ is also unichain. Hence, Lemma~\ref{lem:equi_bias} implies that $B_i^{\theta_i}(\vh^\pi)=0$ for all $i$. Since $\theta\neq\pi$, there exists at least one state $i\in\gS$ such that $\theta_i\neq\pi_i$.
\end{proof}



%TOCONTINUE

