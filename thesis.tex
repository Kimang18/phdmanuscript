%!TEX encoding = UTF-8 Unicode

% The structure of this document is based on the my-thesis.tex example provided
% with the cleanthesis package (see http://cleanthesis.der-ric.de/).
% It also is inspired by the PhD manuscripts of Raphaël Bleuse, David Beniamine
% (see https://github.com/dbeniamine/Ph.D_thesis) and David Glesser.
% Uneeded packages and comments have been stripped out.

\pdfobjcompresslevel 0  % Produce PDF file respecting CINES requirements

\documentclass[%
    paper=A4,              % paper size
    twoside=true,          % onesite or twoside printing
    openright,             % doublepage cleaning ends up right side
    parskip=half,          % spacing value / method for paragraphs
    chapterprefix=true,    % prefix for chapter marks
    11pt,                  % font size
    headings=normal,       % size of headings
    bibliography=totoc,    % include bib in toc
    listof=totoc,          % include listof entries in toc
    titlepage=on,          % own page for each title page
    captions=tableabove,   % display table captions above the float env
    chapterprefix=false,   % do not display a prefix for chapters
    appendixprefix=false,  % do not display a prefix for appendix chapters
    draft=false,           % value for draft version
]{scrreprt}


% === PACKAGES IMPORT / CONFIGURATION =========================================
% --- ENCODING / FONTS -----------------

\usepackage[utf8]{inputenc}
\usepackage[T1]{fontenc}
\usepackage{textcomp}
\usepackage{lipsum}

% --- MATHEMATICS TYPESETTING ----------

\usepackage{amsmath}
\usepackage{amssymb}
\usepackage{amsthm}
\usepackage{bm}

% --- SPECIFIC ABBREVIATIONS MACROS -------------

% *** SPECIFIC MACROS ***
% specific macro definition for the paper
%
% requires amsthm, xspace

% theorem-like environment
\newtheorem{thm}{Theorem}
\newtheorem{lem}[thm]{Lemma}
\newtheorem{prop}{Proposition}
\newtheorem{propty}{Property}
\newtheorem{defn}{Definition}

% asymptotic complexity (Landau) notations
\newcommand{\landauO}{\ensuremath{\mathcal{O}}\xspace}
\newcommand{\landauOmega}{\ensuremath{\Omega}\xspace}
\newcommand{\landauTheta}{\ensuremath{\Theta}\xspace}
\newcommand{\landauo}{\ensuremath{o}\xspace}
\newcommand{\landauomega}{\ensuremath{\omega}\xspace}
\newcommand{\landauorder}{\ensuremath{\sim}\xspace}

% scheduling notations
\newcommand{\graham}[3]{\mbox{\ensuremath{#1\mid#2\mid#3}}}
\newcommand{\Cmax}{\ensuremath{C_{\max}}\xspace}

% complexity classes
\newcommand{\cNP}{\textbf{NP}\xspace}  % bold notations
\newcommand{\cP}{\textbf{P}\xspace}
% \newcommand{\cNP}{\ensuremath{\mathcal{N\!P}}\xspace}  % round notations
% \newcommand{\cP}{\ensuremath{\mathcal{P}}\xspace}

% Machine states
\newcommand{\computing}{computing}
\newcommand{\idle}{idle}
\newcommand{\off}{of\!f}
\newcommand{\on}{on}
\newcommand{\ontooff}{\on\rightarrow\off}
\newcommand{\offtoon}{\off\rightarrow\on}

% CCGRID 2017
\newcommand{\ra}[1]{\renewcommand{\arraystretch}{#1}}

% Cluster 2017
\newcommand{\overbar}[1]{\mkern 1.5mu\overline{\mkern-1.5mu#1\mkern-1.5mu}\mkern 1.5mu}

\newlength\mylen % algorithm2e hack
\newcommand\myinput[1]{%
  \settowidth\mylen{\KwIn{}}%
  \setlength\hangindent{\mylen}%
  \hspace*{\mylen}#1\\}

\newcommand\myoutput[1]{% algorithm2e hack
  \settowidth\mylen{\KwOut{}}%
  \setlength\hangindent{\mylen}%
  \hspace*{\mylen}#1\\}


% Float environments
\renewcommand{\topfraction}{0.8}
\renewcommand{\bottomfraction}{0.8}



%%%%%%%%%%%%%%%%%%%%%%%
% Modular compilation %
%%%%%%%%%%%%%%%%%%%%%%%
\newif\ifwatermark
\watermarkfalse

\newif\iftotalcompilation
\totalcompilationtrue

\newcommand{\inputchapter}[2]{%
  \ifdef{#1}%
    {\input{#2}\cleardoublepage}%
    {}%
}

% Modeling notations
\newcommand{\model}[2][]{\ensuremath{\bm{M}_{#1}\ifthenelse{\equal{#2}{}}{}{\!-\!}{#2}}\xspace}
\newcommand{\modelp}[2][]{\ensuremath{\bm{M'}_{#1}\!\ifthenelse{\equal{#2}{}}{}{\!-\!}{#2}}\xspace}
\newcommand{\noise}[2][]{\ensuremath{\bm{N}_{#1}\ifthenelse{\equal{#2}{}}{}{\!-\!}{#2}}\xspace}
\newcommand{\noisep}[2][]{\ensuremath{\bm{N'}_{#1}\!\ifthenelse{\equal{#2}{}}{}{\!-\!}{#2}}\xspace}
\newcommand{\norm}{\ensuremath{\mathcal{N}}\xspace}
\newcommand{\mcdots}{\ensuremath{\!\cdot\!\cdot\!\cdot\!}\xspace}
\newcommand{\unif}[2]{\ensuremath{\mathcal{U}\left({#1},{#2}\right)}\xspace}

% Abreviations
\newcommand{\eg}{e.g.\@\xspace}
\newcommand{\ie}{i.e.\@\xspace}
\newcommand{\aka}{a.k.a.\@\xspace}
\newcommand{\resp}{resp.\@\xspace}
\newcommand{\etal}{et~al.\@\xspace}
\newcommand{\dgemm}{\texttt{dgemm}\@\xspace}
\newcommand{\recv}{\texttt{MPI\_Recv}\@\xspace}
\newcommand{\send}{\texttt{MPI\_Send}\@\xspace}
\newcommand{\isend}{\texttt{MPI\_Isend}\@\xspace}
\newcommand{\iprobe}{\texttt{MPI\_Iprobe}\@\xspace}
\newcommand{\pyce}{\texttt{pycewise}\@\xspace}
\newcommand{\peanut}{\texttt{peanut}\@\xspace}
\newcommand{\cashew}{\texttt{cashew}\@\xspace}
\newcommand{\execo}{\texttt{execo}\@\xspace}
\newcommand{\enoslib}{\texttt{EnOSlib}\@\xspace}
\newcommand{\cluster}[2][]{\textcolor{myblue}{\texttt{#2\ifthenelse{\equal{#1}{}}{}{-#1}}}\@\xspace}
\newcommand{\dahu}[1][]{\cluster[#1]{dahu}}
\newcommand{\dada}[1]{\cluster[#1]{dahu}}  % for some reason, optionnal arguments do not work in subfigure, so I need this crap...
\newcommand{\drac}{\cluster{drac}}
\newcommand{\yeti}[1][]{\cluster[#1]{yeti}}
\newcommand{\troll}[1][]{\cluster[#1]{troll}}
\newcommand{\gros}{\cluster{gros}}
\newcommand{\grisou}[1][]{\cluster[#1]{grisou}}
\newcommand{\grvingt}{\cluster{grvingt}}
\newcommand{\paravance}{\cluster{paravance}}
\newcommand{\parasilo}[1][]{\cluster[#1]{parasilo}}
\newcommand{\paranoia}{\cluster{paranoia}}
\newcommand{\chiclet}{\cluster{chiclet}}
\newcommand{\chetemi}{\cluster{chetemi}}
\newcommand{\pyxis}{\cluster{pyxis}}
\newcommand{\nova}{\cluster{nova}}
\newcommand{\taurus}{\cluster{taurus}}
\newcommand{\ecotype}[1][]{\cluster[#1]{ecotype}}

% Referencing several times a footnote
% See https://tex.stackexchange.com/a/54240
\newcommand{\savefootnote}[2]{\footnote{\label{#1}#2}}
\newcommand{\repeatfootnote}[1]{\textsuperscript{\ref{#1}}}

\input{macros.include.tex}

\ifwatermark
% https://texblog.org/2012/02/17/watermarks-draft-review-approved-confidential/
\usepackage{draftwatermark}
\usepackage{datetime}
 \SetWatermarkColor[rgb]{0.8,0.9,1}
 \SetWatermarkText{%
    \begin{minipage}{6in}
    \begin{center}
      \settimeformat{hhmmsstime}
      {\fontsize{24}{12}\selectfont DRAFT\\\today, \currenttime}
    \end{center}
    \end{minipage}
  }
\SetWatermarkScale{1}

\fi

% \usepackage{kpfonts}

% --- i18n, l10n -----------------------

\usepackage[french,english]{babel}  % load default language last
\usepackage[french,english]{isodate}  % load default language last

% --- PAGE SETTING ---------------------

\usepackage[%
    figuresep=colon,          % label separator for captions
    hangfigurecaption=false,  % use hanging figure label (within margin)
    hangsection=true,         % use hanging section label (within margin)
    hangsubsection=true,      % use hanging subsection label (within margin)
    colorize=full,            % define color mode
    colortheme=bluemagenta,   % what colors to use
    bibsys=biber,             % citation manamgement engine
    bibfile=references,       % bibtex file
    bibstyle=alphabetic,      % reference style
]{cleanthesis}

% In some 'online' bibtex entries, we do have a DOI, thanks to Zenodo. Here we want to display it.
\AtEveryBibitem{%
    \ifentrytype{online}{
        \csappto{blx@bbx@\thefield{entrytype}}{% put at end of entry
            \iffieldundef{doi}{}{%
                \printfield{doi}
            }
        }
    }
}

% --- GRAPHICS / FIGURES ---------------

\usepackage{epsfig}  % XXX: to be removed when CCPE figures are out

\newcommand{\SetFigFont}[3]{\fontsize{#1}{#2pt}\normalfont\sffamily}  % XXX: to be removed when CCPE figures are out

\usepackage{tikz}
\usepackage{pgfplots}
\pgfplotsset{compat=1.13}
\usetikzlibrary{arrows,shapes,positioning,shadows,trees,calc,decorations.text}


%\usepackage{titlesec}
%titlesec is outdated and create Warning Missing number blablabla
%https://tex.stackexchange.com/questions/302111/multiple-missing-number-treated-as-zero-and-illegal-unit-of-measure-pt-insert
\makeatletter
\renewcommand{\sectionlinesformat}[4]{%
  \Ifstr{#1}{section}{\clearpage}{}%
  \@hangfrom{\hskip #2#3}{#4}%
}
\makeatother

% --- MODULAR CHAPTER COMPILATION -----

\usepackage{etoolbox}

% --- MISC. PACKAGES -------------------

\usepackage[a4paper]{uga}  % UGA title page style (see titlepage.tex)
\usepackage{xspace}  % easy macros definition
\usepackage{booktabs}
%\usepackage[font=footnotesize]{subfig}
\usepackage{subcaption}
\usepackage[linesnumbered,ruled,noend,vlined]{algorithm2e}
\SetKwProg{Init}{Init.}{}{}
\newenvironment{subroutine}[1][htb]
{\renewcommand{\algorithmcfname}{Subroutine}% Update algorithm name
   \begin{algorithm}[#1]%
}{\end{algorithm}}


\usepackage{csquotes}
\usepackage{tabularx}
\usepackage{pifont}
\newcommand{\cmark}{\ding{51}}%
\newcommand{\xmark}{\ding{55}}%
\usepackage{pdflscape}

\usepackage[binary-units,group-digits,group-separator={,}]{siunitx}
% Uncomment to have stuff like 6.7e-11, keep it commented for 6.7\times10^{-11}
%\sisetup{output-exponent-marker=\ensuremath{\mathrm{e}}}
\DeclareSIUnit\flop{Flop}
\DeclareSIUnit\flops{\flop\per\second}
\newcommand{\Num}[1]{\num[group-separator={,}]{#1}\xspace}
\newcommand{\NSI}[2]{\SI[group-separator={,}]{#1}{#2}\xspace}
\DeclareSIUnit\core{core}

\usepackage{color,colortbl}
\definecolor{gray98}{rgb}{0.98,0.98,0.98}
\definecolor{gray95}{rgb}{0.95,0.95,0.95}
\definecolor{gray20}{rgb}{0.20,0.20,0.20}
\definecolor{gray25}{rgb}{0.25,0.25,0.25}
\definecolor{gray16}{rgb}{0.161,0.161,0.161}
\definecolor{gray80}{rgb}{0.8,0.8,0.8}
\definecolor{gray60}{rgb}{0.6,0.6,0.6}
\definecolor{gray30}{rgb}{0.3,0.3,0.3}
\definecolor{bgray}{RGB}{248, 248, 248}
\definecolor{amgreen}{RGB}{77, 175, 74}
\definecolor{amblu}{RGB}{55, 126, 184}
\definecolor{amred}{RGB}{228,26,28}
\definecolor{amdove}{RGB}{102,102,122}
\usepackage{xcolor}
\definecolor{myblue}{cmyk}{1, .50, .10, .01}  % see definition of "bluemagenta" in cleanthesis.sty
\usepackage[procnames]{listings}
\lstset{ %
    backgroundcolor=\color{gray98},   % choose the background color; you must add \usepackage{color} or \usepackage{xcolor}
    basicstyle=\ttfamily\scriptsize,  % the size of the fonts that are used for the code
    breakatwhitespace=false,          % sets if automatic breaks should only happen at whitespace
    breaklines=true,                  % sets automatic line breaking
    showlines=true,                   % sets automatic line breaking
    captionpos=b,                     % sets the caption-position to bottom
    commentstyle=\color{gray60},      % comment style
    extendedchars=true,               % lets you use non-ASCII characters; for 8-bits encodings only, does not work with UTF-8
    frame=single,                     % adds a frame around the code
    keepspaces=true,                  % keeps spaces in text, useful for keeping indentation of code (possibly needs columns=flexible)
    columns=flexible,
    keywordstyle=\color{amblu},       % keyword style
    procnamestyle=\color{colorfuncall},      % procedures style
    language=[95]fortran,             % the language of the code
    numbers=left,                     % where to put the line-numbers; possible values are (none, left, right)
    numbersep=5pt,                    % how far the line-numbers are from the code
    numberstyle=\tiny\color{gray20},  % the style that is used for the line-numbers
    rulecolor=\color{gray20},         % if not set, the frame-color may be changed on line-breaks within not-black text (\eg comments (green here))
    showspaces=false,                 % show spaces everywhere adding particular underscores; it overrides 'showstringspaces'
    showstringspaces=false,           % underline spaces within strings only
    showtabs=false,                   % show tabs within strings adding particular underscores
    stepnumber=2,                     % the step between two line-numbers. If it's 1, each line will be numbered
    stringstyle=\color{amdove},       % string literal style
    tabsize=2,                        % sets default tabsize to 2 spaces
    % title=\lstname,                 % show the filename of files included with \lstinputlisting; also try caption instead of title
    procnamekeys={call}
}
\definecolor{colorfuncall}{rgb}{0.6,0,0}

% --- PDF SUPPORT ----------------------

\usepackage{bookmark,hyperref}  % import as last package

%% Citing software
\ExecuteBibliographyOptions{
    halid=true,
    swhid=true,
    swlabels=true,
    vcs=true,
    license=true
}

\hypersetup{
    unicode=true,
    plainpages=false,
    colorlinks=false,
    pdfborder={0 0 0},
    breaklinks=true,  % allow line breaks inside links
    bookmarksnumbered=true,
    bookmarksopen=true,
}

% --- Custom TODO line ------------------------
%\usepackage{mdframed}
%\newmdenv[%
%  linecolor=red,%
%  backgroundcolor=red!20,%
%  linewidth=3pt,%
%  hidealllines=true,%
%  leftline=true,%
%]{todoenv}
%\newcommand{\todo}[1]{\begin{todoenv}#1\end{todoenv}}
%\newcommand{\todoC}[1]{\todo{#1}}
\usepackage[textwidth=\linewidth]{todonotes}		% for todo's
\renewcommand{\NG}[1]{\todo[color=blue!10,author=\textbf{\small NG},inline]{\small #1\\}}
\newcommand{\BG}[1]{\todo[color=red!10,author=\textbf{\small BG},inline]{\small #1\\}}
\newcommand{\KK}[1]{\todo[color=green!10,author=\textbf{\small KK},inline]{\small #1\\}}
\graphicspath{{}{img/}}

% -- command for box colored --
\usepackage[most]{tcolorbox}
\tcolorboxenvironment{thm}{
    enhanced,
    breakable,
    fonttitle=\bfseries,
    colback=RoyalBlue!5,
    colframe=RoyalBlue!30!black,
}
\tcolorboxenvironment{prop}{
    enhanced,
    breakable,
    boxrule=0pt,frame hidden,
    borderline west={2pt}{0pt}{RoyalBlue!90!black},
    colback=gray95,
    sharp corners,
}
\tcolorboxenvironment{defn}{
    colback=gray95,
    colframe=amgreen, 
    rightrule=0pt, toprule=0pt,
    enhanced,
    breakable,
    sharp corners,
}
\tcolorboxenvironment{lem}{
    enhanced,
    breakable,
    boxrule=0pt,frame hidden,
    borderline west={4pt}{0pt}{RoyalBlue!60!black},
    colback=gray95,
    sharp corners,
}



% === MACROS DEFINITION =======================================================

% --- COMMON ABBREVIATIONS MACROS -------------

% latin abbreviations, see:
%   - http://www.sussex.ac.uk/informatics/punctuation/capsandabbr/abbr
% comment by Sascha Hunold, see also:
%   - https://www.ieee.org/documents/style_manual.pdf
%     (p. 32, Short Reference List of Italics)
%   - http://web.ece.ucdavis.edu/~jowens/commonerrors.html

% === META DATA ===============================================================
\title{Indexability and Learning Algorithms for Markovian Bandits}
\author{Kimang \textsc{Khun}}

% fill pdf meta data
\makeatletter
\hypersetup{pdftitle=\@author's thesis}
\hypersetup{pdfsubject=\@title}
\hypersetup{pdfauthor=\@author}
\makeatother

\usepackage{enumitem}
\setlistdepth{9}
%\usepackage{minitoc}

% === DOCUMENT CONTENT ========================================================

%% Comment the following to have chapters numbered without interruption (numbering through parts)
%\makeatletter\@addtoreset{chapter}{part}\makeatother%

\begin{document}

% --- FRONT MATTER ---------------------

\iftotalcompilation
\pdfbookmark[-1]{Front Matter}{Front Matter}

\pagenumbering{gobble}  % do not count title page
\pagestyle{empty}  % no header nor footer

%!TEX encoding = UTF-8 Unicode

\begin{titlepage}
%
\pdfbookmark[0]{Cover}{Cover}
%
\begin{otherlanguage}{french}

% UGA meta data ------------------------

% \Universite{}
% \Grade{}
\Specialite{Mathématique et Informatique}
\Arrete{\printdate{2016-05-25}}
\Directeur{Bruno \textsc{Gaujal}, Inria}
\CoDirecteur{Nicolas \textsc{Gast}, Inria}
\Laboratoire{Laboratoire d'Informatique de Grenoble}
\EcoleDoctorale{MSTII}
\Depot{\printdate{2023-03-30}} % XXXXXXXXXXXXXXXXXXXXXXXXXXXXXXXXXXXXXX

\makeatletter
\Auteur{{\@author}}
\Titre{Des Algorithmes pour les Bandits\\ Markoviens: Indexabilité et Apprentissage}
\Soustitre{\foreignlanguage{english}{\@title}}
\makeatother

% UGA meta data: jury ------------------

\Jury{
    \UGTRapporteur{Konstantin \textsc{Avrachenkov}}{Researcher, Inria, France}

    \UGTRapporteur{Aditya \textsc{Mahajan}}{Professor associate, McGill University, Canada}
    
    \UGTExaminatrice{Ana \textsc{Busic}}{Researcher, Inria, France}

    %\UGTDirecteur{Bruno \textsc{Gaujal}}{Researcher, Inria, France}

    \UGTExaminateur{Franck \textsc{Iutzeler}}{Maître de conférences, Université Grenoble-Alpes, France}

    \UGTPresident{Aurélien \textsc{Garivier}}{Professor, École Normale Supérieure de Lyon, France}
    %\UGTExaminateur{Aurélien \textsc{Garivier}}{Professor, École Normale Supérieure de Lyon, France}

    %\UGTCoEncadrant{Giorgio \textsc{Lucarelli}}
    %              {Maître de conférences,
    %              LCOMS, Université de Lorraine, Metz, France}

	% \UGTInvite{}{}  % guest
}

\Sethpageshift{0pt}  % adjust horizontal position
\Setvpageshift{100pt}  % adjust vertical position
\MakeUGthesePDG  % actually generate the title page

\end{otherlanguage}
\end{titlepage}
 %%%%%%%%%%%%%%%%%%%%%%%%%%%%%%%%%%%%%%%%%%%%%%%%%%%%%%%%%%%%%%%%%%  COMMENTED OUT FOR SEPARATE CHAPTER COMPILATIONS
\cleardoublepage %%%%%%%%%%%%%%%%%%%%%%%%%%%%%%%%%%%%%%%%%%%%%%%%%%%%%%%%%%%%%%%%%%%%%%%  COMMENTED OUT FOR SEPARATE CHAPTER COMPILATIONS

\pagenumbering{roman}  % roman page numbering (e.g., i ii iii), reset counter

%%!TEX encoding = UTF-8 Unicode

\pdfbookmark[0]{Dedication}{Dedication}
%
\null\vspace{\stretch{1}}
%
\begin{thesis_quotation}
%
\begin{flushright}

I dedicate this thesis to my grumpy cat.

\end{flushright}
%
\end{thesis_quotation}
%
\vspace{\stretch{2}}\null
 %%%%%%%%%%%%%%%%%%%%%%%%%%%%%%%%%%%%%%%%%%%%%%%%%%%%%%%%%%%%%%%%%  COMMENTED OUT FOR SEPARATE CHAPTER COMPILATIONS
%\cleardoublepage %%%%%%%%%%%%%%%%%%%%%%%%%%%%%%%%%%%%%%%%%%%%%%%%%%%%%%%%%%%%%%%%%%%%%%%  COMMENTED OUT FOR SEPARATE CHAPTER COMPILATIONS
%
%!TEX encoding = UTF-8 Unicode

\pdfbookmark[0]{Epigraph}{Epigraph}
%
\null\vspace{\stretch{1}}
%
\begin{otherlanguage}{french}
%
\cleanchapterquote%
{
    Elle est o\`{u} la poulette ?
}%
{
	Kadoc \textsc{De Vannes}
}%
{
}
%
\end{otherlanguage}
%
\vspace{\stretch{2}}\null
 %%%%%%%%%%%%%%%%%%%%%%%%%%%%%%%%%%%%%%%%%%%%%%%%%%%%%%%%%%%%%%%%%%%  COMMENTED OUT FOR SEPARATE CHAPTER COMPILATIONS
\cleardoublepage %%%%%%%%%%%%%%%%%%%%%%%%%%%%%%%%%%%%%%%%%%%%%%%%%%%%%%%%%%%%%%%%%%%%%%%  COMMENTED OUT FOR SEPARATE CHAPTER COMPILATIONS

\pagestyle{plain}  % display page number only

%!TEX encoding = UTF-8 Unicode

\addchap[Acknowledgments]{%
	\foreignlanguage{french}{Remerciements} {\Large (Acknowledgments)}
}

\vspace*{-5mm}
I would like to thank everyone, except from Dobby the free elf.

\begin{otherlanguage}{french}

Merci public !

\end{otherlanguage}
 %%%%%%%%%%%%%%%%%%%%%%%%%%%%%%%%%%%%%%%%%%%%%%%%%%%%%%%%%%%%  COMMENTED OUT FOR SEPARATE CHAPTER COMPILATIONS
\cleardoublepage %%%%%%%%%%%%%%%%%%%%%%%%%%%%%%%%%%%%%%%%%%%%%%%%%%%%%%%%%%%%%%%%%%%%%%%  COMMENTED OUT FOR SEPARATE CHAPTER COMPILATIONS

\addchap{Abstract / \foreignlanguage{french}{Résumé}}

\section*{Abstract}

A Markovian bandit is a sequential decision problem in which the decision maker has to activate a set of bandit's arms at each time, and the active arms evolve in a Markovian manner.
%A multi-armed bandit is a sequential decision problem where one has to activate a set of bandit's arms at each time.
%It is called ``Markovian bandit'' when its arms evolve in a Markovian manner.
%A multi-armed bandit problem is a sequential allocation problem of a finite set of resources over its arms.
%The activated arms evolve in an active Markovian manner.
%The arms that are not activated either remain frozen -- then one falls into the category of \emph{rested} Markovian bandits -- or evolve in a passive Markovian manner -- the setting is then called \emph{restless} Markovian bandit.
%Each arm generates a reward depending on its state and the active or passive evolution.
%The decision maker wants to maximize its cumulative reward over an infinite horizon of time.
%The arms that are not activated either remain frozen -- then one falls into the category of \emph{rested} Markovian bandits -- or evolve in a passive Markovian manner -- the setting is then called \emph{restless} Markovian bandit.
There are two types of Markovian bandits: (i) \emph{rested} bandits, where the arms that are not activated remain frozen, and (ii) \emph{restless} bandits, where the arms that are not activated evolve in a Markovian manner.
In general, Markovian bandits suffer from the curse of dimensionality that often makes the exact solution computationally intractable.
So, one has to resort to tractable heuristics such as index policies.
Two celebrated indices are the Gittins index for rested bandits and the Whittle index for restless bandits.

This thesis focuses on two questions (1) index computation when all model parameters are known and (2) learning algorithms when the parameters are unknown.

%For index computation, we first cover the ambiguities in the classical condition that guarantees the existence of the Whittle index in restless bandits.
For index computation, we point out the ambiguities in the classical indexability definition and propose a definition that assures the uniqueness of the Whittle index when this latter exists. %in restless bandit arms.
%We then introduce a new univocal definition of indexability that assures the uniqueness of the Whittle index when it exists.
%We then refine this condition to assure the uniqueness of the index when it exists.
We then develop an algorithm for testing the indexability and computing the Whittle indices of a restless arm.
%This algorithm can also compute the Gittins index.
%Our algorithm is built on three tools: (1) a careful characterization of Whittle index that allows one to recursively compute the $k$th smallest index from the $(k-1)$th smallest, and to test indexability, (2) the use of the Sherman-Morrison formula to make this recursive computation efficient, and (3) a sporadic use of the fastest matrix inversion and multiplication methods to obtain a subcubic complexity.
%We show that an efficient use of the Sherman-Morrison formula leads to an algorithm that computes Whittle index in $(2/3)S^3+o(S^3)$ arithmetic operations, where $S$ is the number of states of the arm.
%Its index computation is done in $(2/3)S^3+o(S^3)$ arithmetic operations, where $S$ is the number of states of the arm.
%The careful use of fast matrix multiplication leads to the first subcubic algorithm to compute Whittle or Gittins index: By using the current fastest matrix multiplication, the theoretical complexity of our algorithm is $\landauO(S^{2.5286})$.
The theoretical complexity of our algorithm is $\landauO(S^{2.5286})$, where $S$ is the number of arm's states.
%We also develop an efficient implementation of this algorithm in python programming language.
%It can compute indices of restless arms with several thousands of states in less than a few seconds.

%For learning setup, we divide our work into two phases: (1) design algorithms with learning performance guarantee in rested Markovian bandits and (2) study the challenges when learning in restless Markovian bandit.
%For rested bandits, Gittins index policy has been proven to be optimal when exactly one arm is activated at each decision and there is a discount on reward.
%For learning in rested bandits, we show that MB-PSRL and MB-UCBVI, modified versions of PSRL and UCBVI algorithms, can leverage Gittins index policy to have a regret guarantee and a runtime scalable in the number of arms. Furthermore, we show that MB-UCRL2, a modified version of UCRL2, also has a regret guarantee scalable in the number of arms. However, MB-UCRL2 has a runtime exponential in the number of arms.
For learning in rested bandits, we propose modifications of PSRL and UCBVI algorithms that we call MB-PSRL and MB-UCBVI.
We show that they can leverage Gittins index policy to have a regret guarantee and a runtime scalable in the number of arms.
Furthermore, we show that MB-UCRL2, a modification of UCRL2, also has a regret guarantee scalable in the number of arms.
However, MB-UCRL2 has a runtime exponential in the number of arms.
When learning in restless bandits, the regret guarantee depends heavily on the structure of the bandit.
We study how the structure of arms translates into the structure of the bandit.
%We show that no learning algorithms can perform uniformly well over the general class of restless bandits whose arms are unichain.
We exhibit a subclass of restless bandits that is not learnable.
We also show that it is difficult to construct a subclass of restless bandits with a desirable learning structure by only making assumptions about arms.
%in which each arm has an internal state that evolves in a Markovian manner depending on the decision maker's actions. 


\clearpage
% -----------------------------------------------------------------------------

\begin{otherlanguage}{french}

\section*{Résumé}

Les bandits markoviens sont une sous-classe de problèmes de bandits à bras multiples où l'on doit activer un sous-ensemble de bras à chaque instant de décision.
Les bras activés évoluent de manière markovienne et active.
Ceux qui ne sont pas activés (c'est-à-dire qui sont passifs) soit restent figés -- on entre alors dans la catégorie des bandits markoviens reposés -- soit évoluent de manière markovienne et passive – on parle alors de bandit markovien agité.
De tels problèmes souffrent de le fléau de la dimension qui rend souvent la solution exacte prohibitive en termes de calcul.
Il faut donc recourir à des heuristiques telles que la politique d'indice.
Deux indices célèbres sont l'indice de Gittins et l'indice de Whittle.
Cette thèse se concentre sur deux configurations : (1) le calcul d'indices lorsque tous les paramètres du modèle sont connus et (2) la conception d'algorithmes lorsque les paramètres sont inconnus.
Pour le calcul de l'indice, nous soulignons les ambiguïtés de la définition classique de l'indexabilité et proposons une définition raffinée qui assure l'unicité de l'indice de Whittle dans les bras de bandits agités.
Nous développons ensuite un algorithme testant l'indexabilité raffinée et calculant les indices de Whittle des bras agités.
La complexité théorique de notre algorithme est $\landauO(S^{2.5286})$, où $S$ est le nombre d'états du bras.
Pour l'apprentissage des bandits reposés, nous montrons que MB-PSRL et MB-UCBVI, des versions modifiées des algorithmes PSRL et UCBVI, peuvent tirer parti de la politique d'indice de Gittins pour avoir une garantie de regret et un temps d'exécution évolutif en nombre de bras.
De plus, nous montrons que MB-UCRL2, une version modifiée de UCRL2, possède également une garantie de regret évolutive en nombre de bras.
Cependant, MB-UCRL2 et toute modification des variantes d'UCRL2 pour le bandit reposé ont probablement un temps d'exécution exponentiel dans le nombre de bras. Lors de l'apprentissage de bandits agités, la garantie de regret dépend fortement de la structure du bandit. Ainsi, nous étudions comment la structure des bras se traduit dans la structure du bandit. Nous identifions une sous-classe de bandits agités qui n'est pas apprenable. Nous montrons également qu'il est difficile de définir une sous-classe de bandits agités avec une structure d'apprentissage souhaitable en ne faisant que des hypothèses sur les bras.


\end{otherlanguage}
 %%%%%%%%%%%%%%%%%%%%%%%%%%%%%%%%%%%%%%%%%%%%%%%%%%%%%%%%%%%%%  COMMENTED OUT FOR SEPARATE CHAPTER COMPILATIONS
\cleardoublepage %%%%%%%%%%%%%%%%%%%%%%%%%%%%%%%%%%%%%%%%%%%%%%%%%%%%%%%%%%%%%%%%%%%%%%%  COMMENTED OUT FOR SEPARATE CHAPTER COMPILATIONS

\setcounter{tocdepth}{2}  % define depth of ToC
\phantomsection\addcontentsline{toc}{chapter}{{\contentsname}}  % display ToC to ToC

% Add a bit of space in the ToC
\makeatletter
\pretocmd{\chapter}{\addtocontents{toc}{\protect\addvspace{15\p@}}}{}{}
\pretocmd{\section}{\addtocontents{toc}{\protect\addvspace{5\p@}}}{}{}
\makeatother

\tableofcontents %%%%%%%%%%%%%%%%%%%%%%%%%%%%%%%%%%%%%%%%%%%%%%%%%%%%%%%%%%%%%%%%%%%%%%%  COMMENTED OUT FOR SEPARATE CHAPTER COMPILATIONS
\cleardoublepage %%%%%%%%%%%%%%%%%%%%%%%%%%%%%%%%%%%%%%%%%%%%%%%%%%%%%%%%%%%%%%%%%%%%%%%  COMMENTED OUT FOR SEPARATE CHAPTER COMPILATIONS
\fi

% --- MAIN MATTER ----------------------

\pagenumbering{arabic}  % arabic page numbering (e.g., 1 2 3), reset counter
\pagestyle{scrheadings}  % display header and footer
\addtocontents{toc}{\bigskip}  % visual hint for numbering change in ToC


\KK{Do not add citation in page 3, nor the résumé in French, nor the acknowledgments when sending to reviewers.}

%%%%%%%%%%%%%%%%%%%%%%%%%%%%%%%%%%%%%%%%%%%%%%%%%%%%%%%%%%
%%%%%%%%%%%%%%%%%%%%%%%%%%%%%%%%%%%%%%%%%%%%%%%%%%%%%%%%%%
% main chapters go here
\bookmarksetup{startatroot}
\inputchapter{\includechapterintroduction}{chapter_introduction.tex}
%TODO: should we put number for introduction and conclusion chapters?

\iftotalcompilation
\part{Background on Markov decision process and reinforcement learning}
\KK{Motivate why this part should be read, what are essential to understand in this part}
\fi

\inputchapter{\includemdpchapter}{markovDecisionProcess_chapter.tex}

\inputchapter{\includereinforcementlearningchapter}{reinforcementLearning_chapter.tex}

\iftotalcompilation
\part{Indexability}
\label{part:idx}
\KK{In this part, we present two markovian bandit models and our contribution...}
\fi

\inputchapter{\includemarkovianbanditchapter}{markovianBandit_chapter.tex}

\inputchapter{\includeindexComputationchapter}{indexComputation_chapter.tex}

%need to add a conclusion

\iftotalcompilation
\part{Learning in Markovian bandits}
\label{part:learning}
\KK{In this part, we work on learning problem...}
\fi

%\bookmarksetup{startatroot}

\inputchapter{\includelearningAlgorithmsRestedMBchapter}{learningAlgorithmsRestedMB_chapter.tex}

\inputchapter{\includelearningAlgorithmsRestlessMBchapter}{learningAlgorithmsRestlessMB_chapter.tex}

\bookmarksetup{startatroot}  % https://stackoverflow.com/questions/1483396/how-to-explicitly-end-a-part-in-latex-with-hyperref
\addtocontents{toc}{\bigskip}
\inputchapter{\includechapterconclusion}{chapter_conclusion.tex}
%%%%%%%%%%%%%%%%%%%%%%%%%%%%%%%%%%%%%%%%%%%%%%%%%%%%%%%%%%
%%%%%%%%%%%%%%%%%%%%%%%%%%%%%%%%%%%%%%%%%%%%%%%%%%%%%%%%%%


% --- BACK MATTER ----------------------

\iftotalcompilation
\pdfbookmark[-1]{Back Matter}{Back Matter}

\pagenumbering{arabic}  % arabic page numbering (e.g., 1 2 3), reset counter
\renewcommand*{\thepage}{A\arabic{page}}  % prepend A to appendix page number
\pagestyle{scrheadings}  % display header and footer
\addtocontents{toc}{\bigskip}  % visual hint for numbering change in ToC
\fi

\appendix
\inputchapter{\includechapterappendix}{chapter_appendix.tex}
\inputchapter{\includechapterappendixtwo}{chapter_appendixtwo.tex}

\iftotalcompilation
\cleardoublepage %%%%%%%%%%%%%%%%%%%%%%%%%%%%%%%%%%%%%%%%%%%%%%%%%%%%%%%%%%%%%%%%%%%%%%%  COMMENTED OUT FOR SEPARATE CHAPTER COMPILATIONS

{
    \setstretch{1.1}
    \renewcommand{\bibfont}{\normalfont\small}
    \setlength{\biblabelsep}{0pt}
    \setlength{\bibitemsep}{0.5\baselineskip plus 0.5\baselineskip}
    \printbibliography
}

{\listoffigures \let\cleardoublepage\  \listoftables} %%%%%%%%%%%%%%%%%%%%%%%%%%%%%%%%%%  COMMENTED OUT FOR SEPARATE CHAPTER COMPILATIONS

\cleardoublepage %%%%%%%%%%%%%%%%%%%%%%%%%%%%%%%%%%%%%%%%%%%%%%%%%%%%%%%%%%%%%%%%%%%%%%%  COMMENTED OUT FOR SEPARATE CHAPTER COMPILATIONS



 %%%%%%%%%%%%%%%%%%%%%%%%%%%%%%%%%%%%%%%%%% List of publications %%%%%%%%%%%%%%%%%
% {
%     \clearpage
%     Additionally, the work conducted in this dissertation directly led to the
%     following communications.

%     \begin{refsection}
%     \makeatletter\@beginparpenalty=10000\makeatother  % prevent page break before list
%     \defbibenvironment{itembib}{\itemize}{\enditemize}{\item}
%     \nocite{*}
%     %
%     \paragraph{Peer-reviewed international conferences}
%     \printbibliography[heading=none,env=itembib,keyword={own},keyword={conference}]
%     %
%     \paragraph{Peer-reviewed international workshops}
%     \printbibliography[heading=none,env=itembib,keyword={own},keyword={workshop}]
%     \paragraph{National workshops}
%     \printbibliography[heading=none,env=itembib,keyword={own},keyword={nworkshop}]
%     \end{refsection}
% }
%%%%%%%%%%%%%%%%%%%%%%%%%%%%%%%%%%%%%%%%%% List of publications %%%%%%%%%%%%%%%%%


\cleardoublepage %%%%%%%%%%%%%%%%%%%%%%%%%%%%%%%%%%%%%%%%%%%%%%%%%%%%%%%%%%%%%%%%%%%%%%%  COMMENTED OUT FOR SEPARATE CHAPTER COMPILATIONS

% --- BACK COVER -----------------------

\pagenumbering{gobble}  % do not count any more
\pagestyle{empty}  % no header nor footer

\cleardoubleevenpage  % ensure even page for back cover %%%%%%%%%%%%%%%%%%%%%%%%%%%%%%%%%  COMMENTED OUT FOR SEPARATE CHAPTER COMPILATIONS
\areaset[0pt]{\paperwidth}{\paperheight}  % hack to reset margins
%
\newlength{\bcmargin}\setlength{\bcmargin}{1.4cm}  % back cover margin length
%
\centering
%
\null\vspace*{\dimexpr\bcmargin-\headsep\relax}
%
\begin{minipage}{\dimexpr\paperwidth-\bcmargin-\bcmargin\relax}
%
\pdfbookmark[0]{Back Cover}{Back Cover}
%
\noindent{\usekomafont{section}Abstract}\par
%
A Markovian bandit is a sequential decision problem in which the decision maker has to activate a set of bandit's arms at each time, and the active arms evolve in a Markovian manner.
%A multi-armed bandit is a sequential decision problem where one has to activate a set of bandit's arms at each time.
%It is called ``Markovian bandit'' when its arms evolve in a Markovian manner.
%A multi-armed bandit problem is a sequential allocation problem of a finite set of resources over its arms.
%The activated arms evolve in an active Markovian manner.
%The arms that are not activated either remain frozen -- then one falls into the category of \emph{rested} Markovian bandits -- or evolve in a passive Markovian manner -- the setting is then called \emph{restless} Markovian bandit.
%Each arm generates a reward depending on its state and the active or passive evolution.
%The decision maker wants to maximize its cumulative reward over an infinite horizon of time.
%The arms that are not activated either remain frozen -- then one falls into the category of \emph{rested} Markovian bandits -- or evolve in a passive Markovian manner -- the setting is then called \emph{restless} Markovian bandit.
There are two types of Markovian bandits: (i) \emph{rested} bandits, where the arms that are not activated remain frozen, and (ii) \emph{restless} bandits, where the arms that are not activated evolve in a Markovian manner.
In general, Markovian bandits suffer from the curse of dimensionality that often makes the exact solution computationally intractable.
So, one has to resort to tractable heuristics such as index policies.
Two celebrated indices are the Gittins index for rested bandits and the Whittle index for restless bandits.

This thesis focuses on two questions (1) index computation when all model parameters are known and (2) learning algorithms when the parameters are unknown.

%For index computation, we first cover the ambiguities in the classical condition that guarantees the existence of the Whittle index in restless bandits.
For index computation, we point out the ambiguities in the classical indexability definition and propose a definition that assures the uniqueness of the Whittle index when this latter exists. %in restless bandit arms.
%We then introduce a new univocal definition of indexability that assures the uniqueness of the Whittle index when it exists.
%We then refine this condition to assure the uniqueness of the index when it exists.
We then develop an algorithm for testing the indexability and computing the Whittle indices of a restless arm.
%This algorithm can also compute the Gittins index.
%Our algorithm is built on three tools: (1) a careful characterization of Whittle index that allows one to recursively compute the $k$th smallest index from the $(k-1)$th smallest, and to test indexability, (2) the use of the Sherman-Morrison formula to make this recursive computation efficient, and (3) a sporadic use of the fastest matrix inversion and multiplication methods to obtain a subcubic complexity.
%We show that an efficient use of the Sherman-Morrison formula leads to an algorithm that computes Whittle index in $(2/3)S^3+o(S^3)$ arithmetic operations, where $S$ is the number of states of the arm.
%Its index computation is done in $(2/3)S^3+o(S^3)$ arithmetic operations, where $S$ is the number of states of the arm.
%The careful use of fast matrix multiplication leads to the first subcubic algorithm to compute Whittle or Gittins index: By using the current fastest matrix multiplication, the theoretical complexity of our algorithm is $\landauO(S^{2.5286})$.
The theoretical complexity of our algorithm is $\landauO(S^{2.5286})$, where $S$ is the number of arm's states.
%We also develop an efficient implementation of this algorithm in python programming language.
%It can compute indices of restless arms with several thousands of states in less than a few seconds.

%For learning setup, we divide our work into two phases: (1) design algorithms with learning performance guarantee in rested Markovian bandits and (2) study the challenges when learning in restless Markovian bandit.
%For rested bandits, Gittins index policy has been proven to be optimal when exactly one arm is activated at each decision and there is a discount on reward.
%For learning in rested bandits, we show that MB-PSRL and MB-UCBVI, modified versions of PSRL and UCBVI algorithms, can leverage Gittins index policy to have a regret guarantee and a runtime scalable in the number of arms. Furthermore, we show that MB-UCRL2, a modified version of UCRL2, also has a regret guarantee scalable in the number of arms. However, MB-UCRL2 has a runtime exponential in the number of arms.
For learning in rested bandits, we propose modifications of PSRL and UCBVI algorithms that we call MB-PSRL and MB-UCBVI.
We show that they can leverage Gittins index policy to have a regret guarantee and a runtime scalable in the number of arms.
Furthermore, we show that MB-UCRL2, a modification of UCRL2, also has a regret guarantee scalable in the number of arms.
However, MB-UCRL2 has a runtime exponential in the number of arms.
When learning in restless bandits, the regret guarantee depends heavily on the structure of the bandit.
We study how the structure of arms translates into the structure of the bandit.
%We show that no learning algorithms can perform uniformly well over the general class of restless bandits whose arms are unichain.
We exhibit a subclass of restless bandits that is not learnable.
We also show that it is difficult to construct a subclass of restless bandits with a desirable learning structure by only making assumptions about arms.
%in which each arm has an internal state that evolves in a Markovian manner depending on the decision maker's actions. 

%
\vspace{3ex}\hrule\vspace{2ex}
%
\begin{otherlanguage}{french}
%
\noindent{\usekomafont{section}Résumé}\par
%
Les bandits markoviens sont une sous-classe de problèmes de bandits à bras multiples où l'on doit activer un sous-ensemble de bras à chaque instant de décision.
Les bras activés évoluent de manière markovienne et active.
Ceux qui ne sont pas activés (c'est-à-dire qui sont passifs) soit restent figés -- on entre alors dans la catégorie des bandits markoviens reposés -- soit évoluent de manière markovienne et passive – on parle alors de bandit markovien agité.
De tels problèmes souffrent de le fléau de la dimension qui rend souvent la solution exacte prohibitive en termes de calcul.
Il faut donc recourir à des heuristiques telles que la politique d'indice.
Deux indices célèbres sont l'indice de Gittins et l'indice de Whittle.
Cette thèse se concentre sur deux configurations : (1) le calcul d'indices lorsque tous les paramètres du modèle sont connus et (2) la conception d'algorithmes lorsque les paramètres sont inconnus.
Pour le calcul de l'indice, nous soulignons les ambiguïtés de la définition classique de l'indexabilité et proposons une définition raffinée qui assure l'unicité de l'indice de Whittle dans les bras de bandits agités.
Nous développons ensuite un algorithme testant l'indexabilité raffinée et calculant les indices de Whittle des bras agités.
La complexité théorique de notre algorithme est $\landauO(S^{2.5286})$, où $S$ est le nombre d'états du bras.
Pour l'apprentissage des bandits reposés, nous montrons que MB-PSRL et MB-UCBVI, des versions modifiées des algorithmes PSRL et UCBVI, peuvent tirer parti de la politique d'indice de Gittins pour avoir une garantie de regret et un temps d'exécution évolutif en nombre de bras.
De plus, nous montrons que MB-UCRL2, une version modifiée de UCRL2, possède également une garantie de regret évolutive en nombre de bras.
Cependant, MB-UCRL2 et toute modification des variantes d'UCRL2 pour le bandit reposé ont probablement un temps d'exécution exponentiel dans le nombre de bras. Lors de l'apprentissage de bandits agités, la garantie de regret dépend fortement de la structure du bandit. Ainsi, nous étudions comment la structure des bras se traduit dans la structure du bandit. Nous identifions une sous-classe de bandits agités qui n'est pas apprenable. Nous montrons également qu'il est difficile de définir une sous-classe de bandits agités avec une structure d'apprentissage souhaitable en ne faisant que des hypothèses sur les bras.

%
\end{otherlanguage}
%
\end{minipage}
 %%%%%%%%%%%%%%%%%%%%%%%%%%%%%%%%%%%%%%%%%%%%%%%%%%%%%%%%%%%%%%%%%%%%%%%  COMMENTED OUT FOR SEPARATE CHAPTER COMPILATIONS
\fi

% === END OF DOCUMENT =========================================================

\end{document}
