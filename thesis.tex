%!TEX encoding = UTF-8 Unicode

% The structure of this document is based on the my-thesis.tex example provided
% with the cleanthesis package (see http://cleanthesis.der-ric.de/).
% It also is inspired by the PhD manuscripts of Raphaël Bleuse, David Beniamine
% (see https://github.com/dbeniamine/Ph.D_thesis) and David Glesser.
% Uneeded packages and comments have been stripped out.

\pdfobjcompresslevel 0  % Produce PDF file respecting CINES requirements

\documentclass[%
    paper=A4,              % paper size
    twoside=true,          % onesite or twoside printing
    openright,             % doublepage cleaning ends up right side
    parskip=half,          % spacing value / method for paragraphs
    chapterprefix=true,    % prefix for chapter marks
    11pt,                  % font size
    headings=normal,       % size of headings
    bibliography=totoc,    % include bib in toc
    listof=totoc,          % include listof entries in toc
    titlepage=on,          % own page for each title page
    captions=tableabove,   % display table captions above the float env
    chapterprefix=false,   % do not display a prefix for chapters
    appendixprefix=false,  % do not display a prefix for appendix chapters
    draft=false,           % value for draft version
]{scrreprt}


% === PACKAGES IMPORT / CONFIGURATION =========================================
% --- ENCODING / FONTS -----------------

\usepackage[utf8]{inputenc}
\usepackage[T1]{fontenc}
\usepackage{textcomp}
\usepackage{lipsum}

% --- MATHEMATICS TYPESETTING ----------

\usepackage{amsmath}
\usepackage{amssymb}
\usepackage{amsthm}
\usepackage{bm}

% --- SPECIFIC ABBREVIATIONS MACROS -------------

% *** SPECIFIC MACROS ***
% specific macro definition for the paper
%
% requires amsthm, xspace

% theorem-like environment
\newtheorem{thm}{Theorem}
\newtheorem{lem}[thm]{Lemma}
\newtheorem{prop}{Proposition}
\newtheorem{propty}{Property}
\newtheorem{defn}{Definition}
\newtheorem{cor}[thm]{Corollary}
\newtheorem{conj}[thm]{Conjecture}
\newtheorem{asmp}[thm]{Assumption}
\newtheorem{prob}{Problem}

% asymptotic complexity (Landau) notations
\newcommand{\landauO}{\ensuremath{\mathcal{O}}\xspace}
\newcommand{\landauOmega}{\ensuremath{\Omega}\xspace}
\newcommand{\landauTheta}{\ensuremath{\Theta}\xspace}
\newcommand{\landauo}{\ensuremath{o}\xspace}
\newcommand{\landauomega}{\ensuremath{\omega}\xspace}
\newcommand{\landauorder}{\ensuremath{\sim}\xspace}

% scheduling notations
\newcommand{\graham}[3]{\mbox{\ensuremath{#1\mid#2\mid#3}}}
\newcommand{\Cmax}{\ensuremath{C_{\max}}\xspace}

% complexity classes
\newcommand{\cNP}{\textbf{NP}\xspace}  % bold notations
\newcommand{\cP}{\textbf{P}\xspace}
% \newcommand{\cNP}{\ensuremath{\mathcal{N\!P}}\xspace}  % round notations
% \newcommand{\cP}{\ensuremath{\mathcal{P}}\xspace}


\newlength\mylen % algorithm2e hack
\newcommand\myinput[1]{%
  \settowidth\mylen{\KwIn{}}%
  \setlength\hangindent{\mylen}%
  \hspace*{\mylen}#1\\}

\newcommand\myoutput[1]{% algorithm2e hack
  \settowidth\mylen{\KwOut{}}%
  \setlength\hangindent{\mylen}%
  \hspace*{\mylen}#1\\}


% Float environments
\renewcommand{\topfraction}{0.8}
\renewcommand{\bottomfraction}{0.8}


%%%%%%%%%%%%%%%%%%%%%%%
% Modular compilation %
%%%%%%%%%%%%%%%%%%%%%%%
\newif\ifwatermark
\watermarkfalse

\newif\iftotalcompilation
\totalcompilationtrue

\newcommand{\inputchapter}[2]{%
  \ifdef{#1}%
    {\input{#2}\cleardoublepage}%
    {}%
}

% Modeling notations
\newcommand{\mcdots}{\ensuremath{\!\cdot\!\cdot\!\cdot\!}\xspace}
\newcommand{\unif}[2]{\ensuremath{\mathcal{U}\left({#1},{#2}\right)}\xspace}

% Abreviations
\newcommand{\eg}{e.g.\@\xspace}
\newcommand{\ie}{i.e.\@\xspace}
\newcommand{\aka}{a.k.a.\@\xspace}
\newcommand{\resp}{resp.\@\xspace}
\newcommand{\etal}{et~al.\@\xspace}

% Referencing several times a footnote
% See https://tex.stackexchange.com/a/54240
\newcommand{\savefootnote}[2]{\footnote{\label{#1}#2}}
\newcommand{\repeatfootnote}[1]{\textsuperscript{\ref{#1}}}


% MDP Notation
\newcommand{\StateSpace}{\mathcal{S}}       % MC state space -->
\newcommand{\sSpace}{\mathcal{S}}       % MC state space -->
\newcommand{\nStateSpace}{S}                % state space size
\newcommand{\sSize}{S}                % state space size
\newcommand{\mdpTran}{p}                       % mdp transition kernel -->
\newcommand{\mdpRew}{r}                        % mdp reward function -->
\newcommand{\tran}{P}                       % MC transition kernel -->
\newcommand{\rew}{\mathbf{r}}                        % MC reward function -->
\newcommand{\pol}{\pi}                      % policy -->
\newcommand{\vfunc}{v}                      % value function -->
\newcommand{\qfunc}{q}                     % action-value function -->
\newcommand{\SC}{SC}                        % sample complexity -->
\newcommand{\Algo}{\mathfrak{A}}            % algorithm used -->
\newcommand{\Reg}{\mathrm{Regret}}              % regret -->
\newcommand{\BayReg}{\mathrm{BayesRegret}}              % bayesian regret -->
\newcommand{\nbStates}{k}                   % number of states in each Markovian bandit -->
\newcommand{\nbBandits}{n}                  % number of Markovian bandits -->
\newcommand{\state}{x}                      % Markovian bandit state -->
\newcommand{\statePrime}{y}                % Markovian bandit next state -->
\newcommand{\mdpModel}{\mathcal{M}}         % MDP Model
\newcommand{\mdpStateSpace}{\mathcal{E}}    % MDP state space

\newcommand{\mSpace}{\mathcal{X}}    % MDP state space
\newcommand{\mSize}{E}    % MDP state space
\newcommand{\ActionSpace}{\mathcal{A}}      % MDP action space -->
\newcommand{\aSpace}{\mathcal{A}}      % MDP action space -->

\newcommand{\Action}{a}                     % MDP action -->
\newcommand{\ActionPrime}{b}                % MDP next action -->
\newcommand{\mdpState}{\mathbf{x}}                   % MDP state -->
\newcommand{\mdpStatePrime}{\mathbf{y}}              % MDP next state -->
\newcommand{\discount}{\beta}               % MDP discount factor -->
\newcommand{\bandit}{\mathcal{B}}           % bandit symbol -->
\newcommand{\GIndex}{\gamma}                % Gittins Index symbol -->
\newcommand{\horizon}{H}                    % episode's length in FiniteHorizon setting -->
\newcommand{\nbEpisodes}{K}                 % the total number of episodes
% Estimate symbols
\newcommand{\estTran}{\hat{\tran}}          % estimated transition
\newcommand{\estRew}{\hat{\rew}}            % estimated reward
\newcommand{\estMdp}{\hat{\mdpModel}}       % estimated MDP Model
\newcommand{\estPol}{\hat{\pol}}       % estimated MDP Model
% Random Variables
\newcommand{\RVmdpState}{X}                 % MDP state random variable
\newcommand{\RVmdpStatePrime}{Y}                 % MDP state random variable
\newcommand{\RVaction}{A}                 % MDP state random variable
\newcommand{\RVstate}{X}                    % Markovian bandit state random variable
\newcommand{\RVrew}{R}                      % reward random variable
\newcommand{\RVvfunc}{V}                    % reward random variable
\newcommand{\RVqfunc}{Q}                    % reward random variable

\newcommand\sto{^{\mathrm{stoc.pb}}} % notation for stochastic bandit problem
\newcommand\stoi{^{\mathrm{stoc.pb},j}} % notation for stochastic bandit problem
\newcommand\p[1]{\left(#1\right)} % parenthesis
\newcommand\ceil[1]{\left\lceil#1\right\rceil}
\newcommand\floor[1]{\left\lfloor#1\right\rfloor}
\newcommand\norm[1]{\left\|#1\right\|}      % norm
\newcommand\abs[1]{\left\vert#1\right\vert}      % norm
\newcommand\ind[1]{\mathbb{I}_{\{#1\}}} % indicator function

\newcommand\Eqref[1]{Equation~\eqref{#1}}

% Matrix
\def\mA{{\bm{A}}}
\def\mB{{\bm{B}}}
\def\mC{{\bm{C}}}
\def\mD{{\bm{D}}}
\def\mE{{\bm{E}}}
\def\mF{{\bm{F}}}
\def\mG{{\bm{G}}}
\def\mH{{\bm{H}}}
\def\mI{{\bm{I}}}
\def\mJ{{\bm{J}}}
\def\mK{{\bm{K}}}
\def\mL{{\bm{L}}}
\def\mM{{\bm{M}}}
\def\mN{{\bm{N}}}
\def\mO{{\bm{O}}}
\def\mP{{\bm{P}}}
\def\mQ{{\bm{Q}}}
\def\mR{{\bm{R}}}
\def\mS{{\bm{S}}}
\def\mT{{\bm{T}}}
\def\mU{{\bm{U}}}
\def\mV{{\bm{V}}}
\def\mW{{\bm{W}}}
\def\mX{{\bm{X}}}
\def\mY{{\bm{Y}}}
\def\mZ{{\bm{Z}}}
\def\mBeta{{\bm{\beta}}}
\def\mPhi{{\bm{\Phi}}}
\def\mDelta{{\bm{\Delta}}}
\def\mLambda{{\bm{\Lambda}}}
\def\mSigma{{\bm{\Sigma}}}

% Graph
\def\gA{{\mathcal{A}}}
\def\gB{{\mathcal{B}}}
\def\gC{{\mathcal{C}}}
\def\gD{{\mathcal{D}}}
\def\gE{{\mathcal{E}}}
\def\gF{{\mathcal{F}}}
\def\gG{{\mathcal{G}}}
\def\gH{{\mathcal{H}}}
\def\gI{{\mathcal{I}}}
\def\gJ{{\mathcal{J}}}
\def\gK{{\mathcal{K}}}
\def\gL{{\mathcal{L}}}
\def\gM{{\mathcal{M}}}
\def\gN{{\mathcal{N}}}
\def\gO{{\mathcal{O}}}
\def\gP{{\mathcal{P}}}
\def\gQ{{\mathcal{Q}}}
\def\gR{{\mathcal{R}}}
\def\gS{{\mathcal{S}}}
\def\gT{{\mathcal{T}}}
\def\gU{{\mathcal{U}}}
\def\gV{{\mathcal{V}}}
\def\gW{{\mathcal{W}}}
\def\gX{{\mathcal{X}}}
\def\gY{{\mathcal{Y}}}
\def\gZ{{\mathcal{Z}}}


% Vectors
\def\vzero{{\bm{0}}}
\def\vone{{\bm{1}}}
\def\vmu{{\bm{\mu}}}
\def\vtheta{{\bm{\theta}}}
\def\vbeta{{\bm{\beta}}}
\def\valpha{{\bm{\alpha}}}
\def\vdelta{{\bm{\delta}}}
\def\vpi{{\bm{\pi}}}
\def\va{{\bm{a}}}
\def\vb{{\bm{b}}}
\def\vc{{\bm{c}}}
\def\vd{{\bm{d}}}
\def\ve{{\bm{e}}}
\def\vf{{\bm{f}}}
\def\vg{{\bm{g}}}
\def\vh{{\bm{h}}}
\def\vi{{\bm{i}}}
\def\vj{{\bm{j}}}
\def\vk{{\bm{k}}}
\def\vl{{\bm{l}}}
\def\vm{{\bm{m}}}
\def\vn{{\bm{n}}}
\def\vo{{\bm{o}}}
\def\vp{{\bm{p}}}
\def\vq{{\bm{q}}}
\def\vr{{\bm{r}}}
\def\vs{{\bm{s}}}
\def\vt{{\bm{t}}}
\def\vu{{\bm{u}}}
\def\vv{{\bm{v}}}
\def\vw{{\bm{w}}}
\def\vx{{\bm{x}}}
\def\vy{{\bm{y}}}
\def\vz{{\bm{z}}}

% Sets
\def\sA{{\mathbb{A}}}
\def\sB{{\mathbb{B}}}
\def\sC{{\mathbb{C}}}
\def\sD{{\mathbb{D}}}
% Don't use a set called E, because this would be the same as our symbol
% for expectation.
\def\sF{{\mathbb{F}}}
\def\sG{{\mathbb{G}}}
\def\sH{{\mathbb{H}}}
\def\sI{{\mathbb{I}}}
\def\sJ{{\mathbb{J}}}
\def\sK{{\mathbb{K}}}
\def\sL{{\mathbb{L}}}
\def\sM{{\mathbb{M}}}
\def\sN{{\mathbb{N}}}
\def\sO{{\mathbb{O}}}
\def\sP{{\mathbb{P}}}
\def\sQ{{\mathbb{Q}}}
\def\sR{{\mathbb{R}}}
\def\sS{{\mathbb{S}}}
\def\sT{{\mathbb{T}}}
\def\sU{{\mathbb{U}}}
\def\sV{{\mathbb{V}}}
\def\sW{{\mathbb{W}}}
\def\sX{{\mathbb{X}}}
\def\sY{{\mathbb{Y}}}
\def\sZ{{\mathbb{Z}}}
\def\hatB{\hat{\gB}}


\newcommand{\widx}{\lambda}
\newcommand{\mcal}[1]{\mathcal{#1}}
\newcommand{\Proba}[1]{\sP\left(#1\right)}
\newcommand{\ex}[1]{\mathbb{E}\left[#1\right]}
\newcommand{\var}[1]{\mathbb{V}\left[#1\right]}
\newcommand{\pik}{\pi_k}
\newcommand{\Mk}{M_k}
\newcommand{\Mbar}{\bar{M}}
\newcommand{\ep}{\varepsilon}
\newcommand{\rhat}{\hat{r}}
\newcommand{\Phat}{\hat{P}}
\newcommand{\Ahat}{\hat{A}}
\newcommand{\Bhat}{\hat{B}}
\newcommand{\Qhat}{\hat{Q}}
\newcommand{\rbar}{\bar{r}}
\newcommand{\Pbar}{\bar{P}}
\newcommand{\I}{\mathbb{I}}
\newcommand{\R}{\mathbb{R}}
\newcommand{\N}{\mathbb{N}}
\newcommand{\bA}{\boldsymbol{A}}
\newcommand{\bX}{\boldsymbol{X}}
\newcommand{\bY}{\boldsymbol{Y}}
\newcommand{\voisin}{\mathcal{V}}
\newcommand{\brhat}{\hat{\br}}
\newcommand{\event}{\xi}
\newcommand{\mTilde}{\tilde{M}}
\newcommand{\aTilde}{\tilde{a}}
\newcommand{\piTilde}{\tilde{\pi}}
\DeclareMathOperator*{\argmax}{arg\,max}
\DeclareMathOperator*{\argmin}{arg\,min}

% For a custom label (in underbrace)
\makeatletter
\newcommand{\customlabel}[2]{%
   \protected@write \@auxout {}{\string \newlabel {#1}{{#2}{\thepage}{#2}{#1}{}} }%
   \hypertarget{#1}{#2}
}
\makeatother

% Math notations




\input{macros.include.tex}

\ifwatermark
% https://texblog.org/2012/02/17/watermarks-draft-review-approved-confidential/
\usepackage{draftwatermark}
\usepackage{datetime}
 \SetWatermarkColor[rgb]{0.8,0.9,1}
 \SetWatermarkText{%
    \begin{minipage}{6in}
    \begin{center}
      \settimeformat{hhmmsstime}
      {\fontsize{24}{12}\selectfont DRAFT\\\today, \currenttime}
    \end{center}
    \end{minipage}
  }
\SetWatermarkScale{1}

\fi

% \usepackage{kpfonts}

% --- i18n, l10n -----------------------

\usepackage[french,english]{babel}  % load default language last
\usepackage[french,english]{isodate}  % load default language last

% --- PAGE SETTING ---------------------

\usepackage[%
    figuresep=colon,          % label separator for captions
    hangfigurecaption=false,  % use hanging figure label (within margin)
    hangsection=true,         % use hanging section label (within margin)
    hangsubsection=true,      % use hanging subsection label (within margin)
    colorize=full,            % define color mode
    colortheme=bluemagenta,   % what colors to use
    bibsys=biber,             % citation manamgement engine
    bibfile=references,       % bibtex file
    bibstyle=alphabetic,      % reference style
]{cleanthesis}

% In some 'online' bibtex entries, we do have a DOI, thanks to Zenodo. Here we want to display it.
\AtEveryBibitem{%
    \ifentrytype{online}{
        \csappto{blx@bbx@\thefield{entrytype}}{% put at end of entry
            \iffieldundef{doi}{}{%
                \printfield{doi}
            }
        }
    }
}

% --- GRAPHICS / FIGURES ---------------

\usepackage{epsfig}  % XXX: to be removed when CCPE figures are out

\newcommand{\SetFigFont}[3]{\fontsize{#1}{#2pt}\normalfont\sffamily}  % XXX: to be removed when CCPE figures are out

\usepackage{tikz}
\usepackage{pgfplots}
\pgfplotsset{compat=1.13}
\usetikzlibrary{arrows,shapes,positioning,shadows,trees,calc,decorations.text}


%\usepackage{titlesec}
%titlesec is outdated and create Warning Missing number blablabla
%https://tex.stackexchange.com/questions/302111/multiple-missing-number-treated-as-zero-and-illegal-unit-of-measure-pt-insert
\makeatletter
\renewcommand{\sectionlinesformat}[4]{%
  \Ifstr{#1}{section}{\clearpage}{}%
  \@hangfrom{\hskip #2#3}{#4}%
}
\makeatother

% --- MODULAR CHAPTER COMPILATION -----

\usepackage{etoolbox}

% --- MISC. PACKAGES -------------------

\usepackage[a4paper]{uga}  % UGA title page style (see titlepage.tex)
\usepackage{xspace}  % easy macros definition
\usepackage{booktabs}
%\usepackage[font=footnotesize]{subfig}
\usepackage{subcaption}
\usepackage[linesnumbered,ruled,noend,vlined]{algorithm2e}
\SetKwProg{Init}{Init.}{}{}
\newenvironment{subroutine}[1][htb]
{\renewcommand{\algorithmcfname}{Subroutine}% Update algorithm name
   \begin{algorithm}[#1]%
}{\end{algorithm}}


\usepackage{csquotes}
\usepackage{tabularx}
\usepackage{pifont}
\newcommand{\cmark}{\ding{51}}%
\newcommand{\xmark}{\ding{55}}%
\usepackage{pdflscape}

\usepackage[binary-units,group-digits,group-separator={,}]{siunitx}
% Uncomment to have stuff like 6.7e-11, keep it commented for 6.7\times10^{-11}
%\sisetup{output-exponent-marker=\ensuremath{\mathrm{e}}}
\DeclareSIUnit\flop{Flop}
\DeclareSIUnit\flops{\flop\per\second}
\newcommand{\Num}[1]{\num[group-separator={,}]{#1}\xspace}
\newcommand{\NSI}[2]{\SI[group-separator={,}]{#1}{#2}\xspace}
\DeclareSIUnit\core{core}

\usepackage{color,colortbl}
\definecolor{gray98}{rgb}{0.98,0.98,0.98}
\definecolor{gray95}{rgb}{0.95,0.95,0.95}
\definecolor{gray20}{rgb}{0.20,0.20,0.20}
\definecolor{gray25}{rgb}{0.25,0.25,0.25}
\definecolor{gray16}{rgb}{0.161,0.161,0.161}
\definecolor{gray80}{rgb}{0.8,0.8,0.8}
\definecolor{gray60}{rgb}{0.6,0.6,0.6}
\definecolor{gray30}{rgb}{0.3,0.3,0.3}
\definecolor{bgray}{RGB}{248, 248, 248}
\definecolor{amgreen}{RGB}{77, 175, 74}
\definecolor{amblu}{RGB}{55, 126, 184}
\definecolor{amred}{RGB}{228,26,28}
\definecolor{amdove}{RGB}{102,102,122}
\usepackage{xcolor}
\definecolor{myblue}{cmyk}{1, .50, .10, .01}  % see definition of "bluemagenta" in cleanthesis.sty
\usepackage[procnames]{listings}
\lstset{ %
    backgroundcolor=\color{gray98},   % choose the background color; you must add \usepackage{color} or \usepackage{xcolor}
    basicstyle=\ttfamily\scriptsize,  % the size of the fonts that are used for the code
    breakatwhitespace=false,          % sets if automatic breaks should only happen at whitespace
    breaklines=true,                  % sets automatic line breaking
    showlines=true,                   % sets automatic line breaking
    captionpos=b,                     % sets the caption-position to bottom
    commentstyle=\color{gray60},      % comment style
    extendedchars=true,               % lets you use non-ASCII characters; for 8-bits encodings only, does not work with UTF-8
    frame=single,                     % adds a frame around the code
    keepspaces=true,                  % keeps spaces in text, useful for keeping indentation of code (possibly needs columns=flexible)
    columns=flexible,
    keywordstyle=\color{amblu},       % keyword style
    procnamestyle=\color{colorfuncall},      % procedures style
    language=[95]fortran,             % the language of the code
    numbers=left,                     % where to put the line-numbers; possible values are (none, left, right)
    numbersep=5pt,                    % how far the line-numbers are from the code
    numberstyle=\tiny\color{gray20},  % the style that is used for the line-numbers
    rulecolor=\color{gray20},         % if not set, the frame-color may be changed on line-breaks within not-black text (\eg comments (green here))
    showspaces=false,                 % show spaces everywhere adding particular underscores; it overrides 'showstringspaces'
    showstringspaces=false,           % underline spaces within strings only
    showtabs=false,                   % show tabs within strings adding particular underscores
    stepnumber=2,                     % the step between two line-numbers. If it's 1, each line will be numbered
    stringstyle=\color{amdove},       % string literal style
    tabsize=2,                        % sets default tabsize to 2 spaces
    % title=\lstname,                 % show the filename of files included with \lstinputlisting; also try caption instead of title
    procnamekeys={call}
}
\definecolor{colorfuncall}{rgb}{0.6,0,0}

% --- PDF SUPPORT ----------------------

\usepackage{bookmark,hyperref}  % import as last package

%% Citing software
\ExecuteBibliographyOptions{
    halid=true,
    swhid=true,
    swlabels=true,
    vcs=true,
    license=true
}

\hypersetup{
    unicode=true,
    plainpages=false,
    colorlinks=false,
    pdfborder={0 0 0},
    breaklinks=true,  % allow line breaks inside links
    bookmarksnumbered=true,
    bookmarksopen=true,
}

% --- Custom TODO line ------------------------
%\usepackage{mdframed}
%\newmdenv[%
%  linecolor=red,%
%  backgroundcolor=red!20,%
%  linewidth=3pt,%
%  hidealllines=true,%
%  leftline=true,%
%]{todoenv}
%\newcommand{\todo}[1]{\begin{todoenv}#1\end{todoenv}}
%\newcommand{\todoC}[1]{\todo{#1}}
\usepackage[textwidth=\linewidth]{todonotes}		% for todo's
\renewcommand{\NG}[1]{\todo[color=blue!10,author=\textbf{\small NG},inline]{\small #1\\}}
\newcommand{\BG}[1]{\todo[color=red!10,author=\textbf{\small BG},inline]{\small #1\\}}
\newcommand{\KK}[1]{\todo[color=green!10,author=\textbf{\small KK},inline]{\small #1\\}}
\graphicspath{{}{img/}}

% -- command for box colored --
\usepackage[most]{tcolorbox}
\tcolorboxenvironment{thm}{
    enhanced,
    breakable,
    fonttitle=\bfseries,
    colback=RoyalBlue!5,
    colframe=RoyalBlue!40!black,
    coltitle=RoyalBlue!40!black
}
\tcolorboxenvironment{prop}{
    enhanced,
    boxrule=0pt,frame hidden,
    borderline west={4pt}{0pt}{amblu},
    colback=gray95,
    coltitle=amblu,
    colframe=amblu,
    sharp corners
}
\tcolorboxenvironment{defn}{
    sharp corners,
    colback=gray95,
    colframe=myblue, 
    rightrule=0pt, bottomrule=0pt,
    enhanced,
    breakable,
    boxed title style={colframe=gray95, sharp corners},
    attach boxed title to top left,
}
\tcolorboxenvironment{lem}{
    sharp corners,
    colback=gray95,
    colframe=amgreen, 
    rightrule=0pt, toprule=0pt,
    enhanced,
    breakable,
    boxed title style={colframe=gray95, sharp corners},
    attach boxed title to top left,
}



% === MACROS DEFINITION =======================================================

% --- COMMON ABBREVIATIONS MACROS -------------

% latin abbreviations, see:
%   - http://www.sussex.ac.uk/informatics/punctuation/capsandabbr/abbr
% comment by Sascha Hunold, see also:
%   - https://www.ieee.org/documents/style_manual.pdf
%     (p. 32, Short Reference List of Italics)
%   - http://web.ece.ucdavis.edu/~jowens/commonerrors.html

% === META DATA ===============================================================
\title{Algorithms for Markovian Bandits: Indexability and Learning}
\author{Kimang KHUN}

% fill pdf meta data
\makeatletter
\hypersetup{pdftitle=\@author's thesis}
\hypersetup{pdfsubject=\@title}
\hypersetup{pdfauthor=\@author}
\makeatother

\usepackage{enumitem}
\setlistdepth{9}
%\usepackage{minitoc}

% === DOCUMENT CONTENT ========================================================

%% Comment the following to have chapters numbered without interruption (numbering through parts)
%\makeatletter\@addtoreset{chapter}{part}\makeatother%

\begin{document}

% --- FRONT MATTER ---------------------

\iftotalcompilation
\pdfbookmark[-1]{Front Matter}{Front Matter}

\pagenumbering{gobble}  % do not count title page
\pagestyle{empty}  % no header nor footer

%!TEX encoding = UTF-8 Unicode

\begin{titlepage}
%
\pdfbookmark[0]{Cover}{Cover}
%
\begin{otherlanguage}{french}

% UGA meta data ------------------------

% \Universite{}
% \Grade{}
\Specialite{Mathématique et Informatique}
\Arrete{\printdate{2016-05-25}}
\Directeur{Bruno \textsc{Gaujal}, Inria}
\CoDirecteur{Nicolas \textsc{Gast}, Inria}
\Laboratoire{Laboratoire d'Informatique de Grenoble}
\EcoleDoctorale{MSTII}
\Depot{\printdate{2023-03-30}} % XXXXXXXXXXXXXXXXXXXXXXXXXXXXXXXXXXXXXX

\makeatletter
\Auteur{{\@author}}
\Titre{Des algorithmes pour les bandits markoviens: indexabilité et apprentissage}
\Soustitre{\foreignlanguage{english}{\@title}}
\makeatother

% UGA meta data: jury ------------------

\Jury{
    \UGTRapporteur{Konstantin \textsc{Avrachenkov}}{Researcher, Inria, France}

    \UGTRapporteur{Aditya \textsc{Mahajan}}{Professor associate, McGill University, Canada}
    
    \UGTExaminatrice{Ana \textsc{Busic}}{Researcher, Inria, France}

    \UGTDirecteur{Bruno \textsc{Gaujal}}{Researcher, Inria, France}

    \UGTExaminateur{Franck \textsc{Iutzeler}}{Maître de conférences, Université Grenoble-Alpes, France}

    \UGTPresident{Aurélien \textsc{Garivier}}{Professor, École Normale Supérieure de Lyon, France}

    %\UGTCoEncadrant{Giorgio \textsc{Lucarelli}}
    %              {Maître de conférences,
    %              LCOMS, Université de Lorraine, Metz, France}

	% \UGTInvite{}{}  % guest
}

\Sethpageshift{0pt}  % adjust horizontal position
\Setvpageshift{100pt}  % adjust vertical position
\MakeUGthesePDG  % actually generate the title page

\end{otherlanguage}
\end{titlepage}
 %%%%%%%%%%%%%%%%%%%%%%%%%%%%%%%%%%%%%%%%%%%%%%%%%%%%%%%%%%%%%%%%%%  COMMENTED OUT FOR SEPARATE CHAPTER COMPILATIONS
\cleardoublepage %%%%%%%%%%%%%%%%%%%%%%%%%%%%%%%%%%%%%%%%%%%%%%%%%%%%%%%%%%%%%%%%%%%%%%%  COMMENTED OUT FOR SEPARATE CHAPTER COMPILATIONS

\pagenumbering{roman}  % roman page numbering (e.g., i ii iii), reset counter

%%!TEX encoding = UTF-8 Unicode

\pdfbookmark[0]{Dedication}{Dedication}
%
\null\vspace{\stretch{1}}
%
\begin{thesis_quotation}
%
\begin{flushright}

I dedicate this thesis to my grumpy cat.

\end{flushright}
%
\end{thesis_quotation}
%
\vspace{\stretch{2}}\null
 %%%%%%%%%%%%%%%%%%%%%%%%%%%%%%%%%%%%%%%%%%%%%%%%%%%%%%%%%%%%%%%%%  COMMENTED OUT FOR SEPARATE CHAPTER COMPILATIONS
%\cleardoublepage %%%%%%%%%%%%%%%%%%%%%%%%%%%%%%%%%%%%%%%%%%%%%%%%%%%%%%%%%%%%%%%%%%%%%%%  COMMENTED OUT FOR SEPARATE CHAPTER COMPILATIONS
%
%!TEX encoding = UTF-8 Unicode

\pdfbookmark[0]{Epigraph}{Epigraph}
%
\null\vspace{\stretch{1}}
%
\begin{otherlanguage}{french}
%
\cleanchapterquote%
{
    Elle est o\`{u} la poulette ?
}%
{
	Kadoc \textsc{De Vannes}
}%
{
}
%
\end{otherlanguage}
%
\vspace{\stretch{2}}\null
 %%%%%%%%%%%%%%%%%%%%%%%%%%%%%%%%%%%%%%%%%%%%%%%%%%%%%%%%%%%%%%%%%%%  COMMENTED OUT FOR SEPARATE CHAPTER COMPILATIONS
\cleardoublepage %%%%%%%%%%%%%%%%%%%%%%%%%%%%%%%%%%%%%%%%%%%%%%%%%%%%%%%%%%%%%%%%%%%%%%%  COMMENTED OUT FOR SEPARATE CHAPTER COMPILATIONS

\pagestyle{plain}  % display page number only

%!TEX encoding = UTF-8 Unicode

\addchap[Acknowledgments]{%
	\foreignlanguage{french}{Remerciements} {\Large (Acknowledgments)}
}

\vspace*{-5mm}
I would like to thank everyone, except from Dobby the free elf.

\begin{otherlanguage}{french}

Merci public !

\end{otherlanguage}
 %%%%%%%%%%%%%%%%%%%%%%%%%%%%%%%%%%%%%%%%%%%%%%%%%%%%%%%%%%%%  COMMENTED OUT FOR SEPARATE CHAPTER COMPILATIONS
\cleardoublepage %%%%%%%%%%%%%%%%%%%%%%%%%%%%%%%%%%%%%%%%%%%%%%%%%%%%%%%%%%%%%%%%%%%%%%%  COMMENTED OUT FOR SEPARATE CHAPTER COMPILATIONS

\addchap{Abstract / \foreignlanguage{french}{Résumé}}

\section*{Abstract}

Markovian bandits are a subclass of multi-armed bandit problems where a decision maker has to activate a set of arms at each decision time.
The activated arms evolve in an active Markovian manner.
Those that are not activated (\ie, are passive) either remain frozen -- then one falls into the category of rested Markovian bandits -- or evolve in a passive Markovian manner -- the setting is then called restless Markovian bandit.
Each arm generates a reward depending on its state and the active or passive evolution.
The decision maker wants to maximize its cumulative reward over an infinite horizon of time.
Such problems suffer from the curse of dimensionality that often makes the exact solution computationally prohibitive.
So, one has to resort to heuristics such as index policy.
Two celebrated index definitions are Gittins index for rested bandits and Whittle index for restless bandits.

In this thesis, we focus on two setups: (1) index computation when all model parameters are known and (2) learning algorithm design when the parameters are unknown.

For index computation, we first cover the ambiguities in the classical condition known as the indexability that guarantees the existence of the Whittle index in restless bandits.
We then introduce a new univocal definition of indexability that assures the uniqueness of the Whittle index when it exists.
We then develop an algorithm to test such the indexability and compute the Whittle indices of any finite-state restless bandit arm.
This algorithm can also compute the Gittins index.
Our algorithm is built on three tools: (1) a careful characterization of Whittle index that allows one to recursively compute the $k$th smallest index from the $(k − 1)$th smallest, and to test indexability, (2) the use of the Sherman-Morrison formula to make this recursive computation efficient, and (3) a sporadic use of the fastest matrix inversion and multiplication methods to obtain a subcubic complexity. We show that an efficient use of the Sherman-Morrison formula leads to an algorithm that computes Whittle index in $(2/3)S^3 + o(S^3)$ arithmetic operations, where $S$ is the number of states of the arm. The careful use of fast matrix multiplication leads to the first subcubic algorithm to compute Whittle or Gittins index: By using the current fastest matrix multiplication, the theoretical complexity of our algorithm is $\landauO(S^{2.5286})$. We also develop an efficient implementation of our algorithm in \texttt{python} programming language. It can compute indices of Markov chains with several thousands of states in less than a few seconds.

For learning setup, we divide our work into two phases: (1) design algorithms with learning performance guarantee in rested Markovian bandits and (2) study the challenges when learning in restless Markovian bandit.
For rested bandits, Gittins index policy has been proven to be optimal when exactly one arm is activated at each decision and there is a discount on reward.
We show that MB-PSRL and MB-UCBVI, respectively the modified versions of PSRL and UCBVI, can leverage Gittins index policy to have a regret bound, which is a bound on the learning performance, and a runtime scalable in the number of arms. We also show that MB-UCRL2, a modified version of UCRL2, also has a regret bound scalable in the number of arms. Yet, we give an example showing that MB-UCRL2 and any modification of UCRL2’s variants to rested bandit likely have a runtime exponential in the number of arms.
For learning in restless bandit with long-term average reward criterion, the regret of learning algorithms depends heavily on the structure of the restless bandit.
So, we study how the structure of arms translate in the structure of the bandit. We provide a few examples showing that no learning algorithms can perform uniformly well over the general class of restless bandits.
Our examples also show that defining a subclass of restless Markovian bandits that have desirable structure for learning by relying on the assumption on arms is difficult.

%in which each arm has an internal state that evolves in a Markovian manner depending on the decision maker's actions. 


\clearpage
% -----------------------------------------------------------------------------

\begin{otherlanguage}{french}

\section*{Résumé}

%Les bandits à bras multiples sont des problèmes d'allocation séquentielle dans lesquels un sous-ensemble de bras doivent être activés à chaque instant de décision.
Un bandit markovien est un problème de décision séquentielle dans lequel un sous-ensemble de bras doivent être activés à chaque instant, et les bras évoluent de manière markovienne.
Il y a deux catégories de bandits markoviens. 
%: les bandits \emph{avec repos} dans lesquels les bras qui ne sont pas activés restent figés, et les bandits \emph{sans repos} dans lesquels 
%Les bandits markoviens sont une sous-classe de problèmes de  où l'on doit activer un sous-ensemble de bras à chaque instant de décision.
%Les bras activés évoluent de manière markovienne et active.
Si les bras qui ne sont pas activés restent figés, on entre alors dans la catégorie des bandits markoviens \emph{avec repos}.
S'ils évoluent de manière markovienne, on parle alors de bandit markovien \emph{sans repos}.
En général, les bandits markoviens souffrent de la malédiction de la dimension qui rend souvent la solution exacte prohibitive en terme de calculs.
Il faut donc recourir à des heuristiques telles que les politiques d'indice.
Deux indices célèbres sont l'indice de Gittins pour les bandits avec repos et l'indice de Whittle pour les bandits sans repos.

Cette thèse se concentre sur deux questions : (1) le calcul d'indices lorsque tous les paramètres du modèle sont connus et (2) les algorithmes d'apprentissage lorsque les paramètres sont inconnus.

Pour le calcul de l'indice, nous relevons les ambiguïtés de la définition classique de l'indexabilité et proposons une définition qui assure l'unicité de l'indice de Whittle quand ce dernier existe.
Nous développons ensuite un algorithme testant l'indexabilité et calculant les indices de Whittle.
La complexité théorique de notre algorithme est $\landauO(S^{2.5286})$, où $S$ est le nombre d'états du bras.

Pour l'apprentissage dans les bandits avec repos, nous montrons que MB-PSRL et MB-UCBVI, des versions modifiées des algorithmes PSRL et UCBVI, peuvent tirer parti de la politique d'indice de Gittins pour avoir une garantie de regret et un temps d'exécution qui passent à l'échelle avec le nombre de bras.
De plus, nous montrons que MB-UCRL2, une version modifiée de UCRL2, possède également une garantie de regret qui passe à l'échelle.
Cependant, MB-UCRL2 a un temps d'exécution exponentiel dans le nombre de bras.
Lors de l'apprentissage dans les bandits sans repos, la garantie de regret dépend fortement de la structure du bandit. Ainsi, nous étudions comment la structure des bras se traduit dans la structure du bandit.
Nous exposons une sous-classe de bandits sans repos qui ne sont pas apprenables.
%Nous montrons également qu'il est difficile de construire une sous-classe de bandits sans repos apprenables efficacement, en ne faisant que des hypothèses sur les bras.
Nous montrons également qu'il est difficile de construire des hypothèses sur les bras qui rendent les bandits sans repos apprenables efficacement.


\end{otherlanguage}
 %%%%%%%%%%%%%%%%%%%%%%%%%%%%%%%%%%%%%%%%%%%%%%%%%%%%%%%%%%%%%  COMMENTED OUT FOR SEPARATE CHAPTER COMPILATIONS
\cleardoublepage %%%%%%%%%%%%%%%%%%%%%%%%%%%%%%%%%%%%%%%%%%%%%%%%%%%%%%%%%%%%%%%%%%%%%%%  COMMENTED OUT FOR SEPARATE CHAPTER COMPILATIONS

\setcounter{tocdepth}{2}  % define depth of ToC
\phantomsection\addcontentsline{toc}{chapter}{{\contentsname}}  % display ToC to ToC

% Add a bit of space in the ToC
\makeatletter
\pretocmd{\chapter}{\addtocontents{toc}{\protect\addvspace{15\p@}}}{}{}
\pretocmd{\section}{\addtocontents{toc}{\protect\addvspace{5\p@}}}{}{}
\makeatother

\tableofcontents %%%%%%%%%%%%%%%%%%%%%%%%%%%%%%%%%%%%%%%%%%%%%%%%%%%%%%%%%%%%%%%%%%%%%%%  COMMENTED OUT FOR SEPARATE CHAPTER COMPILATIONS
\cleardoublepage %%%%%%%%%%%%%%%%%%%%%%%%%%%%%%%%%%%%%%%%%%%%%%%%%%%%%%%%%%%%%%%%%%%%%%%  COMMENTED OUT FOR SEPARATE CHAPTER COMPILATIONS
\fi

% --- MAIN MATTER ----------------------

\pagenumbering{arabic}  % arabic page numbering (e.g., 1 2 3), reset counter
\pagestyle{scrheadings}  % display header and footer
\addtocontents{toc}{\bigskip}  % visual hint for numbering change in ToC


%%%%%%%%%%%%%%%%%%%%%%%%%%%%%%%%%%%%%%%%%%%%%%%%%%%%%%%%%%
%%%%%%%%%%%%%%%%%%%%%%%%%%%%%%%%%%%%%%%%%%%%%%%%%%%%%%%%%%
% main chapters go here
\bookmarksetup{startatroot}
\inputchapter{\includechapterintroduction}{chapter_introduction.tex}
%TODO: should we put number for introduction and conclusion chapters?

\iftotalcompilation
\part{Background}
\fi

\inputchapter{\includemdpchapter}{markovDecisionProcess_chapter.tex}

\inputchapter{\includereinforcementlearningchapter}{reinforcementLearning_chapter.tex}

\iftotalcompilation
\part{Indexability}
\fi

\inputchapter{\includemarkovianbanditchapter}{markovianBandit_chapter.tex}

\inputchapter{\includeindexComputationchapter}{indexComputation_chapter.tex}

%need to add a conclusion

\iftotalcompilation
\part{Learning in Markovian bandits}
\fi

%\bookmarksetup{startatroot}

\inputchapter{\includelearningAlgorithmsRestedMBchapter}{learningAlgorithmsRestedMB_chapter.tex}

\inputchapter{\includelearningAlgorithmsRestlessMBchapter}{learningAlgorithmsRestlessMB_chapter.tex}

\bookmarksetup{startatroot}  % https://stackoverflow.com/questions/1483396/how-to-explicitly-end-a-part-in-latex-with-hyperref
\addtocontents{toc}{\bigskip}
\inputchapter{\includechapterconclusion}{chapter_conclusion.tex}
%%%%%%%%%%%%%%%%%%%%%%%%%%%%%%%%%%%%%%%%%%%%%%%%%%%%%%%%%%
%%%%%%%%%%%%%%%%%%%%%%%%%%%%%%%%%%%%%%%%%%%%%%%%%%%%%%%%%%


% --- BACK MATTER ----------------------

\iftotalcompilation
\pdfbookmark[-1]{Back Matter}{Back Matter}

\pagenumbering{arabic}  % arabic page numbering (e.g., 1 2 3), reset counter
\renewcommand*{\thepage}{A\arabic{page}}  % prepend A to appendix page number
\pagestyle{scrheadings}  % display header and footer
\addtocontents{toc}{\bigskip}  % visual hint for numbering change in ToC
\fi

\appendix
\inputchapter{\includechapterappendix}{chapter_appendix.tex}
\inputchapter{\includechapterappendixtwo}{chapter_appendixtwo.tex}

\iftotalcompilation
\cleardoublepage %%%%%%%%%%%%%%%%%%%%%%%%%%%%%%%%%%%%%%%%%%%%%%%%%%%%%%%%%%%%%%%%%%%%%%%  COMMENTED OUT FOR SEPARATE CHAPTER COMPILATIONS

{
    \setstretch{1.1}
    \renewcommand{\bibfont}{\normalfont\small}
    \setlength{\biblabelsep}{0pt}
    \setlength{\bibitemsep}{0.5\baselineskip plus 0.5\baselineskip}
    \printbibliography
}

{\listoffigures \let\cleardoublepage\  \listoftables} %%%%%%%%%%%%%%%%%%%%%%%%%%%%%%%%%%  COMMENTED OUT FOR SEPARATE CHAPTER COMPILATIONS

\cleardoublepage %%%%%%%%%%%%%%%%%%%%%%%%%%%%%%%%%%%%%%%%%%%%%%%%%%%%%%%%%%%%%%%%%%%%%%%  COMMENTED OUT FOR SEPARATE CHAPTER COMPILATIONS



 %%%%%%%%%%%%%%%%%%%%%%%%%%%%%%%%%%%%%%%%%% List of publications %%%%%%%%%%%%%%%%%
% {
%     \clearpage
%     Additionally, the work conducted in this dissertation directly led to the
%     following communications.

%     \begin{refsection}
%     \makeatletter\@beginparpenalty=10000\makeatother  % prevent page break before list
%     \defbibenvironment{itembib}{\itemize}{\enditemize}{\item}
%     \nocite{*}
%     %
%     \paragraph{Peer-reviewed international conferences}
%     \printbibliography[heading=none,env=itembib,keyword={own},keyword={conference}]
%     %
%     \paragraph{Peer-reviewed international workshops}
%     \printbibliography[heading=none,env=itembib,keyword={own},keyword={workshop}]
%     \paragraph{National workshops}
%     \printbibliography[heading=none,env=itembib,keyword={own},keyword={nworkshop}]
%     \end{refsection}
% }
%%%%%%%%%%%%%%%%%%%%%%%%%%%%%%%%%%%%%%%%%% List of publications %%%%%%%%%%%%%%%%%


\cleardoublepage %%%%%%%%%%%%%%%%%%%%%%%%%%%%%%%%%%%%%%%%%%%%%%%%%%%%%%%%%%%%%%%%%%%%%%%  COMMENTED OUT FOR SEPARATE CHAPTER COMPILATIONS

% --- BACK COVER -----------------------

\pagenumbering{gobble}  % do not count any more
\pagestyle{empty}  % no header nor footer

\cleardoubleevenpage  % ensure even page for back cover %%%%%%%%%%%%%%%%%%%%%%%%%%%%%%%%%  COMMENTED OUT FOR SEPARATE CHAPTER COMPILATIONS
\areaset[0pt]{\paperwidth}{\paperheight}  % hack to reset margins
%
\newlength{\bcmargin}\setlength{\bcmargin}{1.4cm}  % back cover margin length
%
\centering
%
\null\vspace*{\dimexpr\bcmargin-\headsep\relax}
%
\begin{minipage}{\dimexpr\paperwidth-\bcmargin-\bcmargin\relax}
%
\pdfbookmark[0]{Back Cover}{Back Cover}
%
\noindent{\usekomafont{section}Abstract}\par
%
Markovian bandits are a subclass of multi-armed bandit problems where a decision maker has to activate a set of arms at each decision time.
The activated arms evolve in an active Markovian manner.
Those that are not activated (\ie, are passive) either remain frozen -- then one falls into the category of rested Markovian bandits -- or evolve in a passive Markovian manner -- the setting is then called restless Markovian bandit.
Each arm generates a reward depending on its state and the active or passive evolution.
The decision maker wants to maximize its cumulative reward over an infinite horizon of time.
Such problems suffer from the curse of dimensionality that often makes the exact solution computationally prohibitive.
So, one has to resort to heuristics such as index policy.
Two celebrated index definitions are Gittins index for rested bandits and Whittle index for restless bandits.

In this thesis, we focus on two setups: (1) index computation when all model parameters are known and (2) learning algorithm design when the parameters are unknown.

For index computation, we first cover the ambiguities in the classical condition known as the indexability that guarantees the existence of the Whittle index in restless bandits.
We then introduce a new univocal definition of indexability that assures the uniqueness of the Whittle index when it exists.
We then develop an algorithm to test such the indexability and compute the Whittle indices of any finite-state restless bandit arm.
This algorithm can also compute the Gittins index.
Our algorithm is built on three tools: (1) a careful characterization of Whittle index that allows one to recursively compute the $k$th smallest index from the $(k − 1)$th smallest, and to test indexability, (2) the use of the Sherman-Morrison formula to make this recursive computation efficient, and (3) a sporadic use of the fastest matrix inversion and multiplication methods to obtain a subcubic complexity. We show that an efficient use of the Sherman-Morrison formula leads to an algorithm that computes Whittle index in $(2/3)S^3 + o(S^3)$ arithmetic operations, where $S$ is the number of states of the arm. The careful use of fast matrix multiplication leads to the first subcubic algorithm to compute Whittle or Gittins index: By using the current fastest matrix multiplication, the theoretical complexity of our algorithm is $\landauO(S^{2.5286})$. We also develop an efficient implementation of our algorithm in \texttt{python} programming language. It can compute indices of Markov chains with several thousands of states in less than a few seconds.

For learning setup, we divide our work into two phases: (1) design algorithms with learning performance guarantee in rested Markovian bandits and (2) study the challenges when learning in restless Markovian bandit.
For rested bandits, Gittins index policy has been proven to be optimal when exactly one arm is activated at each decision and there is a discount on reward.
We show that MB-PSRL and MB-UCBVI, respectively the modified versions of PSRL and UCBVI, can leverage Gittins index policy to have a regret bound, which is a bound on the learning performance, and a runtime scalable in the number of arms. We also show that MB-UCRL2, a modified version of UCRL2, also has a regret bound scalable in the number of arms. Yet, we give an example showing that MB-UCRL2 and any modification of UCRL2’s variants to rested bandit likely have a runtime exponential in the number of arms.
For learning in restless bandit with long-term average reward criterion, the regret of learning algorithms depends heavily on the structure of the restless bandit.
So, we study how the structure of arms translate in the structure of the bandit. We provide a few examples showing that no learning algorithms can perform uniformly well over the general class of restless bandits.
Our examples also show that defining a subclass of restless Markovian bandits that have desirable structure for learning by relying on the assumption on arms is difficult.

%in which each arm has an internal state that evolves in a Markovian manner depending on the decision maker's actions. 

%
\vspace{3ex}\hrule\vspace{2ex}
%
%\begin{otherlanguage}{french}
%%
%\noindent{\usekomafont{section}Résumé}\par
%%
%%Les bandits à bras multiples sont des problèmes d'allocation séquentielle dans lesquels un sous-ensemble de bras doivent être activés à chaque instant de décision.
Un bandit markovien est un problème de décision séquentielle dans lequel un sous-ensemble de bras doivent être activés à chaque instant, et les bras évoluent de manière markovienne.
Il y a deux catégories de bandits markoviens. 
%: les bandits \emph{avec repos} dans lesquels les bras qui ne sont pas activés restent figés, et les bandits \emph{sans repos} dans lesquels 
%Les bandits markoviens sont une sous-classe de problèmes de  où l'on doit activer un sous-ensemble de bras à chaque instant de décision.
%Les bras activés évoluent de manière markovienne et active.
Si les bras qui ne sont pas activés restent figés, on entre alors dans la catégorie des bandits markoviens \emph{avec repos}.
S'ils évoluent de manière markovienne, on parle alors de bandit markovien \emph{sans repos}.
En général, les bandits markoviens souffrent de la malédiction de la dimension qui rend souvent la solution exacte prohibitive en terme de calculs.
Il faut donc recourir à des heuristiques telles que les politiques d'indice.
Deux indices célèbres sont l'indice de Gittins pour les bandits avec repos et l'indice de Whittle pour les bandits sans repos.

Cette thèse se concentre sur deux questions : (1) le calcul d'indices lorsque tous les paramètres du modèle sont connus et (2) les algorithmes d'apprentissage lorsque les paramètres sont inconnus.

Pour le calcul de l'indice, nous relevons les ambiguïtés de la définition classique de l'indexabilité et proposons une définition qui assure l'unicité de l'indice de Whittle quand ce dernier existe.
Nous développons ensuite un algorithme testant l'indexabilité et calculant les indices de Whittle.
La complexité théorique de notre algorithme est $\landauO(S^{2.5286})$, où $S$ est le nombre d'états du bras.

Pour l'apprentissage dans les bandits avec repos, nous montrons que MB-PSRL et MB-UCBVI, des versions modifiées des algorithmes PSRL et UCBVI, peuvent tirer parti de la politique d'indice de Gittins pour avoir une garantie de regret et un temps d'exécution qui passent à l'échelle avec le nombre de bras.
De plus, nous montrons que MB-UCRL2, une version modifiée de UCRL2, possède également une garantie de regret qui passe à l'échelle.
Cependant, MB-UCRL2 a un temps d'exécution exponentiel dans le nombre de bras.
Lors de l'apprentissage dans les bandits sans repos, la garantie de regret dépend fortement de la structure du bandit. Ainsi, nous étudions comment la structure des bras se traduit dans la structure du bandit.
Nous exposons une sous-classe de bandits sans repos qui ne sont pas apprenables.
%Nous montrons également qu'il est difficile de construire une sous-classe de bandits sans repos apprenables efficacement, en ne faisant que des hypothèses sur les bras.
Nous montrons également qu'il est difficile de construire des hypothèses sur les bras qui rendent les bandits sans repos apprenables efficacement.

%%
%\end{otherlanguage}
%
\end{minipage}
 %%%%%%%%%%%%%%%%%%%%%%%%%%%%%%%%%%%%%%%%%%%%%%%%%%%%%%%%%%%%%%%%%%%%%%%  COMMENTED OUT FOR SEPARATE CHAPTER COMPILATIONS
\fi

% === END OF DOCUMENT =========================================================

\end{document}
