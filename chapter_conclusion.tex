\begingroup

\let\clearpage\relax

\chapter{Conclusions and Future Work}
\label{chapter:conclusion}


\section{Conclusions}

In this thesis, two grand questions in Markovian bandits have been addressed: index computation given the bandit's parameters and minimizing the regret using the index policy when the bandit's parameters are unknown.
For the former, we have introduced an algorithm for computing the Whittle or Gittins index in subcubic time complexity in the indexable Markovian arm. 
For the latter, we have proposed three algorithms with a regret guarantee for rested bandits with discount.
Two of the three algorithms can leverage the Gittins index.
We have also pointed out the difficulties for minimizing the regret in learning the restless bandit with the average reward critertion.

From this thesis, we have seen that the structure of arms plays a crucial role in the Markovian bandit problems.
First, when characterizing the indexability of undiscounted restless arms, the notion of indexability becomes elusive when a few optimal policies are multichain.
%This is because the indexability definition implicitly relies on the bias function of the optimal policy, and the multichain optimal policy induces bias functions that are computationally hard to characterize.
%For instance, from \cite{schweitzer1978functional}, the bias of the multichain optimal policy is defined on the basis of the policy's recurrent classes (or subchains), which are expensive to compute.
%By consequence, our algorithm cannot characterize the indexability of all multichain arms.
%For instance, given a multichain policy, one needs to compute its recurrent classes (or its subchains) characterizing the bias functions or 
%The work of \cite{schweitzer1978functional}
Second, the regret guarantee of our three algorithms in the rested bandit is sublinear in the number of arms because the three algorithms learn the arms's parameters instead of the bandit.
Lastly, when learning a restless bandit, the MDP properties, such as the diameter of the bandit, can be exponential in the number of arms.

This thesis also provides an argument that supports the power of Bayesian algorithms: They can be easily tailored to the structure of the problem to learn.
For instance, MB-PSRL has a regret guarantee and a runtime scalable in the number of arms in learning rested bandits with discount.
Also, RB-TSDE \cite{akbarzadeh2022learning} can leverage the Whittle index to have a scalable runtime in learning restless bandits with the average reward critertion.
Yet, the optimistic algorithms that use confidence bonuses on the arms' state transition are likely to have a runtime non-scalable in the number of arms in the Markovian bandit.


\section{Future work}

There are several directions to extend the work developed in this thesis. Some of them are outlined in the following.

\paragraph{Computing the Whittle index of arms with sparse transition structure}

The possibility of designing a Whittle index computation algorithm in restless Markovian arms with sparse transition matrix warrants further investigation.
Many applications in which the Whittle index policy performs exceptionally well usually admit a sparse arm's transition structure (see, \eg, \cite{wang1995finite, nino2002dynamic, aalto2018whittle}, also \cite{wang2020restless} and references therein).
On the basis of our index computation algorithm, one may want to investigate the combination of the sparse matrix inversion (see \eg, \cite{dulmage1962inversion, niessner1983computing}) with the Sherman-Morrison-Woodbury formula.
In this direction, the work of \cite{vanderbei1991splitting} investigates how the Sherman-Morrison-Woodbury formula can be used in the inversion of sparse matrix with dense columns.
It would be exciting to adapt this work in our algorithm.

\paragraph{Optimistic posterior sampling algorithms for Markovian bandits}

Theorem~\ref{thm:regret_upper_bound} provides a Bayesian regret guarantee for MB-PSRL.
Yet, it is well-known that the Bayesian regret guarantee is weaker than the worst-case regret guarantee.
Finding a prior that would allow MB-PSRL to have a worst-case regret guarantee is an exciting open question. 
Alternatively, deriving an optimistic posterior sampling version of MB-PSRL is also interesting.
Such an algorithm will have a worst-case regret guarantee.
The work of \cite{ishfaq2021randomized,agrawal2021improved,wang2020reinforcement,agrawal2017posterior} considers this idea for general MDPs.

\paragraph{Model-free algorithms for Markovian bandits}

A natural direction for future research is to extend 

\endgroup
